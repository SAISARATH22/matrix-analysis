\begin{enumerate}[label=\thesection.\arabic*,ref=\thesection.\theenumi]
\numberwithin{equation}{enumi}
\numberwithin{figure}{enumi}
\numberwithin{table}{enumi}

\item 
\item 
\item  Reduce the following equations into normal form. Find their perpendicular distances from the origin and angle between perpendicular and the positive $x$-axis.
\label{chapters/11/10/3/3}
\begin{enumerate}
	\item $x-\sqrt{3}y+8=0$ 
	\item $y-2=0$
	\item $x-y=4$
\end{enumerate}
\solution
\begin{enumerate}[label=\thesection.\arabic*,ref=\thesection.\theenumi]
\numberwithin{equation}{enumi}
\numberwithin{figure}{enumi}
\numberwithin{table}{enumi}

\item 
\item 
\item  Reduce the following equations into normal form. Find their perpendicular distances from the origin and angle between perpendicular and the positive $x$-axis.
\label{chapters/11/10/3/3}
\begin{enumerate}
	\item $x-\sqrt{3}y+8=0$ 
	\item $y-2=0$
	\item $x-y=4$
\end{enumerate}
\solution
\begin{enumerate}[label=\thesection.\arabic*,ref=\thesection.\theenumi]
\numberwithin{equation}{enumi}
\numberwithin{figure}{enumi}
\numberwithin{table}{enumi}

\item 
\item 
\item  Reduce the following equations into normal form. Find their perpendicular distances from the origin and angle between perpendicular and the positive $x$-axis.
\label{chapters/11/10/3/3}
\begin{enumerate}
	\item $x-\sqrt{3}y+8=0$ 
	\item $y-2=0$
	\item $x-y=4$
\end{enumerate}
\solution
\begin{enumerate}[label=\thesection.\arabic*,ref=\thesection.\theenumi]
\numberwithin{equation}{enumi}
\numberwithin{figure}{enumi}
\numberwithin{table}{enumi}

\item 
\item 
\item  Reduce the following equations into normal form. Find their perpendicular distances from the origin and angle between perpendicular and the positive $x$-axis.
\label{chapters/11/10/3/3}
\begin{enumerate}
	\item $x-\sqrt{3}y+8=0$ 
	\item $y-2=0$
	\item $x-y=4$
\end{enumerate}
\solution
\input{chapters/11/10/3/3/perp.tex}
\item Find the distance of the point $(-1,1)$ from the line $12\brak{x+6} = 5\brak{y-2}$. 
\label{chapters/11/10/3/4}
\input{chapters/11/10/3/4/line.tex}
\item Find the points on the x-axis, whose distances from the line $\frac{x}{3}+\frac{y}{4}=1$ are 4 units.
\label{chapters/11/10/3/5}
	\\
	\solution
\input{chapters/11/10/3/5/line.tex}
\item Find the distance between parallel lines
\label{chapters/11/10/3/6}
\begin{enumerate}
	\item $15x+8y-34=0$ and  $15x+8y+31=0$ \\
	\item  $l(x+y)+p=0$ and  $l(x+y)-r=0$
\end{enumerate}
	\solution
\input{chapters/11/10/3/6/dist.tex}
\item Find the coordinates of the foot of the perpendicular from $(-1, 3)$ to the line $3x-4y-16=0$.  
\label{chapters/11/10/3/14}
\\
\solution
\input{chapters/11/10/3/14/line.tex}
\item  If ${p}$ and ${q}$ are the lengths of perpendiculars from the origin to the lines ${x}\cos\theta - {y}\sin\theta =  {k}\cos2\theta$ and ${x}\sec\theta + {y}\cosec\theta = {k}$, respectively, prove that ${p}^2 + 4{q}^2 = {k}^2$
\label{chapters/11/10/3/16}
\\
\solution
\input{chapters/11/10/3/16/line.tex}
\item In the triangle $ABC$ with vertices $\vec{A} \brak{2, 3}$, $\vec{B} \brak{4, –1}$ and $\vec{C} \brak{1, 2}$, find the equation and length of altitude from the vertex $\vec{A}$.
\label{chapters/11/10/3/17}
\\
\solution
\input{chapters/11/10/3/17/line.tex}
\item If $p$ is the length of perpendicular from origin to the line whose intercepts on the axes are $a$ and $b$, then show that 
\begin{align}
	\frac{1}{p^2} = \frac{1}{a^2}+ \frac{1}{b^2}
\end{align}
\label{chapters/11/10/3/18}
\input{chapters/11/10/3/18/dist.tex}
\item What are the points on the y-axis whose distance from the line $\frac{x}{3}+\frac{y}{4}=1$ is 4 units.
\\
\solution
		\input{chapters/11/10/4/4/line.tex}
\item Find perpendicular distance from the origin to the line joining the points$(\cos\theta,\sin\theta)$ and $(\cos\phi,\sin\phi)$.
\\
\solution
		\input{chapters/11/10/4/5/dist.tex}
\item Find the equation of line which is equidistant from parallel lines $9x+6y-7=0$ and $3x+2y+6=0$.
\\
\solution
		\input{chapters/11/10/4/21/line.tex}
	\item Prove that the products of the lengths of the perpendiculars drawn from the points $\myvec{\sqrt{a^2-b^2}\\0}$ and $\myvec{-\sqrt{a^2-b^2} \\0} $ to the line $\frac{x}{a} \cos{\theta} + \frac{y}{b}\sin{\theta} =1 $ is $ b^2 $.
\\
    \solution 
		\input{chapters/11/10/4/23/dist.tex}
\item Find the equation of line  drawn perpendicular to the line $\frac{x}{4}+\frac{y}{6}=1$ through the point where it meets the y-axis \\
\solution
		\input{chapters/11/10/4/7/line.tex}
 \item  In each of the following cases, determine the direction cosines of the normal to
the plane and the distance from the origin.
\begin{enumerate}
	\item $z=2$ 
	\item $x + y + z = 1$
	\item $2x + 3y – z = 5$
	\item $5y + 8 = 0$
\end{enumerate}
    \solution
		\input{chapters/12/11/3/1/dist.tex}
\item
\input{chapters/12/11/4/3/vec4.tex}
\end{enumerate}

\item Find the distance of the point $(-1,1)$ from the line $12\brak{x+6} = 5\brak{y-2}$. 
\label{chapters/11/10/3/4}
\iffalse
\documentclass[journal,12pt,twocolumn]{IEEEtran}
\usepackage{setspace}
\usepackage{gensymb}
\usepackage{xcolor}
\usepackage{caption}
\singlespacing
\usepackage{siunitx}
\usepackage[cmex10]{amsmath}
\usepackage{mathtools}
\usepackage{hyperref}
\usepackage{amsthm}
\usepackage{mathrsfs}
\usepackage{txfonts}
\usepackage{stfloats}
\usepackage{cite}
\usepackage{cases}
\usepackage{subfig}
\usepackage{longtable}
\usepackage{multirow}
\usepackage{enumitem}
\usepackage{mathtools}
\usepackage{listings}
\usepackage{tikz}
\usetikzlibrary{shapes,arrows,positioning}
\usepackage{circuitikz}
\let\vec\mathbf
\DeclareMathOperator*{\Res}{Res}
\renewcommand\thesection{\arabic{section}}
\renewcommand\thesubsection{\thesection.\arabic{subsection}}
\renewcommand\thesubsubsection{\thesubsection.\arabic{subsubsection}}

\renewcommand\thesectiondis{\arabic{section}}
\renewcommand\thesubsectiondis{\thesectiondis.\arabic{subsection}}
\renewcommand\thesubsubsectiondis{\thesubsectiondis.\arabic{subsubsection}}
\hyphenation{op-tical net-works semi-conduc-tor}

\lstset{
language=Python,
frame=single, 
breaklines=true,
columns=fullflexible
}
\begin{document}
\theoremstyle{definition}
\newtheorem{theorem}{Theorem}[section]
\newtheorem{problem}{Problem}
\newtheorem{proposition}{Proposition}[section]
\newtheorem{lemma}{Lemma}[section]
\newtheorem{corollary}[theorem]{Corollary}
\newtheorem{example}{Example}[section]
\newtheorem{definition}{Definition}[section]
\newcommand{\BEQA}{\begin{eqnarray}}
\newcommand{\EEQA}{\end{eqnarray}}
\newcommand{\define}{\stackrel{\triangle}{=}}
\newenvironment{amatrix}[1]{%
  \left(\begin{array}{@{}*{#1}{c}|c@{}}
}{%
  \end{array}\right)
}
\newcommand{\myvec}[1]{\ensuremath{\begin{pmatrix}#1\end{pmatrix}}}
\newcommand{\myaugvec}[2]{\ensuremath{\begin{amatrix}{#1}#2\end{amatrix}}}
\newcommand{\mydet}[1]{\ensuremath{\begin{vmatrix}#1\end{vmatrix}}}
\bibliographystyle{IEEEtran}
\providecommand{\nCr}[2]{\,^{#1}C_{#2}} % nCr
\providecommand{\nPr}[2]{\,^{#1}P_{#2}} % nPr
\providecommand{\mbf}{\mathbf}
\providecommand{\pr}[1]{\ensuremath{\Pr\left(#1\right)}}
\providecommand{\qfunc}[1]{\ensuremath{Q\left(#1\right)}}
\providecommand{\sbrak}[1]{\ensuremath{{}\left[#1\right]}}
\providecommand{\lsbrak}[1]{\ensuremath{{}\left[#1\right.}}
\providecommand{\rsbrak}[1]{\ensuremath{{}\left.#1\right]}}
\providecommand{\brak}[1]{\ensuremath{\left(#1\right)}}
\providecommand{\lbrak}[1]{\ensuremath{\left(#1\right.}}
\providecommand{\rbrak}[1]{\ensuremath{\left.#1\right)}}
\providecommand{\cbrak}[1]{\ensuremath{\left\{#1\right\}}}
\providecommand{\lcbrak}[1]{\ensuremath{\left\{#1\right.}}
\providecommand{\rcbrak}[1]{\ensuremath{\left.#1\right\}}}
\theoremstyle{remark}
\newtheorem{rem}{Remark}
\newcommand{\sgn}{\mathop{\mathrm{sgn}}}
\newcommand{\rect}{\mathop{\mathrm{rect}}}
\newcommand{\sinc}{\mathop{\mathrm{sinc}}}
\providecommand{\abs}[1]{\left\vert#1\right\vert}
\providecommand{\res}[1]{\Res\displaylimits_{#1}} 
\providecommand{\norm}[1]{\left\Vert#1\right\Vert}
\providecommand{\mtx}[1]{\mathbf{#1}}
\providecommand{\mean}[1]{E\left[ #1 \right]}
\providecommand{\fourier}{\overset{\mathcal{F}}{ \rightleftharpoons}}
\providecommand{\ztrans}{\overset{\mathcal{Z}}{ \rightleftharpoons}}
\providecommand{\system}[1]{\overset{\mathcal{#1}}{ \longleftrightarrow}}
\newcommand{\solution}{\noindent \textbf{Solution: }}
\providecommand{\dec}[2]{\ensuremath{\overset{#1}{\underset{#2}{\gtrless}}}}
\let\StandardTheFigure\thefigure
\def\putbox#1#2#3{\makebox[0in][l]{\makebox[#1][l]{}\raisebox{\baselineskip}[0in][0in]{\raisebox{#2}[0in][0in]{#3}}}}
     \def\rightbox#1{\makebox[0in][r]{#1}}
     \def\centbox#1{\makebox[0in]{#1}}
     \def\topbox#1{\raisebox{-\baselineskip}[0in][0in]{#1}}
     \def\midbox#1{\raisebox{-0.5\baselineskip}[0in][0in]{#1}}

\vspace{3cm}
\title{Line Assignment}
\author{Gautam Singh}
\maketitle
\bigskip

\begin{abstract}
    This document contains the solution to Question 24 of Exercise 4 
    in Chapter 10 of the class 11 NCERT textbook.
\end{abstract}

\begin{enumerate}
\fi
		We first find the coordinates of the intersection of \eqref{eq:chapters/11/10/4/24/L1}
    and \eqref{eq:chapters/11/10/4/24/L2}. Using the augmented matrix and row reduction methods,
    \begin{align}
        \myaugvec{2}{2&-3&-4\\3&4&5} &\xleftrightarrow[]{R_2\rightarrow2R_2-3R_1} 
        \myaugvec{2}{2&-3&-4\\0&17&22} \\
                      &\xleftrightarrow[]{R_1\rightarrow17R_1+3R_2} \myaugvec{2}{17&0&-1\\0&17&22} \\
                      &\xleftrightarrow[]{\substack{R_1\rightarrow\frac{R_1}{17}\\R_2\rightarrow\frac{R_2}{17}}} \myaugvec{2}{1&0&-\frac{1}{17}\\0&1&\frac{22}{17}}
        \label{eq:chapters/11/10/4/24/intersect}
    \end{align}
    the intersection of the lines is
    \begin{align}
        \vec{A} = \frac{1}{17}\myvec{-1\\22}
    \end{align}
    Clearly, the man should follow the path perpendicular to \eqref{eq:chapters/11/10/4/24/L3} from
    $\vec{A}$ to reach it in the shortest time. The normal vector 
    of \eqref{eq:chapters/11/10/4/24/L3} is 
    \begin{align}
        \vec{m} = \myvec{6\\-7}
        \label{eq:chapters/11/10/4/24/L3-norm}
    \end{align}
    which is consequently the direction vector of the required line. Therefore, 
    the required normal vector is given by
    \begin{align}
        \vec{n} = \myvec{7\\6}
        \label{eq:chapters/11/10/4/24/L4-norm}
    \end{align}
    and hence, the equation of the line is
   \begin{align}
        \vec{n}^\top\vec{x} &= \vec{n}^\top\vec{A} \\
        \implies \myvec{7&6}\vec{x} &= \frac{1}{17}\myvec{7&6}\myvec{-1\\22} = \frac{125}{17}
        \label{eq:chapters/11/10/4/24/L4}
    \end{align}
		See Fig. \ref{fig:chapters/11/10/4/24/crossing}. In this figure $\vec{F}$ represents 
    the foot of the prependicular drawn from $\vec{A}$ onto \eqref{eq:chapters/11/10/4/24/L3}.

\item Find the points on the x-axis, whose distances from the line $\frac{x}{3}+\frac{y}{4}=1$ are 4 units.
\label{chapters/11/10/3/5}
	\\
	\solution
\iffalse
\documentclass[journal,12pt,twocolumn]{IEEEtran}
\usepackage{setspace}
\usepackage{gensymb}
\usepackage{xcolor}
\usepackage{caption}
\singlespacing
\usepackage{siunitx}
\usepackage[cmex10]{amsmath}
\usepackage{mathtools}
\usepackage{hyperref}
\usepackage{amsthm}
\usepackage{mathrsfs}
\usepackage{txfonts}
\usepackage{stfloats}
\usepackage{cite}
\usepackage{cases}
\usepackage{subfig}
\usepackage{longtable}
\usepackage{multirow}
\usepackage{enumitem}
\usepackage{mathtools}
\usepackage{listings}
\usepackage{tikz}
\usetikzlibrary{shapes,arrows,positioning}
\usepackage{circuitikz}
\let\vec\mathbf
\DeclareMathOperator*{\Res}{Res}
\renewcommand\thesection{\arabic{section}}
\renewcommand\thesubsection{\thesection.\arabic{subsection}}
\renewcommand\thesubsubsection{\thesubsection.\arabic{subsubsection}}

\renewcommand\thesectiondis{\arabic{section}}
\renewcommand\thesubsectiondis{\thesectiondis.\arabic{subsection}}
\renewcommand\thesubsubsectiondis{\thesubsectiondis.\arabic{subsubsection}}
\hyphenation{op-tical net-works semi-conduc-tor}

\lstset{
language=Python,
frame=single, 
breaklines=true,
columns=fullflexible
}
\begin{document}
\theoremstyle{definition}
\newtheorem{theorem}{Theorem}[section]
\newtheorem{problem}{Problem}
\newtheorem{proposition}{Proposition}[section]
\newtheorem{lemma}{Lemma}[section]
\newtheorem{corollary}[theorem]{Corollary}
\newtheorem{example}{Example}[section]
\newtheorem{definition}{Definition}[section]
\newcommand{\BEQA}{\begin{eqnarray}}
\newcommand{\EEQA}{\end{eqnarray}}
\newcommand{\define}{\stackrel{\triangle}{=}}
\newenvironment{amatrix}[1]{%
  \left(\begin{array}{@{}*{#1}{c}|c@{}}
}{%
  \end{array}\right)
}
\newcommand{\myvec}[1]{\ensuremath{\begin{pmatrix}#1\end{pmatrix}}}
\newcommand{\myaugvec}[2]{\ensuremath{\begin{amatrix}{#1}#2\end{amatrix}}}
\newcommand{\mydet}[1]{\ensuremath{\begin{vmatrix}#1\end{vmatrix}}}
\bibliographystyle{IEEEtran}
\providecommand{\nCr}[2]{\,^{#1}C_{#2}} % nCr
\providecommand{\nPr}[2]{\,^{#1}P_{#2}} % nPr
\providecommand{\mbf}{\mathbf}
\providecommand{\pr}[1]{\ensuremath{\Pr\left(#1\right)}}
\providecommand{\qfunc}[1]{\ensuremath{Q\left(#1\right)}}
\providecommand{\sbrak}[1]{\ensuremath{{}\left[#1\right]}}
\providecommand{\lsbrak}[1]{\ensuremath{{}\left[#1\right.}}
\providecommand{\rsbrak}[1]{\ensuremath{{}\left.#1\right]}}
\providecommand{\brak}[1]{\ensuremath{\left(#1\right)}}
\providecommand{\lbrak}[1]{\ensuremath{\left(#1\right.}}
\providecommand{\rbrak}[1]{\ensuremath{\left.#1\right)}}
\providecommand{\cbrak}[1]{\ensuremath{\left\{#1\right\}}}
\providecommand{\lcbrak}[1]{\ensuremath{\left\{#1\right.}}
\providecommand{\rcbrak}[1]{\ensuremath{\left.#1\right\}}}
\theoremstyle{remark}
\newtheorem{rem}{Remark}
\newcommand{\sgn}{\mathop{\mathrm{sgn}}}
\newcommand{\rect}{\mathop{\mathrm{rect}}}
\newcommand{\sinc}{\mathop{\mathrm{sinc}}}
\providecommand{\abs}[1]{\left\vert#1\right\vert}
\providecommand{\res}[1]{\Res\displaylimits_{#1}} 
\providecommand{\norm}[1]{\left\Vert#1\right\Vert}
\providecommand{\mtx}[1]{\mathbf{#1}}
\providecommand{\mean}[1]{E\left[ #1 \right]}
\providecommand{\fourier}{\overset{\mathcal{F}}{ \rightleftharpoons}}
\providecommand{\ztrans}{\overset{\mathcal{Z}}{ \rightleftharpoons}}
\providecommand{\system}[1]{\overset{\mathcal{#1}}{ \longleftrightarrow}}
\newcommand{\solution}{\noindent \textbf{Solution: }}
\providecommand{\dec}[2]{\ensuremath{\overset{#1}{\underset{#2}{\gtrless}}}}
\let\StandardTheFigure\thefigure
\def\putbox#1#2#3{\makebox[0in][l]{\makebox[#1][l]{}\raisebox{\baselineskip}[0in][0in]{\raisebox{#2}[0in][0in]{#3}}}}
     \def\rightbox#1{\makebox[0in][r]{#1}}
     \def\centbox#1{\makebox[0in]{#1}}
     \def\topbox#1{\raisebox{-\baselineskip}[0in][0in]{#1}}
     \def\midbox#1{\raisebox{-0.5\baselineskip}[0in][0in]{#1}}

\vspace{3cm}
\title{Line Assignment}
\author{Gautam Singh}
\maketitle
\bigskip

\begin{abstract}
    This document contains the solution to Question 24 of Exercise 4 
    in Chapter 10 of the class 11 NCERT textbook.
\end{abstract}

\begin{enumerate}
\fi
		We first find the coordinates of the intersection of \eqref{eq:chapters/11/10/4/24/L1}
    and \eqref{eq:chapters/11/10/4/24/L2}. Using the augmented matrix and row reduction methods,
    \begin{align}
        \myaugvec{2}{2&-3&-4\\3&4&5} &\xleftrightarrow[]{R_2\rightarrow2R_2-3R_1} 
        \myaugvec{2}{2&-3&-4\\0&17&22} \\
                      &\xleftrightarrow[]{R_1\rightarrow17R_1+3R_2} \myaugvec{2}{17&0&-1\\0&17&22} \\
                      &\xleftrightarrow[]{\substack{R_1\rightarrow\frac{R_1}{17}\\R_2\rightarrow\frac{R_2}{17}}} \myaugvec{2}{1&0&-\frac{1}{17}\\0&1&\frac{22}{17}}
        \label{eq:chapters/11/10/4/24/intersect}
    \end{align}
    the intersection of the lines is
    \begin{align}
        \vec{A} = \frac{1}{17}\myvec{-1\\22}
    \end{align}
    Clearly, the man should follow the path perpendicular to \eqref{eq:chapters/11/10/4/24/L3} from
    $\vec{A}$ to reach it in the shortest time. The normal vector 
    of \eqref{eq:chapters/11/10/4/24/L3} is 
    \begin{align}
        \vec{m} = \myvec{6\\-7}
        \label{eq:chapters/11/10/4/24/L3-norm}
    \end{align}
    which is consequently the direction vector of the required line. Therefore, 
    the required normal vector is given by
    \begin{align}
        \vec{n} = \myvec{7\\6}
        \label{eq:chapters/11/10/4/24/L4-norm}
    \end{align}
    and hence, the equation of the line is
   \begin{align}
        \vec{n}^\top\vec{x} &= \vec{n}^\top\vec{A} \\
        \implies \myvec{7&6}\vec{x} &= \frac{1}{17}\myvec{7&6}\myvec{-1\\22} = \frac{125}{17}
        \label{eq:chapters/11/10/4/24/L4}
    \end{align}
		See Fig. \ref{fig:chapters/11/10/4/24/crossing}. In this figure $\vec{F}$ represents 
    the foot of the prependicular drawn from $\vec{A}$ onto \eqref{eq:chapters/11/10/4/24/L3}.

\item Find the distance between parallel lines
\label{chapters/11/10/3/6}
\begin{enumerate}
	\item $15x+8y-34=0$ and  $15x+8y+31=0$ \\
	\item  $l(x+y)+p=0$ and  $l(x+y)-r=0$
\end{enumerate}
	\solution
\input{chapters/11/10/3/6/dist.tex}
\item Find the coordinates of the foot of the perpendicular from $(-1, 3)$ to the line $3x-4y-16=0$.  
\label{chapters/11/10/3/14}
\\
\solution
\iffalse
\documentclass[journal,12pt,twocolumn]{IEEEtran}
\usepackage{setspace}
\usepackage{gensymb}
\usepackage{xcolor}
\usepackage{caption}
\singlespacing
\usepackage{siunitx}
\usepackage[cmex10]{amsmath}
\usepackage{mathtools}
\usepackage{hyperref}
\usepackage{amsthm}
\usepackage{mathrsfs}
\usepackage{txfonts}
\usepackage{stfloats}
\usepackage{cite}
\usepackage{cases}
\usepackage{subfig}
\usepackage{longtable}
\usepackage{multirow}
\usepackage{enumitem}
\usepackage{mathtools}
\usepackage{listings}
\usepackage{tikz}
\usetikzlibrary{shapes,arrows,positioning}
\usepackage{circuitikz}
\let\vec\mathbf
\DeclareMathOperator*{\Res}{Res}
\renewcommand\thesection{\arabic{section}}
\renewcommand\thesubsection{\thesection.\arabic{subsection}}
\renewcommand\thesubsubsection{\thesubsection.\arabic{subsubsection}}

\renewcommand\thesectiondis{\arabic{section}}
\renewcommand\thesubsectiondis{\thesectiondis.\arabic{subsection}}
\renewcommand\thesubsubsectiondis{\thesubsectiondis.\arabic{subsubsection}}
\hyphenation{op-tical net-works semi-conduc-tor}

\lstset{
language=Python,
frame=single, 
breaklines=true,
columns=fullflexible
}
\begin{document}
\theoremstyle{definition}
\newtheorem{theorem}{Theorem}[section]
\newtheorem{problem}{Problem}
\newtheorem{proposition}{Proposition}[section]
\newtheorem{lemma}{Lemma}[section]
\newtheorem{corollary}[theorem]{Corollary}
\newtheorem{example}{Example}[section]
\newtheorem{definition}{Definition}[section]
\newcommand{\BEQA}{\begin{eqnarray}}
\newcommand{\EEQA}{\end{eqnarray}}
\newcommand{\define}{\stackrel{\triangle}{=}}
\newenvironment{amatrix}[1]{%
  \left(\begin{array}{@{}*{#1}{c}|c@{}}
}{%
  \end{array}\right)
}
\newcommand{\myvec}[1]{\ensuremath{\begin{pmatrix}#1\end{pmatrix}}}
\newcommand{\myaugvec}[2]{\ensuremath{\begin{amatrix}{#1}#2\end{amatrix}}}
\newcommand{\mydet}[1]{\ensuremath{\begin{vmatrix}#1\end{vmatrix}}}
\bibliographystyle{IEEEtran}
\providecommand{\nCr}[2]{\,^{#1}C_{#2}} % nCr
\providecommand{\nPr}[2]{\,^{#1}P_{#2}} % nPr
\providecommand{\mbf}{\mathbf}
\providecommand{\pr}[1]{\ensuremath{\Pr\left(#1\right)}}
\providecommand{\qfunc}[1]{\ensuremath{Q\left(#1\right)}}
\providecommand{\sbrak}[1]{\ensuremath{{}\left[#1\right]}}
\providecommand{\lsbrak}[1]{\ensuremath{{}\left[#1\right.}}
\providecommand{\rsbrak}[1]{\ensuremath{{}\left.#1\right]}}
\providecommand{\brak}[1]{\ensuremath{\left(#1\right)}}
\providecommand{\lbrak}[1]{\ensuremath{\left(#1\right.}}
\providecommand{\rbrak}[1]{\ensuremath{\left.#1\right)}}
\providecommand{\cbrak}[1]{\ensuremath{\left\{#1\right\}}}
\providecommand{\lcbrak}[1]{\ensuremath{\left\{#1\right.}}
\providecommand{\rcbrak}[1]{\ensuremath{\left.#1\right\}}}
\theoremstyle{remark}
\newtheorem{rem}{Remark}
\newcommand{\sgn}{\mathop{\mathrm{sgn}}}
\newcommand{\rect}{\mathop{\mathrm{rect}}}
\newcommand{\sinc}{\mathop{\mathrm{sinc}}}
\providecommand{\abs}[1]{\left\vert#1\right\vert}
\providecommand{\res}[1]{\Res\displaylimits_{#1}} 
\providecommand{\norm}[1]{\left\Vert#1\right\Vert}
\providecommand{\mtx}[1]{\mathbf{#1}}
\providecommand{\mean}[1]{E\left[ #1 \right]}
\providecommand{\fourier}{\overset{\mathcal{F}}{ \rightleftharpoons}}
\providecommand{\ztrans}{\overset{\mathcal{Z}}{ \rightleftharpoons}}
\providecommand{\system}[1]{\overset{\mathcal{#1}}{ \longleftrightarrow}}
\newcommand{\solution}{\noindent \textbf{Solution: }}
\providecommand{\dec}[2]{\ensuremath{\overset{#1}{\underset{#2}{\gtrless}}}}
\let\StandardTheFigure\thefigure
\def\putbox#1#2#3{\makebox[0in][l]{\makebox[#1][l]{}\raisebox{\baselineskip}[0in][0in]{\raisebox{#2}[0in][0in]{#3}}}}
     \def\rightbox#1{\makebox[0in][r]{#1}}
     \def\centbox#1{\makebox[0in]{#1}}
     \def\topbox#1{\raisebox{-\baselineskip}[0in][0in]{#1}}
     \def\midbox#1{\raisebox{-0.5\baselineskip}[0in][0in]{#1}}

\vspace{3cm}
\title{Line Assignment}
\author{Gautam Singh}
\maketitle
\bigskip

\begin{abstract}
    This document contains the solution to Question 24 of Exercise 4 
    in Chapter 10 of the class 11 NCERT textbook.
\end{abstract}

\begin{enumerate}
\fi
		We first find the coordinates of the intersection of \eqref{eq:chapters/11/10/4/24/L1}
    and \eqref{eq:chapters/11/10/4/24/L2}. Using the augmented matrix and row reduction methods,
    \begin{align}
        \myaugvec{2}{2&-3&-4\\3&4&5} &\xleftrightarrow[]{R_2\rightarrow2R_2-3R_1} 
        \myaugvec{2}{2&-3&-4\\0&17&22} \\
                      &\xleftrightarrow[]{R_1\rightarrow17R_1+3R_2} \myaugvec{2}{17&0&-1\\0&17&22} \\
                      &\xleftrightarrow[]{\substack{R_1\rightarrow\frac{R_1}{17}\\R_2\rightarrow\frac{R_2}{17}}} \myaugvec{2}{1&0&-\frac{1}{17}\\0&1&\frac{22}{17}}
        \label{eq:chapters/11/10/4/24/intersect}
    \end{align}
    the intersection of the lines is
    \begin{align}
        \vec{A} = \frac{1}{17}\myvec{-1\\22}
    \end{align}
    Clearly, the man should follow the path perpendicular to \eqref{eq:chapters/11/10/4/24/L3} from
    $\vec{A}$ to reach it in the shortest time. The normal vector 
    of \eqref{eq:chapters/11/10/4/24/L3} is 
    \begin{align}
        \vec{m} = \myvec{6\\-7}
        \label{eq:chapters/11/10/4/24/L3-norm}
    \end{align}
    which is consequently the direction vector of the required line. Therefore, 
    the required normal vector is given by
    \begin{align}
        \vec{n} = \myvec{7\\6}
        \label{eq:chapters/11/10/4/24/L4-norm}
    \end{align}
    and hence, the equation of the line is
   \begin{align}
        \vec{n}^\top\vec{x} &= \vec{n}^\top\vec{A} \\
        \implies \myvec{7&6}\vec{x} &= \frac{1}{17}\myvec{7&6}\myvec{-1\\22} = \frac{125}{17}
        \label{eq:chapters/11/10/4/24/L4}
    \end{align}
		See Fig. \ref{fig:chapters/11/10/4/24/crossing}. In this figure $\vec{F}$ represents 
    the foot of the prependicular drawn from $\vec{A}$ onto \eqref{eq:chapters/11/10/4/24/L3}.

\item  If ${p}$ and ${q}$ are the lengths of perpendiculars from the origin to the lines ${x}\cos\theta - {y}\sin\theta =  {k}\cos2\theta$ and ${x}\sec\theta + {y}\cosec\theta = {k}$, respectively, prove that ${p}^2 + 4{q}^2 = {k}^2$
\label{chapters/11/10/3/16}
\\
\solution
\iffalse
\documentclass[journal,12pt,twocolumn]{IEEEtran}
\usepackage{setspace}
\usepackage{gensymb}
\usepackage{xcolor}
\usepackage{caption}
\singlespacing
\usepackage{siunitx}
\usepackage[cmex10]{amsmath}
\usepackage{mathtools}
\usepackage{hyperref}
\usepackage{amsthm}
\usepackage{mathrsfs}
\usepackage{txfonts}
\usepackage{stfloats}
\usepackage{cite}
\usepackage{cases}
\usepackage{subfig}
\usepackage{longtable}
\usepackage{multirow}
\usepackage{enumitem}
\usepackage{mathtools}
\usepackage{listings}
\usepackage{tikz}
\usetikzlibrary{shapes,arrows,positioning}
\usepackage{circuitikz}
\let\vec\mathbf
\DeclareMathOperator*{\Res}{Res}
\renewcommand\thesection{\arabic{section}}
\renewcommand\thesubsection{\thesection.\arabic{subsection}}
\renewcommand\thesubsubsection{\thesubsection.\arabic{subsubsection}}

\renewcommand\thesectiondis{\arabic{section}}
\renewcommand\thesubsectiondis{\thesectiondis.\arabic{subsection}}
\renewcommand\thesubsubsectiondis{\thesubsectiondis.\arabic{subsubsection}}
\hyphenation{op-tical net-works semi-conduc-tor}

\lstset{
language=Python,
frame=single, 
breaklines=true,
columns=fullflexible
}
\begin{document}
\theoremstyle{definition}
\newtheorem{theorem}{Theorem}[section]
\newtheorem{problem}{Problem}
\newtheorem{proposition}{Proposition}[section]
\newtheorem{lemma}{Lemma}[section]
\newtheorem{corollary}[theorem]{Corollary}
\newtheorem{example}{Example}[section]
\newtheorem{definition}{Definition}[section]
\newcommand{\BEQA}{\begin{eqnarray}}
\newcommand{\EEQA}{\end{eqnarray}}
\newcommand{\define}{\stackrel{\triangle}{=}}
\newenvironment{amatrix}[1]{%
  \left(\begin{array}{@{}*{#1}{c}|c@{}}
}{%
  \end{array}\right)
}
\newcommand{\myvec}[1]{\ensuremath{\begin{pmatrix}#1\end{pmatrix}}}
\newcommand{\myaugvec}[2]{\ensuremath{\begin{amatrix}{#1}#2\end{amatrix}}}
\newcommand{\mydet}[1]{\ensuremath{\begin{vmatrix}#1\end{vmatrix}}}
\bibliographystyle{IEEEtran}
\providecommand{\nCr}[2]{\,^{#1}C_{#2}} % nCr
\providecommand{\nPr}[2]{\,^{#1}P_{#2}} % nPr
\providecommand{\mbf}{\mathbf}
\providecommand{\pr}[1]{\ensuremath{\Pr\left(#1\right)}}
\providecommand{\qfunc}[1]{\ensuremath{Q\left(#1\right)}}
\providecommand{\sbrak}[1]{\ensuremath{{}\left[#1\right]}}
\providecommand{\lsbrak}[1]{\ensuremath{{}\left[#1\right.}}
\providecommand{\rsbrak}[1]{\ensuremath{{}\left.#1\right]}}
\providecommand{\brak}[1]{\ensuremath{\left(#1\right)}}
\providecommand{\lbrak}[1]{\ensuremath{\left(#1\right.}}
\providecommand{\rbrak}[1]{\ensuremath{\left.#1\right)}}
\providecommand{\cbrak}[1]{\ensuremath{\left\{#1\right\}}}
\providecommand{\lcbrak}[1]{\ensuremath{\left\{#1\right.}}
\providecommand{\rcbrak}[1]{\ensuremath{\left.#1\right\}}}
\theoremstyle{remark}
\newtheorem{rem}{Remark}
\newcommand{\sgn}{\mathop{\mathrm{sgn}}}
\newcommand{\rect}{\mathop{\mathrm{rect}}}
\newcommand{\sinc}{\mathop{\mathrm{sinc}}}
\providecommand{\abs}[1]{\left\vert#1\right\vert}
\providecommand{\res}[1]{\Res\displaylimits_{#1}} 
\providecommand{\norm}[1]{\left\Vert#1\right\Vert}
\providecommand{\mtx}[1]{\mathbf{#1}}
\providecommand{\mean}[1]{E\left[ #1 \right]}
\providecommand{\fourier}{\overset{\mathcal{F}}{ \rightleftharpoons}}
\providecommand{\ztrans}{\overset{\mathcal{Z}}{ \rightleftharpoons}}
\providecommand{\system}[1]{\overset{\mathcal{#1}}{ \longleftrightarrow}}
\newcommand{\solution}{\noindent \textbf{Solution: }}
\providecommand{\dec}[2]{\ensuremath{\overset{#1}{\underset{#2}{\gtrless}}}}
\let\StandardTheFigure\thefigure
\def\putbox#1#2#3{\makebox[0in][l]{\makebox[#1][l]{}\raisebox{\baselineskip}[0in][0in]{\raisebox{#2}[0in][0in]{#3}}}}
     \def\rightbox#1{\makebox[0in][r]{#1}}
     \def\centbox#1{\makebox[0in]{#1}}
     \def\topbox#1{\raisebox{-\baselineskip}[0in][0in]{#1}}
     \def\midbox#1{\raisebox{-0.5\baselineskip}[0in][0in]{#1}}

\vspace{3cm}
\title{Line Assignment}
\author{Gautam Singh}
\maketitle
\bigskip

\begin{abstract}
    This document contains the solution to Question 24 of Exercise 4 
    in Chapter 10 of the class 11 NCERT textbook.
\end{abstract}

\begin{enumerate}
\fi
		We first find the coordinates of the intersection of \eqref{eq:chapters/11/10/4/24/L1}
    and \eqref{eq:chapters/11/10/4/24/L2}. Using the augmented matrix and row reduction methods,
    \begin{align}
        \myaugvec{2}{2&-3&-4\\3&4&5} &\xleftrightarrow[]{R_2\rightarrow2R_2-3R_1} 
        \myaugvec{2}{2&-3&-4\\0&17&22} \\
                      &\xleftrightarrow[]{R_1\rightarrow17R_1+3R_2} \myaugvec{2}{17&0&-1\\0&17&22} \\
                      &\xleftrightarrow[]{\substack{R_1\rightarrow\frac{R_1}{17}\\R_2\rightarrow\frac{R_2}{17}}} \myaugvec{2}{1&0&-\frac{1}{17}\\0&1&\frac{22}{17}}
        \label{eq:chapters/11/10/4/24/intersect}
    \end{align}
    the intersection of the lines is
    \begin{align}
        \vec{A} = \frac{1}{17}\myvec{-1\\22}
    \end{align}
    Clearly, the man should follow the path perpendicular to \eqref{eq:chapters/11/10/4/24/L3} from
    $\vec{A}$ to reach it in the shortest time. The normal vector 
    of \eqref{eq:chapters/11/10/4/24/L3} is 
    \begin{align}
        \vec{m} = \myvec{6\\-7}
        \label{eq:chapters/11/10/4/24/L3-norm}
    \end{align}
    which is consequently the direction vector of the required line. Therefore, 
    the required normal vector is given by
    \begin{align}
        \vec{n} = \myvec{7\\6}
        \label{eq:chapters/11/10/4/24/L4-norm}
    \end{align}
    and hence, the equation of the line is
   \begin{align}
        \vec{n}^\top\vec{x} &= \vec{n}^\top\vec{A} \\
        \implies \myvec{7&6}\vec{x} &= \frac{1}{17}\myvec{7&6}\myvec{-1\\22} = \frac{125}{17}
        \label{eq:chapters/11/10/4/24/L4}
    \end{align}
		See Fig. \ref{fig:chapters/11/10/4/24/crossing}. In this figure $\vec{F}$ represents 
    the foot of the prependicular drawn from $\vec{A}$ onto \eqref{eq:chapters/11/10/4/24/L3}.

\item In the triangle $ABC$ with vertices $\vec{A} \brak{2, 3}$, $\vec{B} \brak{4, –1}$ and $\vec{C} \brak{1, 2}$, find the equation and length of altitude from the vertex $\vec{A}$.
\label{chapters/11/10/3/17}
\\
\solution
\iffalse
\documentclass[journal,12pt,twocolumn]{IEEEtran}
\usepackage{setspace}
\usepackage{gensymb}
\usepackage{xcolor}
\usepackage{caption}
\singlespacing
\usepackage{siunitx}
\usepackage[cmex10]{amsmath}
\usepackage{mathtools}
\usepackage{hyperref}
\usepackage{amsthm}
\usepackage{mathrsfs}
\usepackage{txfonts}
\usepackage{stfloats}
\usepackage{cite}
\usepackage{cases}
\usepackage{subfig}
\usepackage{longtable}
\usepackage{multirow}
\usepackage{enumitem}
\usepackage{mathtools}
\usepackage{listings}
\usepackage{tikz}
\usetikzlibrary{shapes,arrows,positioning}
\usepackage{circuitikz}
\let\vec\mathbf
\DeclareMathOperator*{\Res}{Res}
\renewcommand\thesection{\arabic{section}}
\renewcommand\thesubsection{\thesection.\arabic{subsection}}
\renewcommand\thesubsubsection{\thesubsection.\arabic{subsubsection}}

\renewcommand\thesectiondis{\arabic{section}}
\renewcommand\thesubsectiondis{\thesectiondis.\arabic{subsection}}
\renewcommand\thesubsubsectiondis{\thesubsectiondis.\arabic{subsubsection}}
\hyphenation{op-tical net-works semi-conduc-tor}

\lstset{
language=Python,
frame=single, 
breaklines=true,
columns=fullflexible
}
\begin{document}
\theoremstyle{definition}
\newtheorem{theorem}{Theorem}[section]
\newtheorem{problem}{Problem}
\newtheorem{proposition}{Proposition}[section]
\newtheorem{lemma}{Lemma}[section]
\newtheorem{corollary}[theorem]{Corollary}
\newtheorem{example}{Example}[section]
\newtheorem{definition}{Definition}[section]
\newcommand{\BEQA}{\begin{eqnarray}}
\newcommand{\EEQA}{\end{eqnarray}}
\newcommand{\define}{\stackrel{\triangle}{=}}
\newenvironment{amatrix}[1]{%
  \left(\begin{array}{@{}*{#1}{c}|c@{}}
}{%
  \end{array}\right)
}
\newcommand{\myvec}[1]{\ensuremath{\begin{pmatrix}#1\end{pmatrix}}}
\newcommand{\myaugvec}[2]{\ensuremath{\begin{amatrix}{#1}#2\end{amatrix}}}
\newcommand{\mydet}[1]{\ensuremath{\begin{vmatrix}#1\end{vmatrix}}}
\bibliographystyle{IEEEtran}
\providecommand{\nCr}[2]{\,^{#1}C_{#2}} % nCr
\providecommand{\nPr}[2]{\,^{#1}P_{#2}} % nPr
\providecommand{\mbf}{\mathbf}
\providecommand{\pr}[1]{\ensuremath{\Pr\left(#1\right)}}
\providecommand{\qfunc}[1]{\ensuremath{Q\left(#1\right)}}
\providecommand{\sbrak}[1]{\ensuremath{{}\left[#1\right]}}
\providecommand{\lsbrak}[1]{\ensuremath{{}\left[#1\right.}}
\providecommand{\rsbrak}[1]{\ensuremath{{}\left.#1\right]}}
\providecommand{\brak}[1]{\ensuremath{\left(#1\right)}}
\providecommand{\lbrak}[1]{\ensuremath{\left(#1\right.}}
\providecommand{\rbrak}[1]{\ensuremath{\left.#1\right)}}
\providecommand{\cbrak}[1]{\ensuremath{\left\{#1\right\}}}
\providecommand{\lcbrak}[1]{\ensuremath{\left\{#1\right.}}
\providecommand{\rcbrak}[1]{\ensuremath{\left.#1\right\}}}
\theoremstyle{remark}
\newtheorem{rem}{Remark}
\newcommand{\sgn}{\mathop{\mathrm{sgn}}}
\newcommand{\rect}{\mathop{\mathrm{rect}}}
\newcommand{\sinc}{\mathop{\mathrm{sinc}}}
\providecommand{\abs}[1]{\left\vert#1\right\vert}
\providecommand{\res}[1]{\Res\displaylimits_{#1}} 
\providecommand{\norm}[1]{\left\Vert#1\right\Vert}
\providecommand{\mtx}[1]{\mathbf{#1}}
\providecommand{\mean}[1]{E\left[ #1 \right]}
\providecommand{\fourier}{\overset{\mathcal{F}}{ \rightleftharpoons}}
\providecommand{\ztrans}{\overset{\mathcal{Z}}{ \rightleftharpoons}}
\providecommand{\system}[1]{\overset{\mathcal{#1}}{ \longleftrightarrow}}
\newcommand{\solution}{\noindent \textbf{Solution: }}
\providecommand{\dec}[2]{\ensuremath{\overset{#1}{\underset{#2}{\gtrless}}}}
\let\StandardTheFigure\thefigure
\def\putbox#1#2#3{\makebox[0in][l]{\makebox[#1][l]{}\raisebox{\baselineskip}[0in][0in]{\raisebox{#2}[0in][0in]{#3}}}}
     \def\rightbox#1{\makebox[0in][r]{#1}}
     \def\centbox#1{\makebox[0in]{#1}}
     \def\topbox#1{\raisebox{-\baselineskip}[0in][0in]{#1}}
     \def\midbox#1{\raisebox{-0.5\baselineskip}[0in][0in]{#1}}

\vspace{3cm}
\title{Line Assignment}
\author{Gautam Singh}
\maketitle
\bigskip

\begin{abstract}
    This document contains the solution to Question 24 of Exercise 4 
    in Chapter 10 of the class 11 NCERT textbook.
\end{abstract}

\begin{enumerate}
\fi
		We first find the coordinates of the intersection of \eqref{eq:chapters/11/10/4/24/L1}
    and \eqref{eq:chapters/11/10/4/24/L2}. Using the augmented matrix and row reduction methods,
    \begin{align}
        \myaugvec{2}{2&-3&-4\\3&4&5} &\xleftrightarrow[]{R_2\rightarrow2R_2-3R_1} 
        \myaugvec{2}{2&-3&-4\\0&17&22} \\
                      &\xleftrightarrow[]{R_1\rightarrow17R_1+3R_2} \myaugvec{2}{17&0&-1\\0&17&22} \\
                      &\xleftrightarrow[]{\substack{R_1\rightarrow\frac{R_1}{17}\\R_2\rightarrow\frac{R_2}{17}}} \myaugvec{2}{1&0&-\frac{1}{17}\\0&1&\frac{22}{17}}
        \label{eq:chapters/11/10/4/24/intersect}
    \end{align}
    the intersection of the lines is
    \begin{align}
        \vec{A} = \frac{1}{17}\myvec{-1\\22}
    \end{align}
    Clearly, the man should follow the path perpendicular to \eqref{eq:chapters/11/10/4/24/L3} from
    $\vec{A}$ to reach it in the shortest time. The normal vector 
    of \eqref{eq:chapters/11/10/4/24/L3} is 
    \begin{align}
        \vec{m} = \myvec{6\\-7}
        \label{eq:chapters/11/10/4/24/L3-norm}
    \end{align}
    which is consequently the direction vector of the required line. Therefore, 
    the required normal vector is given by
    \begin{align}
        \vec{n} = \myvec{7\\6}
        \label{eq:chapters/11/10/4/24/L4-norm}
    \end{align}
    and hence, the equation of the line is
   \begin{align}
        \vec{n}^\top\vec{x} &= \vec{n}^\top\vec{A} \\
        \implies \myvec{7&6}\vec{x} &= \frac{1}{17}\myvec{7&6}\myvec{-1\\22} = \frac{125}{17}
        \label{eq:chapters/11/10/4/24/L4}
    \end{align}
		See Fig. \ref{fig:chapters/11/10/4/24/crossing}. In this figure $\vec{F}$ represents 
    the foot of the prependicular drawn from $\vec{A}$ onto \eqref{eq:chapters/11/10/4/24/L3}.

\item If $p$ is the length of perpendicular from origin to the line whose intercepts on the axes are $a$ and $b$, then show that 
\begin{align}
	\frac{1}{p^2} = \frac{1}{a^2}+ \frac{1}{b^2}
\end{align}
\label{chapters/11/10/3/18}
\input{chapters/11/10/3/18/dist.tex}
\item What are the points on the y-axis whose distance from the line $\frac{x}{3}+\frac{y}{4}=1$ is 4 units.
\\
\solution
		\iffalse
\documentclass[journal,12pt,twocolumn]{IEEEtran}
\usepackage{setspace}
\usepackage{gensymb}
\usepackage{xcolor}
\usepackage{caption}
\singlespacing
\usepackage{siunitx}
\usepackage[cmex10]{amsmath}
\usepackage{mathtools}
\usepackage{hyperref}
\usepackage{amsthm}
\usepackage{mathrsfs}
\usepackage{txfonts}
\usepackage{stfloats}
\usepackage{cite}
\usepackage{cases}
\usepackage{subfig}
\usepackage{longtable}
\usepackage{multirow}
\usepackage{enumitem}
\usepackage{mathtools}
\usepackage{listings}
\usepackage{tikz}
\usetikzlibrary{shapes,arrows,positioning}
\usepackage{circuitikz}
\let\vec\mathbf
\DeclareMathOperator*{\Res}{Res}
\renewcommand\thesection{\arabic{section}}
\renewcommand\thesubsection{\thesection.\arabic{subsection}}
\renewcommand\thesubsubsection{\thesubsection.\arabic{subsubsection}}

\renewcommand\thesectiondis{\arabic{section}}
\renewcommand\thesubsectiondis{\thesectiondis.\arabic{subsection}}
\renewcommand\thesubsubsectiondis{\thesubsectiondis.\arabic{subsubsection}}
\hyphenation{op-tical net-works semi-conduc-tor}

\lstset{
language=Python,
frame=single, 
breaklines=true,
columns=fullflexible
}
\begin{document}
\theoremstyle{definition}
\newtheorem{theorem}{Theorem}[section]
\newtheorem{problem}{Problem}
\newtheorem{proposition}{Proposition}[section]
\newtheorem{lemma}{Lemma}[section]
\newtheorem{corollary}[theorem]{Corollary}
\newtheorem{example}{Example}[section]
\newtheorem{definition}{Definition}[section]
\newcommand{\BEQA}{\begin{eqnarray}}
\newcommand{\EEQA}{\end{eqnarray}}
\newcommand{\define}{\stackrel{\triangle}{=}}
\newenvironment{amatrix}[1]{%
  \left(\begin{array}{@{}*{#1}{c}|c@{}}
}{%
  \end{array}\right)
}
\newcommand{\myvec}[1]{\ensuremath{\begin{pmatrix}#1\end{pmatrix}}}
\newcommand{\myaugvec}[2]{\ensuremath{\begin{amatrix}{#1}#2\end{amatrix}}}
\newcommand{\mydet}[1]{\ensuremath{\begin{vmatrix}#1\end{vmatrix}}}
\bibliographystyle{IEEEtran}
\providecommand{\nCr}[2]{\,^{#1}C_{#2}} % nCr
\providecommand{\nPr}[2]{\,^{#1}P_{#2}} % nPr
\providecommand{\mbf}{\mathbf}
\providecommand{\pr}[1]{\ensuremath{\Pr\left(#1\right)}}
\providecommand{\qfunc}[1]{\ensuremath{Q\left(#1\right)}}
\providecommand{\sbrak}[1]{\ensuremath{{}\left[#1\right]}}
\providecommand{\lsbrak}[1]{\ensuremath{{}\left[#1\right.}}
\providecommand{\rsbrak}[1]{\ensuremath{{}\left.#1\right]}}
\providecommand{\brak}[1]{\ensuremath{\left(#1\right)}}
\providecommand{\lbrak}[1]{\ensuremath{\left(#1\right.}}
\providecommand{\rbrak}[1]{\ensuremath{\left.#1\right)}}
\providecommand{\cbrak}[1]{\ensuremath{\left\{#1\right\}}}
\providecommand{\lcbrak}[1]{\ensuremath{\left\{#1\right.}}
\providecommand{\rcbrak}[1]{\ensuremath{\left.#1\right\}}}
\theoremstyle{remark}
\newtheorem{rem}{Remark}
\newcommand{\sgn}{\mathop{\mathrm{sgn}}}
\newcommand{\rect}{\mathop{\mathrm{rect}}}
\newcommand{\sinc}{\mathop{\mathrm{sinc}}}
\providecommand{\abs}[1]{\left\vert#1\right\vert}
\providecommand{\res}[1]{\Res\displaylimits_{#1}} 
\providecommand{\norm}[1]{\left\Vert#1\right\Vert}
\providecommand{\mtx}[1]{\mathbf{#1}}
\providecommand{\mean}[1]{E\left[ #1 \right]}
\providecommand{\fourier}{\overset{\mathcal{F}}{ \rightleftharpoons}}
\providecommand{\ztrans}{\overset{\mathcal{Z}}{ \rightleftharpoons}}
\providecommand{\system}[1]{\overset{\mathcal{#1}}{ \longleftrightarrow}}
\newcommand{\solution}{\noindent \textbf{Solution: }}
\providecommand{\dec}[2]{\ensuremath{\overset{#1}{\underset{#2}{\gtrless}}}}
\let\StandardTheFigure\thefigure
\def\putbox#1#2#3{\makebox[0in][l]{\makebox[#1][l]{}\raisebox{\baselineskip}[0in][0in]{\raisebox{#2}[0in][0in]{#3}}}}
     \def\rightbox#1{\makebox[0in][r]{#1}}
     \def\centbox#1{\makebox[0in]{#1}}
     \def\topbox#1{\raisebox{-\baselineskip}[0in][0in]{#1}}
     \def\midbox#1{\raisebox{-0.5\baselineskip}[0in][0in]{#1}}

\vspace{3cm}
\title{Line Assignment}
\author{Gautam Singh}
\maketitle
\bigskip

\begin{abstract}
    This document contains the solution to Question 24 of Exercise 4 
    in Chapter 10 of the class 11 NCERT textbook.
\end{abstract}

\begin{enumerate}
\fi
		We first find the coordinates of the intersection of \eqref{eq:chapters/11/10/4/24/L1}
    and \eqref{eq:chapters/11/10/4/24/L2}. Using the augmented matrix and row reduction methods,
    \begin{align}
        \myaugvec{2}{2&-3&-4\\3&4&5} &\xleftrightarrow[]{R_2\rightarrow2R_2-3R_1} 
        \myaugvec{2}{2&-3&-4\\0&17&22} \\
                      &\xleftrightarrow[]{R_1\rightarrow17R_1+3R_2} \myaugvec{2}{17&0&-1\\0&17&22} \\
                      &\xleftrightarrow[]{\substack{R_1\rightarrow\frac{R_1}{17}\\R_2\rightarrow\frac{R_2}{17}}} \myaugvec{2}{1&0&-\frac{1}{17}\\0&1&\frac{22}{17}}
        \label{eq:chapters/11/10/4/24/intersect}
    \end{align}
    the intersection of the lines is
    \begin{align}
        \vec{A} = \frac{1}{17}\myvec{-1\\22}
    \end{align}
    Clearly, the man should follow the path perpendicular to \eqref{eq:chapters/11/10/4/24/L3} from
    $\vec{A}$ to reach it in the shortest time. The normal vector 
    of \eqref{eq:chapters/11/10/4/24/L3} is 
    \begin{align}
        \vec{m} = \myvec{6\\-7}
        \label{eq:chapters/11/10/4/24/L3-norm}
    \end{align}
    which is consequently the direction vector of the required line. Therefore, 
    the required normal vector is given by
    \begin{align}
        \vec{n} = \myvec{7\\6}
        \label{eq:chapters/11/10/4/24/L4-norm}
    \end{align}
    and hence, the equation of the line is
   \begin{align}
        \vec{n}^\top\vec{x} &= \vec{n}^\top\vec{A} \\
        \implies \myvec{7&6}\vec{x} &= \frac{1}{17}\myvec{7&6}\myvec{-1\\22} = \frac{125}{17}
        \label{eq:chapters/11/10/4/24/L4}
    \end{align}
		See Fig. \ref{fig:chapters/11/10/4/24/crossing}. In this figure $\vec{F}$ represents 
    the foot of the prependicular drawn from $\vec{A}$ onto \eqref{eq:chapters/11/10/4/24/L3}.

\item Find perpendicular distance from the origin to the line joining the points$(\cos\theta,\sin\theta)$ and $(\cos\phi,\sin\phi)$.
\\
\solution
		\input{chapters/11/10/4/5/dist.tex}
\item Find the equation of line which is equidistant from parallel lines $9x+6y-7=0$ and $3x+2y+6=0$.
\\
\solution
		\iffalse
\documentclass[journal,12pt,twocolumn]{IEEEtran}
\usepackage{setspace}
\usepackage{gensymb}
\usepackage{xcolor}
\usepackage{caption}
\singlespacing
\usepackage{siunitx}
\usepackage[cmex10]{amsmath}
\usepackage{mathtools}
\usepackage{hyperref}
\usepackage{amsthm}
\usepackage{mathrsfs}
\usepackage{txfonts}
\usepackage{stfloats}
\usepackage{cite}
\usepackage{cases}
\usepackage{subfig}
\usepackage{longtable}
\usepackage{multirow}
\usepackage{enumitem}
\usepackage{mathtools}
\usepackage{listings}
\usepackage{tikz}
\usetikzlibrary{shapes,arrows,positioning}
\usepackage{circuitikz}
\let\vec\mathbf
\DeclareMathOperator*{\Res}{Res}
\renewcommand\thesection{\arabic{section}}
\renewcommand\thesubsection{\thesection.\arabic{subsection}}
\renewcommand\thesubsubsection{\thesubsection.\arabic{subsubsection}}

\renewcommand\thesectiondis{\arabic{section}}
\renewcommand\thesubsectiondis{\thesectiondis.\arabic{subsection}}
\renewcommand\thesubsubsectiondis{\thesubsectiondis.\arabic{subsubsection}}
\hyphenation{op-tical net-works semi-conduc-tor}

\lstset{
language=Python,
frame=single, 
breaklines=true,
columns=fullflexible
}
\begin{document}
\theoremstyle{definition}
\newtheorem{theorem}{Theorem}[section]
\newtheorem{problem}{Problem}
\newtheorem{proposition}{Proposition}[section]
\newtheorem{lemma}{Lemma}[section]
\newtheorem{corollary}[theorem]{Corollary}
\newtheorem{example}{Example}[section]
\newtheorem{definition}{Definition}[section]
\newcommand{\BEQA}{\begin{eqnarray}}
\newcommand{\EEQA}{\end{eqnarray}}
\newcommand{\define}{\stackrel{\triangle}{=}}
\newenvironment{amatrix}[1]{%
  \left(\begin{array}{@{}*{#1}{c}|c@{}}
}{%
  \end{array}\right)
}
\newcommand{\myvec}[1]{\ensuremath{\begin{pmatrix}#1\end{pmatrix}}}
\newcommand{\myaugvec}[2]{\ensuremath{\begin{amatrix}{#1}#2\end{amatrix}}}
\newcommand{\mydet}[1]{\ensuremath{\begin{vmatrix}#1\end{vmatrix}}}
\bibliographystyle{IEEEtran}
\providecommand{\nCr}[2]{\,^{#1}C_{#2}} % nCr
\providecommand{\nPr}[2]{\,^{#1}P_{#2}} % nPr
\providecommand{\mbf}{\mathbf}
\providecommand{\pr}[1]{\ensuremath{\Pr\left(#1\right)}}
\providecommand{\qfunc}[1]{\ensuremath{Q\left(#1\right)}}
\providecommand{\sbrak}[1]{\ensuremath{{}\left[#1\right]}}
\providecommand{\lsbrak}[1]{\ensuremath{{}\left[#1\right.}}
\providecommand{\rsbrak}[1]{\ensuremath{{}\left.#1\right]}}
\providecommand{\brak}[1]{\ensuremath{\left(#1\right)}}
\providecommand{\lbrak}[1]{\ensuremath{\left(#1\right.}}
\providecommand{\rbrak}[1]{\ensuremath{\left.#1\right)}}
\providecommand{\cbrak}[1]{\ensuremath{\left\{#1\right\}}}
\providecommand{\lcbrak}[1]{\ensuremath{\left\{#1\right.}}
\providecommand{\rcbrak}[1]{\ensuremath{\left.#1\right\}}}
\theoremstyle{remark}
\newtheorem{rem}{Remark}
\newcommand{\sgn}{\mathop{\mathrm{sgn}}}
\newcommand{\rect}{\mathop{\mathrm{rect}}}
\newcommand{\sinc}{\mathop{\mathrm{sinc}}}
\providecommand{\abs}[1]{\left\vert#1\right\vert}
\providecommand{\res}[1]{\Res\displaylimits_{#1}} 
\providecommand{\norm}[1]{\left\Vert#1\right\Vert}
\providecommand{\mtx}[1]{\mathbf{#1}}
\providecommand{\mean}[1]{E\left[ #1 \right]}
\providecommand{\fourier}{\overset{\mathcal{F}}{ \rightleftharpoons}}
\providecommand{\ztrans}{\overset{\mathcal{Z}}{ \rightleftharpoons}}
\providecommand{\system}[1]{\overset{\mathcal{#1}}{ \longleftrightarrow}}
\newcommand{\solution}{\noindent \textbf{Solution: }}
\providecommand{\dec}[2]{\ensuremath{\overset{#1}{\underset{#2}{\gtrless}}}}
\let\StandardTheFigure\thefigure
\def\putbox#1#2#3{\makebox[0in][l]{\makebox[#1][l]{}\raisebox{\baselineskip}[0in][0in]{\raisebox{#2}[0in][0in]{#3}}}}
     \def\rightbox#1{\makebox[0in][r]{#1}}
     \def\centbox#1{\makebox[0in]{#1}}
     \def\topbox#1{\raisebox{-\baselineskip}[0in][0in]{#1}}
     \def\midbox#1{\raisebox{-0.5\baselineskip}[0in][0in]{#1}}

\vspace{3cm}
\title{Line Assignment}
\author{Gautam Singh}
\maketitle
\bigskip

\begin{abstract}
    This document contains the solution to Question 24 of Exercise 4 
    in Chapter 10 of the class 11 NCERT textbook.
\end{abstract}

\begin{enumerate}
\fi
		We first find the coordinates of the intersection of \eqref{eq:chapters/11/10/4/24/L1}
    and \eqref{eq:chapters/11/10/4/24/L2}. Using the augmented matrix and row reduction methods,
    \begin{align}
        \myaugvec{2}{2&-3&-4\\3&4&5} &\xleftrightarrow[]{R_2\rightarrow2R_2-3R_1} 
        \myaugvec{2}{2&-3&-4\\0&17&22} \\
                      &\xleftrightarrow[]{R_1\rightarrow17R_1+3R_2} \myaugvec{2}{17&0&-1\\0&17&22} \\
                      &\xleftrightarrow[]{\substack{R_1\rightarrow\frac{R_1}{17}\\R_2\rightarrow\frac{R_2}{17}}} \myaugvec{2}{1&0&-\frac{1}{17}\\0&1&\frac{22}{17}}
        \label{eq:chapters/11/10/4/24/intersect}
    \end{align}
    the intersection of the lines is
    \begin{align}
        \vec{A} = \frac{1}{17}\myvec{-1\\22}
    \end{align}
    Clearly, the man should follow the path perpendicular to \eqref{eq:chapters/11/10/4/24/L3} from
    $\vec{A}$ to reach it in the shortest time. The normal vector 
    of \eqref{eq:chapters/11/10/4/24/L3} is 
    \begin{align}
        \vec{m} = \myvec{6\\-7}
        \label{eq:chapters/11/10/4/24/L3-norm}
    \end{align}
    which is consequently the direction vector of the required line. Therefore, 
    the required normal vector is given by
    \begin{align}
        \vec{n} = \myvec{7\\6}
        \label{eq:chapters/11/10/4/24/L4-norm}
    \end{align}
    and hence, the equation of the line is
   \begin{align}
        \vec{n}^\top\vec{x} &= \vec{n}^\top\vec{A} \\
        \implies \myvec{7&6}\vec{x} &= \frac{1}{17}\myvec{7&6}\myvec{-1\\22} = \frac{125}{17}
        \label{eq:chapters/11/10/4/24/L4}
    \end{align}
		See Fig. \ref{fig:chapters/11/10/4/24/crossing}. In this figure $\vec{F}$ represents 
    the foot of the prependicular drawn from $\vec{A}$ onto \eqref{eq:chapters/11/10/4/24/L3}.

	\item Prove that the products of the lengths of the perpendiculars drawn from the points $\myvec{\sqrt{a^2-b^2}\\0}$ and $\myvec{-\sqrt{a^2-b^2} \\0} $ to the line $\frac{x}{a} \cos{\theta} + \frac{y}{b}\sin{\theta} =1 $ is $ b^2 $.
\\
    \solution 
		\input{chapters/11/10/4/23/dist.tex}
\item Find the equation of line  drawn perpendicular to the line $\frac{x}{4}+\frac{y}{6}=1$ through the point where it meets the y-axis \\
\solution
		\iffalse
\documentclass[journal,12pt,twocolumn]{IEEEtran}
\usepackage{setspace}
\usepackage{gensymb}
\usepackage{xcolor}
\usepackage{caption}
\singlespacing
\usepackage{siunitx}
\usepackage[cmex10]{amsmath}
\usepackage{mathtools}
\usepackage{hyperref}
\usepackage{amsthm}
\usepackage{mathrsfs}
\usepackage{txfonts}
\usepackage{stfloats}
\usepackage{cite}
\usepackage{cases}
\usepackage{subfig}
\usepackage{longtable}
\usepackage{multirow}
\usepackage{enumitem}
\usepackage{mathtools}
\usepackage{listings}
\usepackage{tikz}
\usetikzlibrary{shapes,arrows,positioning}
\usepackage{circuitikz}
\let\vec\mathbf
\DeclareMathOperator*{\Res}{Res}
\renewcommand\thesection{\arabic{section}}
\renewcommand\thesubsection{\thesection.\arabic{subsection}}
\renewcommand\thesubsubsection{\thesubsection.\arabic{subsubsection}}

\renewcommand\thesectiondis{\arabic{section}}
\renewcommand\thesubsectiondis{\thesectiondis.\arabic{subsection}}
\renewcommand\thesubsubsectiondis{\thesubsectiondis.\arabic{subsubsection}}
\hyphenation{op-tical net-works semi-conduc-tor}

\lstset{
language=Python,
frame=single, 
breaklines=true,
columns=fullflexible
}
\begin{document}
\theoremstyle{definition}
\newtheorem{theorem}{Theorem}[section]
\newtheorem{problem}{Problem}
\newtheorem{proposition}{Proposition}[section]
\newtheorem{lemma}{Lemma}[section]
\newtheorem{corollary}[theorem]{Corollary}
\newtheorem{example}{Example}[section]
\newtheorem{definition}{Definition}[section]
\newcommand{\BEQA}{\begin{eqnarray}}
\newcommand{\EEQA}{\end{eqnarray}}
\newcommand{\define}{\stackrel{\triangle}{=}}
\newenvironment{amatrix}[1]{%
  \left(\begin{array}{@{}*{#1}{c}|c@{}}
}{%
  \end{array}\right)
}
\newcommand{\myvec}[1]{\ensuremath{\begin{pmatrix}#1\end{pmatrix}}}
\newcommand{\myaugvec}[2]{\ensuremath{\begin{amatrix}{#1}#2\end{amatrix}}}
\newcommand{\mydet}[1]{\ensuremath{\begin{vmatrix}#1\end{vmatrix}}}
\bibliographystyle{IEEEtran}
\providecommand{\nCr}[2]{\,^{#1}C_{#2}} % nCr
\providecommand{\nPr}[2]{\,^{#1}P_{#2}} % nPr
\providecommand{\mbf}{\mathbf}
\providecommand{\pr}[1]{\ensuremath{\Pr\left(#1\right)}}
\providecommand{\qfunc}[1]{\ensuremath{Q\left(#1\right)}}
\providecommand{\sbrak}[1]{\ensuremath{{}\left[#1\right]}}
\providecommand{\lsbrak}[1]{\ensuremath{{}\left[#1\right.}}
\providecommand{\rsbrak}[1]{\ensuremath{{}\left.#1\right]}}
\providecommand{\brak}[1]{\ensuremath{\left(#1\right)}}
\providecommand{\lbrak}[1]{\ensuremath{\left(#1\right.}}
\providecommand{\rbrak}[1]{\ensuremath{\left.#1\right)}}
\providecommand{\cbrak}[1]{\ensuremath{\left\{#1\right\}}}
\providecommand{\lcbrak}[1]{\ensuremath{\left\{#1\right.}}
\providecommand{\rcbrak}[1]{\ensuremath{\left.#1\right\}}}
\theoremstyle{remark}
\newtheorem{rem}{Remark}
\newcommand{\sgn}{\mathop{\mathrm{sgn}}}
\newcommand{\rect}{\mathop{\mathrm{rect}}}
\newcommand{\sinc}{\mathop{\mathrm{sinc}}}
\providecommand{\abs}[1]{\left\vert#1\right\vert}
\providecommand{\res}[1]{\Res\displaylimits_{#1}} 
\providecommand{\norm}[1]{\left\Vert#1\right\Vert}
\providecommand{\mtx}[1]{\mathbf{#1}}
\providecommand{\mean}[1]{E\left[ #1 \right]}
\providecommand{\fourier}{\overset{\mathcal{F}}{ \rightleftharpoons}}
\providecommand{\ztrans}{\overset{\mathcal{Z}}{ \rightleftharpoons}}
\providecommand{\system}[1]{\overset{\mathcal{#1}}{ \longleftrightarrow}}
\newcommand{\solution}{\noindent \textbf{Solution: }}
\providecommand{\dec}[2]{\ensuremath{\overset{#1}{\underset{#2}{\gtrless}}}}
\let\StandardTheFigure\thefigure
\def\putbox#1#2#3{\makebox[0in][l]{\makebox[#1][l]{}\raisebox{\baselineskip}[0in][0in]{\raisebox{#2}[0in][0in]{#3}}}}
     \def\rightbox#1{\makebox[0in][r]{#1}}
     \def\centbox#1{\makebox[0in]{#1}}
     \def\topbox#1{\raisebox{-\baselineskip}[0in][0in]{#1}}
     \def\midbox#1{\raisebox{-0.5\baselineskip}[0in][0in]{#1}}

\vspace{3cm}
\title{Line Assignment}
\author{Gautam Singh}
\maketitle
\bigskip

\begin{abstract}
    This document contains the solution to Question 24 of Exercise 4 
    in Chapter 10 of the class 11 NCERT textbook.
\end{abstract}

\begin{enumerate}
\fi
		We first find the coordinates of the intersection of \eqref{eq:chapters/11/10/4/24/L1}
    and \eqref{eq:chapters/11/10/4/24/L2}. Using the augmented matrix and row reduction methods,
    \begin{align}
        \myaugvec{2}{2&-3&-4\\3&4&5} &\xleftrightarrow[]{R_2\rightarrow2R_2-3R_1} 
        \myaugvec{2}{2&-3&-4\\0&17&22} \\
                      &\xleftrightarrow[]{R_1\rightarrow17R_1+3R_2} \myaugvec{2}{17&0&-1\\0&17&22} \\
                      &\xleftrightarrow[]{\substack{R_1\rightarrow\frac{R_1}{17}\\R_2\rightarrow\frac{R_2}{17}}} \myaugvec{2}{1&0&-\frac{1}{17}\\0&1&\frac{22}{17}}
        \label{eq:chapters/11/10/4/24/intersect}
    \end{align}
    the intersection of the lines is
    \begin{align}
        \vec{A} = \frac{1}{17}\myvec{-1\\22}
    \end{align}
    Clearly, the man should follow the path perpendicular to \eqref{eq:chapters/11/10/4/24/L3} from
    $\vec{A}$ to reach it in the shortest time. The normal vector 
    of \eqref{eq:chapters/11/10/4/24/L3} is 
    \begin{align}
        \vec{m} = \myvec{6\\-7}
        \label{eq:chapters/11/10/4/24/L3-norm}
    \end{align}
    which is consequently the direction vector of the required line. Therefore, 
    the required normal vector is given by
    \begin{align}
        \vec{n} = \myvec{7\\6}
        \label{eq:chapters/11/10/4/24/L4-norm}
    \end{align}
    and hence, the equation of the line is
   \begin{align}
        \vec{n}^\top\vec{x} &= \vec{n}^\top\vec{A} \\
        \implies \myvec{7&6}\vec{x} &= \frac{1}{17}\myvec{7&6}\myvec{-1\\22} = \frac{125}{17}
        \label{eq:chapters/11/10/4/24/L4}
    \end{align}
		See Fig. \ref{fig:chapters/11/10/4/24/crossing}. In this figure $\vec{F}$ represents 
    the foot of the prependicular drawn from $\vec{A}$ onto \eqref{eq:chapters/11/10/4/24/L3}.

 \item  In each of the following cases, determine the direction cosines of the normal to
the plane and the distance from the origin.
\begin{enumerate}
	\item $z=2$ 
	\item $x + y + z = 1$
	\item $2x + 3y – z = 5$
	\item $5y + 8 = 0$
\end{enumerate}
    \solution
		\input{chapters/12/11/3/1/dist.tex}
\item
Find the angle between the lines whose direction ratios are $a,b,c$ and $b-c,c-a,a-b$.

\textbf{Solution :}
    \begin{align}
    \vec{m _1} &= \myvec{a\\b\\c}\\
    \vec{m_2} &= \myvec{b-c\\c-a\\a-b}\\
    \cos{\theta}&= \frac{\vec{m_1}^{\top}\vec{m_2}}{\vec{\norm{m_1}\norm{m_2}}
   } \\
   &=\frac{\myvec{a&b&c}\myvec{b-c\\c-a\\a-b}}{\sqrt{a^2+b^2+c^2}\sqrt{\brak{b-c}^2+\brak{c-a}^2+\brak{a-b}^2}}\\
   &=0\\
   or,\theta&=\frac{\pi}{2}
    \end{align}

\end{enumerate}

\item Find the distance of the point $(-1,1)$ from the line $12\brak{x+6} = 5\brak{y-2}$. 
\label{chapters/11/10/3/4}
\iffalse
\documentclass[journal,12pt,twocolumn]{IEEEtran}
\usepackage{setspace}
\usepackage{gensymb}
\usepackage{xcolor}
\usepackage{caption}
\singlespacing
\usepackage{siunitx}
\usepackage[cmex10]{amsmath}
\usepackage{mathtools}
\usepackage{hyperref}
\usepackage{amsthm}
\usepackage{mathrsfs}
\usepackage{txfonts}
\usepackage{stfloats}
\usepackage{cite}
\usepackage{cases}
\usepackage{subfig}
\usepackage{longtable}
\usepackage{multirow}
\usepackage{enumitem}
\usepackage{mathtools}
\usepackage{listings}
\usepackage{tikz}
\usetikzlibrary{shapes,arrows,positioning}
\usepackage{circuitikz}
\let\vec\mathbf
\DeclareMathOperator*{\Res}{Res}
\renewcommand\thesection{\arabic{section}}
\renewcommand\thesubsection{\thesection.\arabic{subsection}}
\renewcommand\thesubsubsection{\thesubsection.\arabic{subsubsection}}

\renewcommand\thesectiondis{\arabic{section}}
\renewcommand\thesubsectiondis{\thesectiondis.\arabic{subsection}}
\renewcommand\thesubsubsectiondis{\thesubsectiondis.\arabic{subsubsection}}
\hyphenation{op-tical net-works semi-conduc-tor}

\lstset{
language=Python,
frame=single, 
breaklines=true,
columns=fullflexible
}
\begin{document}
\theoremstyle{definition}
\newtheorem{theorem}{Theorem}[section]
\newtheorem{problem}{Problem}
\newtheorem{proposition}{Proposition}[section]
\newtheorem{lemma}{Lemma}[section]
\newtheorem{corollary}[theorem]{Corollary}
\newtheorem{example}{Example}[section]
\newtheorem{definition}{Definition}[section]
\newcommand{\BEQA}{\begin{eqnarray}}
\newcommand{\EEQA}{\end{eqnarray}}
\newcommand{\define}{\stackrel{\triangle}{=}}
\newenvironment{amatrix}[1]{%
  \left(\begin{array}{@{}*{#1}{c}|c@{}}
}{%
  \end{array}\right)
}
\newcommand{\myvec}[1]{\ensuremath{\begin{pmatrix}#1\end{pmatrix}}}
\newcommand{\myaugvec}[2]{\ensuremath{\begin{amatrix}{#1}#2\end{amatrix}}}
\newcommand{\mydet}[1]{\ensuremath{\begin{vmatrix}#1\end{vmatrix}}}
\bibliographystyle{IEEEtran}
\providecommand{\nCr}[2]{\,^{#1}C_{#2}} % nCr
\providecommand{\nPr}[2]{\,^{#1}P_{#2}} % nPr
\providecommand{\mbf}{\mathbf}
\providecommand{\pr}[1]{\ensuremath{\Pr\left(#1\right)}}
\providecommand{\qfunc}[1]{\ensuremath{Q\left(#1\right)}}
\providecommand{\sbrak}[1]{\ensuremath{{}\left[#1\right]}}
\providecommand{\lsbrak}[1]{\ensuremath{{}\left[#1\right.}}
\providecommand{\rsbrak}[1]{\ensuremath{{}\left.#1\right]}}
\providecommand{\brak}[1]{\ensuremath{\left(#1\right)}}
\providecommand{\lbrak}[1]{\ensuremath{\left(#1\right.}}
\providecommand{\rbrak}[1]{\ensuremath{\left.#1\right)}}
\providecommand{\cbrak}[1]{\ensuremath{\left\{#1\right\}}}
\providecommand{\lcbrak}[1]{\ensuremath{\left\{#1\right.}}
\providecommand{\rcbrak}[1]{\ensuremath{\left.#1\right\}}}
\theoremstyle{remark}
\newtheorem{rem}{Remark}
\newcommand{\sgn}{\mathop{\mathrm{sgn}}}
\newcommand{\rect}{\mathop{\mathrm{rect}}}
\newcommand{\sinc}{\mathop{\mathrm{sinc}}}
\providecommand{\abs}[1]{\left\vert#1\right\vert}
\providecommand{\res}[1]{\Res\displaylimits_{#1}} 
\providecommand{\norm}[1]{\left\Vert#1\right\Vert}
\providecommand{\mtx}[1]{\mathbf{#1}}
\providecommand{\mean}[1]{E\left[ #1 \right]}
\providecommand{\fourier}{\overset{\mathcal{F}}{ \rightleftharpoons}}
\providecommand{\ztrans}{\overset{\mathcal{Z}}{ \rightleftharpoons}}
\providecommand{\system}[1]{\overset{\mathcal{#1}}{ \longleftrightarrow}}
\newcommand{\solution}{\noindent \textbf{Solution: }}
\providecommand{\dec}[2]{\ensuremath{\overset{#1}{\underset{#2}{\gtrless}}}}
\let\StandardTheFigure\thefigure
\def\putbox#1#2#3{\makebox[0in][l]{\makebox[#1][l]{}\raisebox{\baselineskip}[0in][0in]{\raisebox{#2}[0in][0in]{#3}}}}
     \def\rightbox#1{\makebox[0in][r]{#1}}
     \def\centbox#1{\makebox[0in]{#1}}
     \def\topbox#1{\raisebox{-\baselineskip}[0in][0in]{#1}}
     \def\midbox#1{\raisebox{-0.5\baselineskip}[0in][0in]{#1}}

\vspace{3cm}
\title{Line Assignment}
\author{Gautam Singh}
\maketitle
\bigskip

\begin{abstract}
    This document contains the solution to Question 24 of Exercise 4 
    in Chapter 10 of the class 11 NCERT textbook.
\end{abstract}

\begin{enumerate}
\fi
		We first find the coordinates of the intersection of \eqref{eq:chapters/11/10/4/24/L1}
    and \eqref{eq:chapters/11/10/4/24/L2}. Using the augmented matrix and row reduction methods,
    \begin{align}
        \myaugvec{2}{2&-3&-4\\3&4&5} &\xleftrightarrow[]{R_2\rightarrow2R_2-3R_1} 
        \myaugvec{2}{2&-3&-4\\0&17&22} \\
                      &\xleftrightarrow[]{R_1\rightarrow17R_1+3R_2} \myaugvec{2}{17&0&-1\\0&17&22} \\
                      &\xleftrightarrow[]{\substack{R_1\rightarrow\frac{R_1}{17}\\R_2\rightarrow\frac{R_2}{17}}} \myaugvec{2}{1&0&-\frac{1}{17}\\0&1&\frac{22}{17}}
        \label{eq:chapters/11/10/4/24/intersect}
    \end{align}
    the intersection of the lines is
    \begin{align}
        \vec{A} = \frac{1}{17}\myvec{-1\\22}
    \end{align}
    Clearly, the man should follow the path perpendicular to \eqref{eq:chapters/11/10/4/24/L3} from
    $\vec{A}$ to reach it in the shortest time. The normal vector 
    of \eqref{eq:chapters/11/10/4/24/L3} is 
    \begin{align}
        \vec{m} = \myvec{6\\-7}
        \label{eq:chapters/11/10/4/24/L3-norm}
    \end{align}
    which is consequently the direction vector of the required line. Therefore, 
    the required normal vector is given by
    \begin{align}
        \vec{n} = \myvec{7\\6}
        \label{eq:chapters/11/10/4/24/L4-norm}
    \end{align}
    and hence, the equation of the line is
   \begin{align}
        \vec{n}^\top\vec{x} &= \vec{n}^\top\vec{A} \\
        \implies \myvec{7&6}\vec{x} &= \frac{1}{17}\myvec{7&6}\myvec{-1\\22} = \frac{125}{17}
        \label{eq:chapters/11/10/4/24/L4}
    \end{align}
		See Fig. \ref{fig:chapters/11/10/4/24/crossing}. In this figure $\vec{F}$ represents 
    the foot of the prependicular drawn from $\vec{A}$ onto \eqref{eq:chapters/11/10/4/24/L3}.

\item Find the points on the x-axis, whose distances from the line $\frac{x}{3}+\frac{y}{4}=1$ are 4 units.
\label{chapters/11/10/3/5}
	\\
	\solution
\iffalse
\documentclass[journal,12pt,twocolumn]{IEEEtran}
\usepackage{setspace}
\usepackage{gensymb}
\usepackage{xcolor}
\usepackage{caption}
\singlespacing
\usepackage{siunitx}
\usepackage[cmex10]{amsmath}
\usepackage{mathtools}
\usepackage{hyperref}
\usepackage{amsthm}
\usepackage{mathrsfs}
\usepackage{txfonts}
\usepackage{stfloats}
\usepackage{cite}
\usepackage{cases}
\usepackage{subfig}
\usepackage{longtable}
\usepackage{multirow}
\usepackage{enumitem}
\usepackage{mathtools}
\usepackage{listings}
\usepackage{tikz}
\usetikzlibrary{shapes,arrows,positioning}
\usepackage{circuitikz}
\let\vec\mathbf
\DeclareMathOperator*{\Res}{Res}
\renewcommand\thesection{\arabic{section}}
\renewcommand\thesubsection{\thesection.\arabic{subsection}}
\renewcommand\thesubsubsection{\thesubsection.\arabic{subsubsection}}

\renewcommand\thesectiondis{\arabic{section}}
\renewcommand\thesubsectiondis{\thesectiondis.\arabic{subsection}}
\renewcommand\thesubsubsectiondis{\thesubsectiondis.\arabic{subsubsection}}
\hyphenation{op-tical net-works semi-conduc-tor}

\lstset{
language=Python,
frame=single, 
breaklines=true,
columns=fullflexible
}
\begin{document}
\theoremstyle{definition}
\newtheorem{theorem}{Theorem}[section]
\newtheorem{problem}{Problem}
\newtheorem{proposition}{Proposition}[section]
\newtheorem{lemma}{Lemma}[section]
\newtheorem{corollary}[theorem]{Corollary}
\newtheorem{example}{Example}[section]
\newtheorem{definition}{Definition}[section]
\newcommand{\BEQA}{\begin{eqnarray}}
\newcommand{\EEQA}{\end{eqnarray}}
\newcommand{\define}{\stackrel{\triangle}{=}}
\newenvironment{amatrix}[1]{%
  \left(\begin{array}{@{}*{#1}{c}|c@{}}
}{%
  \end{array}\right)
}
\newcommand{\myvec}[1]{\ensuremath{\begin{pmatrix}#1\end{pmatrix}}}
\newcommand{\myaugvec}[2]{\ensuremath{\begin{amatrix}{#1}#2\end{amatrix}}}
\newcommand{\mydet}[1]{\ensuremath{\begin{vmatrix}#1\end{vmatrix}}}
\bibliographystyle{IEEEtran}
\providecommand{\nCr}[2]{\,^{#1}C_{#2}} % nCr
\providecommand{\nPr}[2]{\,^{#1}P_{#2}} % nPr
\providecommand{\mbf}{\mathbf}
\providecommand{\pr}[1]{\ensuremath{\Pr\left(#1\right)}}
\providecommand{\qfunc}[1]{\ensuremath{Q\left(#1\right)}}
\providecommand{\sbrak}[1]{\ensuremath{{}\left[#1\right]}}
\providecommand{\lsbrak}[1]{\ensuremath{{}\left[#1\right.}}
\providecommand{\rsbrak}[1]{\ensuremath{{}\left.#1\right]}}
\providecommand{\brak}[1]{\ensuremath{\left(#1\right)}}
\providecommand{\lbrak}[1]{\ensuremath{\left(#1\right.}}
\providecommand{\rbrak}[1]{\ensuremath{\left.#1\right)}}
\providecommand{\cbrak}[1]{\ensuremath{\left\{#1\right\}}}
\providecommand{\lcbrak}[1]{\ensuremath{\left\{#1\right.}}
\providecommand{\rcbrak}[1]{\ensuremath{\left.#1\right\}}}
\theoremstyle{remark}
\newtheorem{rem}{Remark}
\newcommand{\sgn}{\mathop{\mathrm{sgn}}}
\newcommand{\rect}{\mathop{\mathrm{rect}}}
\newcommand{\sinc}{\mathop{\mathrm{sinc}}}
\providecommand{\abs}[1]{\left\vert#1\right\vert}
\providecommand{\res}[1]{\Res\displaylimits_{#1}} 
\providecommand{\norm}[1]{\left\Vert#1\right\Vert}
\providecommand{\mtx}[1]{\mathbf{#1}}
\providecommand{\mean}[1]{E\left[ #1 \right]}
\providecommand{\fourier}{\overset{\mathcal{F}}{ \rightleftharpoons}}
\providecommand{\ztrans}{\overset{\mathcal{Z}}{ \rightleftharpoons}}
\providecommand{\system}[1]{\overset{\mathcal{#1}}{ \longleftrightarrow}}
\newcommand{\solution}{\noindent \textbf{Solution: }}
\providecommand{\dec}[2]{\ensuremath{\overset{#1}{\underset{#2}{\gtrless}}}}
\let\StandardTheFigure\thefigure
\def\putbox#1#2#3{\makebox[0in][l]{\makebox[#1][l]{}\raisebox{\baselineskip}[0in][0in]{\raisebox{#2}[0in][0in]{#3}}}}
     \def\rightbox#1{\makebox[0in][r]{#1}}
     \def\centbox#1{\makebox[0in]{#1}}
     \def\topbox#1{\raisebox{-\baselineskip}[0in][0in]{#1}}
     \def\midbox#1{\raisebox{-0.5\baselineskip}[0in][0in]{#1}}

\vspace{3cm}
\title{Line Assignment}
\author{Gautam Singh}
\maketitle
\bigskip

\begin{abstract}
    This document contains the solution to Question 24 of Exercise 4 
    in Chapter 10 of the class 11 NCERT textbook.
\end{abstract}

\begin{enumerate}
\fi
		We first find the coordinates of the intersection of \eqref{eq:chapters/11/10/4/24/L1}
    and \eqref{eq:chapters/11/10/4/24/L2}. Using the augmented matrix and row reduction methods,
    \begin{align}
        \myaugvec{2}{2&-3&-4\\3&4&5} &\xleftrightarrow[]{R_2\rightarrow2R_2-3R_1} 
        \myaugvec{2}{2&-3&-4\\0&17&22} \\
                      &\xleftrightarrow[]{R_1\rightarrow17R_1+3R_2} \myaugvec{2}{17&0&-1\\0&17&22} \\
                      &\xleftrightarrow[]{\substack{R_1\rightarrow\frac{R_1}{17}\\R_2\rightarrow\frac{R_2}{17}}} \myaugvec{2}{1&0&-\frac{1}{17}\\0&1&\frac{22}{17}}
        \label{eq:chapters/11/10/4/24/intersect}
    \end{align}
    the intersection of the lines is
    \begin{align}
        \vec{A} = \frac{1}{17}\myvec{-1\\22}
    \end{align}
    Clearly, the man should follow the path perpendicular to \eqref{eq:chapters/11/10/4/24/L3} from
    $\vec{A}$ to reach it in the shortest time. The normal vector 
    of \eqref{eq:chapters/11/10/4/24/L3} is 
    \begin{align}
        \vec{m} = \myvec{6\\-7}
        \label{eq:chapters/11/10/4/24/L3-norm}
    \end{align}
    which is consequently the direction vector of the required line. Therefore, 
    the required normal vector is given by
    \begin{align}
        \vec{n} = \myvec{7\\6}
        \label{eq:chapters/11/10/4/24/L4-norm}
    \end{align}
    and hence, the equation of the line is
   \begin{align}
        \vec{n}^\top\vec{x} &= \vec{n}^\top\vec{A} \\
        \implies \myvec{7&6}\vec{x} &= \frac{1}{17}\myvec{7&6}\myvec{-1\\22} = \frac{125}{17}
        \label{eq:chapters/11/10/4/24/L4}
    \end{align}
		See Fig. \ref{fig:chapters/11/10/4/24/crossing}. In this figure $\vec{F}$ represents 
    the foot of the prependicular drawn from $\vec{A}$ onto \eqref{eq:chapters/11/10/4/24/L3}.

\item Find the distance between parallel lines
\label{chapters/11/10/3/6}
\begin{enumerate}
	\item $15x+8y-34=0$ and  $15x+8y+31=0$ \\
	\item  $l(x+y)+p=0$ and  $l(x+y)-r=0$
\end{enumerate}
	\solution
\input{chapters/11/10/3/6/dist.tex}
\item Find the coordinates of the foot of the perpendicular from $(-1, 3)$ to the line $3x-4y-16=0$.  
\label{chapters/11/10/3/14}
\\
\solution
\iffalse
\documentclass[journal,12pt,twocolumn]{IEEEtran}
\usepackage{setspace}
\usepackage{gensymb}
\usepackage{xcolor}
\usepackage{caption}
\singlespacing
\usepackage{siunitx}
\usepackage[cmex10]{amsmath}
\usepackage{mathtools}
\usepackage{hyperref}
\usepackage{amsthm}
\usepackage{mathrsfs}
\usepackage{txfonts}
\usepackage{stfloats}
\usepackage{cite}
\usepackage{cases}
\usepackage{subfig}
\usepackage{longtable}
\usepackage{multirow}
\usepackage{enumitem}
\usepackage{mathtools}
\usepackage{listings}
\usepackage{tikz}
\usetikzlibrary{shapes,arrows,positioning}
\usepackage{circuitikz}
\let\vec\mathbf
\DeclareMathOperator*{\Res}{Res}
\renewcommand\thesection{\arabic{section}}
\renewcommand\thesubsection{\thesection.\arabic{subsection}}
\renewcommand\thesubsubsection{\thesubsection.\arabic{subsubsection}}

\renewcommand\thesectiondis{\arabic{section}}
\renewcommand\thesubsectiondis{\thesectiondis.\arabic{subsection}}
\renewcommand\thesubsubsectiondis{\thesubsectiondis.\arabic{subsubsection}}
\hyphenation{op-tical net-works semi-conduc-tor}

\lstset{
language=Python,
frame=single, 
breaklines=true,
columns=fullflexible
}
\begin{document}
\theoremstyle{definition}
\newtheorem{theorem}{Theorem}[section]
\newtheorem{problem}{Problem}
\newtheorem{proposition}{Proposition}[section]
\newtheorem{lemma}{Lemma}[section]
\newtheorem{corollary}[theorem]{Corollary}
\newtheorem{example}{Example}[section]
\newtheorem{definition}{Definition}[section]
\newcommand{\BEQA}{\begin{eqnarray}}
\newcommand{\EEQA}{\end{eqnarray}}
\newcommand{\define}{\stackrel{\triangle}{=}}
\newenvironment{amatrix}[1]{%
  \left(\begin{array}{@{}*{#1}{c}|c@{}}
}{%
  \end{array}\right)
}
\newcommand{\myvec}[1]{\ensuremath{\begin{pmatrix}#1\end{pmatrix}}}
\newcommand{\myaugvec}[2]{\ensuremath{\begin{amatrix}{#1}#2\end{amatrix}}}
\newcommand{\mydet}[1]{\ensuremath{\begin{vmatrix}#1\end{vmatrix}}}
\bibliographystyle{IEEEtran}
\providecommand{\nCr}[2]{\,^{#1}C_{#2}} % nCr
\providecommand{\nPr}[2]{\,^{#1}P_{#2}} % nPr
\providecommand{\mbf}{\mathbf}
\providecommand{\pr}[1]{\ensuremath{\Pr\left(#1\right)}}
\providecommand{\qfunc}[1]{\ensuremath{Q\left(#1\right)}}
\providecommand{\sbrak}[1]{\ensuremath{{}\left[#1\right]}}
\providecommand{\lsbrak}[1]{\ensuremath{{}\left[#1\right.}}
\providecommand{\rsbrak}[1]{\ensuremath{{}\left.#1\right]}}
\providecommand{\brak}[1]{\ensuremath{\left(#1\right)}}
\providecommand{\lbrak}[1]{\ensuremath{\left(#1\right.}}
\providecommand{\rbrak}[1]{\ensuremath{\left.#1\right)}}
\providecommand{\cbrak}[1]{\ensuremath{\left\{#1\right\}}}
\providecommand{\lcbrak}[1]{\ensuremath{\left\{#1\right.}}
\providecommand{\rcbrak}[1]{\ensuremath{\left.#1\right\}}}
\theoremstyle{remark}
\newtheorem{rem}{Remark}
\newcommand{\sgn}{\mathop{\mathrm{sgn}}}
\newcommand{\rect}{\mathop{\mathrm{rect}}}
\newcommand{\sinc}{\mathop{\mathrm{sinc}}}
\providecommand{\abs}[1]{\left\vert#1\right\vert}
\providecommand{\res}[1]{\Res\displaylimits_{#1}} 
\providecommand{\norm}[1]{\left\Vert#1\right\Vert}
\providecommand{\mtx}[1]{\mathbf{#1}}
\providecommand{\mean}[1]{E\left[ #1 \right]}
\providecommand{\fourier}{\overset{\mathcal{F}}{ \rightleftharpoons}}
\providecommand{\ztrans}{\overset{\mathcal{Z}}{ \rightleftharpoons}}
\providecommand{\system}[1]{\overset{\mathcal{#1}}{ \longleftrightarrow}}
\newcommand{\solution}{\noindent \textbf{Solution: }}
\providecommand{\dec}[2]{\ensuremath{\overset{#1}{\underset{#2}{\gtrless}}}}
\let\StandardTheFigure\thefigure
\def\putbox#1#2#3{\makebox[0in][l]{\makebox[#1][l]{}\raisebox{\baselineskip}[0in][0in]{\raisebox{#2}[0in][0in]{#3}}}}
     \def\rightbox#1{\makebox[0in][r]{#1}}
     \def\centbox#1{\makebox[0in]{#1}}
     \def\topbox#1{\raisebox{-\baselineskip}[0in][0in]{#1}}
     \def\midbox#1{\raisebox{-0.5\baselineskip}[0in][0in]{#1}}

\vspace{3cm}
\title{Line Assignment}
\author{Gautam Singh}
\maketitle
\bigskip

\begin{abstract}
    This document contains the solution to Question 24 of Exercise 4 
    in Chapter 10 of the class 11 NCERT textbook.
\end{abstract}

\begin{enumerate}
\fi
		We first find the coordinates of the intersection of \eqref{eq:chapters/11/10/4/24/L1}
    and \eqref{eq:chapters/11/10/4/24/L2}. Using the augmented matrix and row reduction methods,
    \begin{align}
        \myaugvec{2}{2&-3&-4\\3&4&5} &\xleftrightarrow[]{R_2\rightarrow2R_2-3R_1} 
        \myaugvec{2}{2&-3&-4\\0&17&22} \\
                      &\xleftrightarrow[]{R_1\rightarrow17R_1+3R_2} \myaugvec{2}{17&0&-1\\0&17&22} \\
                      &\xleftrightarrow[]{\substack{R_1\rightarrow\frac{R_1}{17}\\R_2\rightarrow\frac{R_2}{17}}} \myaugvec{2}{1&0&-\frac{1}{17}\\0&1&\frac{22}{17}}
        \label{eq:chapters/11/10/4/24/intersect}
    \end{align}
    the intersection of the lines is
    \begin{align}
        \vec{A} = \frac{1}{17}\myvec{-1\\22}
    \end{align}
    Clearly, the man should follow the path perpendicular to \eqref{eq:chapters/11/10/4/24/L3} from
    $\vec{A}$ to reach it in the shortest time. The normal vector 
    of \eqref{eq:chapters/11/10/4/24/L3} is 
    \begin{align}
        \vec{m} = \myvec{6\\-7}
        \label{eq:chapters/11/10/4/24/L3-norm}
    \end{align}
    which is consequently the direction vector of the required line. Therefore, 
    the required normal vector is given by
    \begin{align}
        \vec{n} = \myvec{7\\6}
        \label{eq:chapters/11/10/4/24/L4-norm}
    \end{align}
    and hence, the equation of the line is
   \begin{align}
        \vec{n}^\top\vec{x} &= \vec{n}^\top\vec{A} \\
        \implies \myvec{7&6}\vec{x} &= \frac{1}{17}\myvec{7&6}\myvec{-1\\22} = \frac{125}{17}
        \label{eq:chapters/11/10/4/24/L4}
    \end{align}
		See Fig. \ref{fig:chapters/11/10/4/24/crossing}. In this figure $\vec{F}$ represents 
    the foot of the prependicular drawn from $\vec{A}$ onto \eqref{eq:chapters/11/10/4/24/L3}.

\item  If ${p}$ and ${q}$ are the lengths of perpendiculars from the origin to the lines ${x}\cos\theta - {y}\sin\theta =  {k}\cos2\theta$ and ${x}\sec\theta + {y}\cosec\theta = {k}$, respectively, prove that ${p}^2 + 4{q}^2 = {k}^2$
\label{chapters/11/10/3/16}
\\
\solution
\iffalse
\documentclass[journal,12pt,twocolumn]{IEEEtran}
\usepackage{setspace}
\usepackage{gensymb}
\usepackage{xcolor}
\usepackage{caption}
\singlespacing
\usepackage{siunitx}
\usepackage[cmex10]{amsmath}
\usepackage{mathtools}
\usepackage{hyperref}
\usepackage{amsthm}
\usepackage{mathrsfs}
\usepackage{txfonts}
\usepackage{stfloats}
\usepackage{cite}
\usepackage{cases}
\usepackage{subfig}
\usepackage{longtable}
\usepackage{multirow}
\usepackage{enumitem}
\usepackage{mathtools}
\usepackage{listings}
\usepackage{tikz}
\usetikzlibrary{shapes,arrows,positioning}
\usepackage{circuitikz}
\let\vec\mathbf
\DeclareMathOperator*{\Res}{Res}
\renewcommand\thesection{\arabic{section}}
\renewcommand\thesubsection{\thesection.\arabic{subsection}}
\renewcommand\thesubsubsection{\thesubsection.\arabic{subsubsection}}

\renewcommand\thesectiondis{\arabic{section}}
\renewcommand\thesubsectiondis{\thesectiondis.\arabic{subsection}}
\renewcommand\thesubsubsectiondis{\thesubsectiondis.\arabic{subsubsection}}
\hyphenation{op-tical net-works semi-conduc-tor}

\lstset{
language=Python,
frame=single, 
breaklines=true,
columns=fullflexible
}
\begin{document}
\theoremstyle{definition}
\newtheorem{theorem}{Theorem}[section]
\newtheorem{problem}{Problem}
\newtheorem{proposition}{Proposition}[section]
\newtheorem{lemma}{Lemma}[section]
\newtheorem{corollary}[theorem]{Corollary}
\newtheorem{example}{Example}[section]
\newtheorem{definition}{Definition}[section]
\newcommand{\BEQA}{\begin{eqnarray}}
\newcommand{\EEQA}{\end{eqnarray}}
\newcommand{\define}{\stackrel{\triangle}{=}}
\newenvironment{amatrix}[1]{%
  \left(\begin{array}{@{}*{#1}{c}|c@{}}
}{%
  \end{array}\right)
}
\newcommand{\myvec}[1]{\ensuremath{\begin{pmatrix}#1\end{pmatrix}}}
\newcommand{\myaugvec}[2]{\ensuremath{\begin{amatrix}{#1}#2\end{amatrix}}}
\newcommand{\mydet}[1]{\ensuremath{\begin{vmatrix}#1\end{vmatrix}}}
\bibliographystyle{IEEEtran}
\providecommand{\nCr}[2]{\,^{#1}C_{#2}} % nCr
\providecommand{\nPr}[2]{\,^{#1}P_{#2}} % nPr
\providecommand{\mbf}{\mathbf}
\providecommand{\pr}[1]{\ensuremath{\Pr\left(#1\right)}}
\providecommand{\qfunc}[1]{\ensuremath{Q\left(#1\right)}}
\providecommand{\sbrak}[1]{\ensuremath{{}\left[#1\right]}}
\providecommand{\lsbrak}[1]{\ensuremath{{}\left[#1\right.}}
\providecommand{\rsbrak}[1]{\ensuremath{{}\left.#1\right]}}
\providecommand{\brak}[1]{\ensuremath{\left(#1\right)}}
\providecommand{\lbrak}[1]{\ensuremath{\left(#1\right.}}
\providecommand{\rbrak}[1]{\ensuremath{\left.#1\right)}}
\providecommand{\cbrak}[1]{\ensuremath{\left\{#1\right\}}}
\providecommand{\lcbrak}[1]{\ensuremath{\left\{#1\right.}}
\providecommand{\rcbrak}[1]{\ensuremath{\left.#1\right\}}}
\theoremstyle{remark}
\newtheorem{rem}{Remark}
\newcommand{\sgn}{\mathop{\mathrm{sgn}}}
\newcommand{\rect}{\mathop{\mathrm{rect}}}
\newcommand{\sinc}{\mathop{\mathrm{sinc}}}
\providecommand{\abs}[1]{\left\vert#1\right\vert}
\providecommand{\res}[1]{\Res\displaylimits_{#1}} 
\providecommand{\norm}[1]{\left\Vert#1\right\Vert}
\providecommand{\mtx}[1]{\mathbf{#1}}
\providecommand{\mean}[1]{E\left[ #1 \right]}
\providecommand{\fourier}{\overset{\mathcal{F}}{ \rightleftharpoons}}
\providecommand{\ztrans}{\overset{\mathcal{Z}}{ \rightleftharpoons}}
\providecommand{\system}[1]{\overset{\mathcal{#1}}{ \longleftrightarrow}}
\newcommand{\solution}{\noindent \textbf{Solution: }}
\providecommand{\dec}[2]{\ensuremath{\overset{#1}{\underset{#2}{\gtrless}}}}
\let\StandardTheFigure\thefigure
\def\putbox#1#2#3{\makebox[0in][l]{\makebox[#1][l]{}\raisebox{\baselineskip}[0in][0in]{\raisebox{#2}[0in][0in]{#3}}}}
     \def\rightbox#1{\makebox[0in][r]{#1}}
     \def\centbox#1{\makebox[0in]{#1}}
     \def\topbox#1{\raisebox{-\baselineskip}[0in][0in]{#1}}
     \def\midbox#1{\raisebox{-0.5\baselineskip}[0in][0in]{#1}}

\vspace{3cm}
\title{Line Assignment}
\author{Gautam Singh}
\maketitle
\bigskip

\begin{abstract}
    This document contains the solution to Question 24 of Exercise 4 
    in Chapter 10 of the class 11 NCERT textbook.
\end{abstract}

\begin{enumerate}
\fi
		We first find the coordinates of the intersection of \eqref{eq:chapters/11/10/4/24/L1}
    and \eqref{eq:chapters/11/10/4/24/L2}. Using the augmented matrix and row reduction methods,
    \begin{align}
        \myaugvec{2}{2&-3&-4\\3&4&5} &\xleftrightarrow[]{R_2\rightarrow2R_2-3R_1} 
        \myaugvec{2}{2&-3&-4\\0&17&22} \\
                      &\xleftrightarrow[]{R_1\rightarrow17R_1+3R_2} \myaugvec{2}{17&0&-1\\0&17&22} \\
                      &\xleftrightarrow[]{\substack{R_1\rightarrow\frac{R_1}{17}\\R_2\rightarrow\frac{R_2}{17}}} \myaugvec{2}{1&0&-\frac{1}{17}\\0&1&\frac{22}{17}}
        \label{eq:chapters/11/10/4/24/intersect}
    \end{align}
    the intersection of the lines is
    \begin{align}
        \vec{A} = \frac{1}{17}\myvec{-1\\22}
    \end{align}
    Clearly, the man should follow the path perpendicular to \eqref{eq:chapters/11/10/4/24/L3} from
    $\vec{A}$ to reach it in the shortest time. The normal vector 
    of \eqref{eq:chapters/11/10/4/24/L3} is 
    \begin{align}
        \vec{m} = \myvec{6\\-7}
        \label{eq:chapters/11/10/4/24/L3-norm}
    \end{align}
    which is consequently the direction vector of the required line. Therefore, 
    the required normal vector is given by
    \begin{align}
        \vec{n} = \myvec{7\\6}
        \label{eq:chapters/11/10/4/24/L4-norm}
    \end{align}
    and hence, the equation of the line is
   \begin{align}
        \vec{n}^\top\vec{x} &= \vec{n}^\top\vec{A} \\
        \implies \myvec{7&6}\vec{x} &= \frac{1}{17}\myvec{7&6}\myvec{-1\\22} = \frac{125}{17}
        \label{eq:chapters/11/10/4/24/L4}
    \end{align}
		See Fig. \ref{fig:chapters/11/10/4/24/crossing}. In this figure $\vec{F}$ represents 
    the foot of the prependicular drawn from $\vec{A}$ onto \eqref{eq:chapters/11/10/4/24/L3}.

\item In the triangle $ABC$ with vertices $\vec{A} \brak{2, 3}$, $\vec{B} \brak{4, –1}$ and $\vec{C} \brak{1, 2}$, find the equation and length of altitude from the vertex $\vec{A}$.
\label{chapters/11/10/3/17}
\\
\solution
\iffalse
\documentclass[journal,12pt,twocolumn]{IEEEtran}
\usepackage{setspace}
\usepackage{gensymb}
\usepackage{xcolor}
\usepackage{caption}
\singlespacing
\usepackage{siunitx}
\usepackage[cmex10]{amsmath}
\usepackage{mathtools}
\usepackage{hyperref}
\usepackage{amsthm}
\usepackage{mathrsfs}
\usepackage{txfonts}
\usepackage{stfloats}
\usepackage{cite}
\usepackage{cases}
\usepackage{subfig}
\usepackage{longtable}
\usepackage{multirow}
\usepackage{enumitem}
\usepackage{mathtools}
\usepackage{listings}
\usepackage{tikz}
\usetikzlibrary{shapes,arrows,positioning}
\usepackage{circuitikz}
\let\vec\mathbf
\DeclareMathOperator*{\Res}{Res}
\renewcommand\thesection{\arabic{section}}
\renewcommand\thesubsection{\thesection.\arabic{subsection}}
\renewcommand\thesubsubsection{\thesubsection.\arabic{subsubsection}}

\renewcommand\thesectiondis{\arabic{section}}
\renewcommand\thesubsectiondis{\thesectiondis.\arabic{subsection}}
\renewcommand\thesubsubsectiondis{\thesubsectiondis.\arabic{subsubsection}}
\hyphenation{op-tical net-works semi-conduc-tor}

\lstset{
language=Python,
frame=single, 
breaklines=true,
columns=fullflexible
}
\begin{document}
\theoremstyle{definition}
\newtheorem{theorem}{Theorem}[section]
\newtheorem{problem}{Problem}
\newtheorem{proposition}{Proposition}[section]
\newtheorem{lemma}{Lemma}[section]
\newtheorem{corollary}[theorem]{Corollary}
\newtheorem{example}{Example}[section]
\newtheorem{definition}{Definition}[section]
\newcommand{\BEQA}{\begin{eqnarray}}
\newcommand{\EEQA}{\end{eqnarray}}
\newcommand{\define}{\stackrel{\triangle}{=}}
\newenvironment{amatrix}[1]{%
  \left(\begin{array}{@{}*{#1}{c}|c@{}}
}{%
  \end{array}\right)
}
\newcommand{\myvec}[1]{\ensuremath{\begin{pmatrix}#1\end{pmatrix}}}
\newcommand{\myaugvec}[2]{\ensuremath{\begin{amatrix}{#1}#2\end{amatrix}}}
\newcommand{\mydet}[1]{\ensuremath{\begin{vmatrix}#1\end{vmatrix}}}
\bibliographystyle{IEEEtran}
\providecommand{\nCr}[2]{\,^{#1}C_{#2}} % nCr
\providecommand{\nPr}[2]{\,^{#1}P_{#2}} % nPr
\providecommand{\mbf}{\mathbf}
\providecommand{\pr}[1]{\ensuremath{\Pr\left(#1\right)}}
\providecommand{\qfunc}[1]{\ensuremath{Q\left(#1\right)}}
\providecommand{\sbrak}[1]{\ensuremath{{}\left[#1\right]}}
\providecommand{\lsbrak}[1]{\ensuremath{{}\left[#1\right.}}
\providecommand{\rsbrak}[1]{\ensuremath{{}\left.#1\right]}}
\providecommand{\brak}[1]{\ensuremath{\left(#1\right)}}
\providecommand{\lbrak}[1]{\ensuremath{\left(#1\right.}}
\providecommand{\rbrak}[1]{\ensuremath{\left.#1\right)}}
\providecommand{\cbrak}[1]{\ensuremath{\left\{#1\right\}}}
\providecommand{\lcbrak}[1]{\ensuremath{\left\{#1\right.}}
\providecommand{\rcbrak}[1]{\ensuremath{\left.#1\right\}}}
\theoremstyle{remark}
\newtheorem{rem}{Remark}
\newcommand{\sgn}{\mathop{\mathrm{sgn}}}
\newcommand{\rect}{\mathop{\mathrm{rect}}}
\newcommand{\sinc}{\mathop{\mathrm{sinc}}}
\providecommand{\abs}[1]{\left\vert#1\right\vert}
\providecommand{\res}[1]{\Res\displaylimits_{#1}} 
\providecommand{\norm}[1]{\left\Vert#1\right\Vert}
\providecommand{\mtx}[1]{\mathbf{#1}}
\providecommand{\mean}[1]{E\left[ #1 \right]}
\providecommand{\fourier}{\overset{\mathcal{F}}{ \rightleftharpoons}}
\providecommand{\ztrans}{\overset{\mathcal{Z}}{ \rightleftharpoons}}
\providecommand{\system}[1]{\overset{\mathcal{#1}}{ \longleftrightarrow}}
\newcommand{\solution}{\noindent \textbf{Solution: }}
\providecommand{\dec}[2]{\ensuremath{\overset{#1}{\underset{#2}{\gtrless}}}}
\let\StandardTheFigure\thefigure
\def\putbox#1#2#3{\makebox[0in][l]{\makebox[#1][l]{}\raisebox{\baselineskip}[0in][0in]{\raisebox{#2}[0in][0in]{#3}}}}
     \def\rightbox#1{\makebox[0in][r]{#1}}
     \def\centbox#1{\makebox[0in]{#1}}
     \def\topbox#1{\raisebox{-\baselineskip}[0in][0in]{#1}}
     \def\midbox#1{\raisebox{-0.5\baselineskip}[0in][0in]{#1}}

\vspace{3cm}
\title{Line Assignment}
\author{Gautam Singh}
\maketitle
\bigskip

\begin{abstract}
    This document contains the solution to Question 24 of Exercise 4 
    in Chapter 10 of the class 11 NCERT textbook.
\end{abstract}

\begin{enumerate}
\fi
		We first find the coordinates of the intersection of \eqref{eq:chapters/11/10/4/24/L1}
    and \eqref{eq:chapters/11/10/4/24/L2}. Using the augmented matrix and row reduction methods,
    \begin{align}
        \myaugvec{2}{2&-3&-4\\3&4&5} &\xleftrightarrow[]{R_2\rightarrow2R_2-3R_1} 
        \myaugvec{2}{2&-3&-4\\0&17&22} \\
                      &\xleftrightarrow[]{R_1\rightarrow17R_1+3R_2} \myaugvec{2}{17&0&-1\\0&17&22} \\
                      &\xleftrightarrow[]{\substack{R_1\rightarrow\frac{R_1}{17}\\R_2\rightarrow\frac{R_2}{17}}} \myaugvec{2}{1&0&-\frac{1}{17}\\0&1&\frac{22}{17}}
        \label{eq:chapters/11/10/4/24/intersect}
    \end{align}
    the intersection of the lines is
    \begin{align}
        \vec{A} = \frac{1}{17}\myvec{-1\\22}
    \end{align}
    Clearly, the man should follow the path perpendicular to \eqref{eq:chapters/11/10/4/24/L3} from
    $\vec{A}$ to reach it in the shortest time. The normal vector 
    of \eqref{eq:chapters/11/10/4/24/L3} is 
    \begin{align}
        \vec{m} = \myvec{6\\-7}
        \label{eq:chapters/11/10/4/24/L3-norm}
    \end{align}
    which is consequently the direction vector of the required line. Therefore, 
    the required normal vector is given by
    \begin{align}
        \vec{n} = \myvec{7\\6}
        \label{eq:chapters/11/10/4/24/L4-norm}
    \end{align}
    and hence, the equation of the line is
   \begin{align}
        \vec{n}^\top\vec{x} &= \vec{n}^\top\vec{A} \\
        \implies \myvec{7&6}\vec{x} &= \frac{1}{17}\myvec{7&6}\myvec{-1\\22} = \frac{125}{17}
        \label{eq:chapters/11/10/4/24/L4}
    \end{align}
		See Fig. \ref{fig:chapters/11/10/4/24/crossing}. In this figure $\vec{F}$ represents 
    the foot of the prependicular drawn from $\vec{A}$ onto \eqref{eq:chapters/11/10/4/24/L3}.

\item If $p$ is the length of perpendicular from origin to the line whose intercepts on the axes are $a$ and $b$, then show that 
\begin{align}
	\frac{1}{p^2} = \frac{1}{a^2}+ \frac{1}{b^2}
\end{align}
\label{chapters/11/10/3/18}
\input{chapters/11/10/3/18/dist.tex}
\item What are the points on the y-axis whose distance from the line $\frac{x}{3}+\frac{y}{4}=1$ is 4 units.
\\
\solution
		\iffalse
\documentclass[journal,12pt,twocolumn]{IEEEtran}
\usepackage{setspace}
\usepackage{gensymb}
\usepackage{xcolor}
\usepackage{caption}
\singlespacing
\usepackage{siunitx}
\usepackage[cmex10]{amsmath}
\usepackage{mathtools}
\usepackage{hyperref}
\usepackage{amsthm}
\usepackage{mathrsfs}
\usepackage{txfonts}
\usepackage{stfloats}
\usepackage{cite}
\usepackage{cases}
\usepackage{subfig}
\usepackage{longtable}
\usepackage{multirow}
\usepackage{enumitem}
\usepackage{mathtools}
\usepackage{listings}
\usepackage{tikz}
\usetikzlibrary{shapes,arrows,positioning}
\usepackage{circuitikz}
\let\vec\mathbf
\DeclareMathOperator*{\Res}{Res}
\renewcommand\thesection{\arabic{section}}
\renewcommand\thesubsection{\thesection.\arabic{subsection}}
\renewcommand\thesubsubsection{\thesubsection.\arabic{subsubsection}}

\renewcommand\thesectiondis{\arabic{section}}
\renewcommand\thesubsectiondis{\thesectiondis.\arabic{subsection}}
\renewcommand\thesubsubsectiondis{\thesubsectiondis.\arabic{subsubsection}}
\hyphenation{op-tical net-works semi-conduc-tor}

\lstset{
language=Python,
frame=single, 
breaklines=true,
columns=fullflexible
}
\begin{document}
\theoremstyle{definition}
\newtheorem{theorem}{Theorem}[section]
\newtheorem{problem}{Problem}
\newtheorem{proposition}{Proposition}[section]
\newtheorem{lemma}{Lemma}[section]
\newtheorem{corollary}[theorem]{Corollary}
\newtheorem{example}{Example}[section]
\newtheorem{definition}{Definition}[section]
\newcommand{\BEQA}{\begin{eqnarray}}
\newcommand{\EEQA}{\end{eqnarray}}
\newcommand{\define}{\stackrel{\triangle}{=}}
\newenvironment{amatrix}[1]{%
  \left(\begin{array}{@{}*{#1}{c}|c@{}}
}{%
  \end{array}\right)
}
\newcommand{\myvec}[1]{\ensuremath{\begin{pmatrix}#1\end{pmatrix}}}
\newcommand{\myaugvec}[2]{\ensuremath{\begin{amatrix}{#1}#2\end{amatrix}}}
\newcommand{\mydet}[1]{\ensuremath{\begin{vmatrix}#1\end{vmatrix}}}
\bibliographystyle{IEEEtran}
\providecommand{\nCr}[2]{\,^{#1}C_{#2}} % nCr
\providecommand{\nPr}[2]{\,^{#1}P_{#2}} % nPr
\providecommand{\mbf}{\mathbf}
\providecommand{\pr}[1]{\ensuremath{\Pr\left(#1\right)}}
\providecommand{\qfunc}[1]{\ensuremath{Q\left(#1\right)}}
\providecommand{\sbrak}[1]{\ensuremath{{}\left[#1\right]}}
\providecommand{\lsbrak}[1]{\ensuremath{{}\left[#1\right.}}
\providecommand{\rsbrak}[1]{\ensuremath{{}\left.#1\right]}}
\providecommand{\brak}[1]{\ensuremath{\left(#1\right)}}
\providecommand{\lbrak}[1]{\ensuremath{\left(#1\right.}}
\providecommand{\rbrak}[1]{\ensuremath{\left.#1\right)}}
\providecommand{\cbrak}[1]{\ensuremath{\left\{#1\right\}}}
\providecommand{\lcbrak}[1]{\ensuremath{\left\{#1\right.}}
\providecommand{\rcbrak}[1]{\ensuremath{\left.#1\right\}}}
\theoremstyle{remark}
\newtheorem{rem}{Remark}
\newcommand{\sgn}{\mathop{\mathrm{sgn}}}
\newcommand{\rect}{\mathop{\mathrm{rect}}}
\newcommand{\sinc}{\mathop{\mathrm{sinc}}}
\providecommand{\abs}[1]{\left\vert#1\right\vert}
\providecommand{\res}[1]{\Res\displaylimits_{#1}} 
\providecommand{\norm}[1]{\left\Vert#1\right\Vert}
\providecommand{\mtx}[1]{\mathbf{#1}}
\providecommand{\mean}[1]{E\left[ #1 \right]}
\providecommand{\fourier}{\overset{\mathcal{F}}{ \rightleftharpoons}}
\providecommand{\ztrans}{\overset{\mathcal{Z}}{ \rightleftharpoons}}
\providecommand{\system}[1]{\overset{\mathcal{#1}}{ \longleftrightarrow}}
\newcommand{\solution}{\noindent \textbf{Solution: }}
\providecommand{\dec}[2]{\ensuremath{\overset{#1}{\underset{#2}{\gtrless}}}}
\let\StandardTheFigure\thefigure
\def\putbox#1#2#3{\makebox[0in][l]{\makebox[#1][l]{}\raisebox{\baselineskip}[0in][0in]{\raisebox{#2}[0in][0in]{#3}}}}
     \def\rightbox#1{\makebox[0in][r]{#1}}
     \def\centbox#1{\makebox[0in]{#1}}
     \def\topbox#1{\raisebox{-\baselineskip}[0in][0in]{#1}}
     \def\midbox#1{\raisebox{-0.5\baselineskip}[0in][0in]{#1}}

\vspace{3cm}
\title{Line Assignment}
\author{Gautam Singh}
\maketitle
\bigskip

\begin{abstract}
    This document contains the solution to Question 24 of Exercise 4 
    in Chapter 10 of the class 11 NCERT textbook.
\end{abstract}

\begin{enumerate}
\fi
		We first find the coordinates of the intersection of \eqref{eq:chapters/11/10/4/24/L1}
    and \eqref{eq:chapters/11/10/4/24/L2}. Using the augmented matrix and row reduction methods,
    \begin{align}
        \myaugvec{2}{2&-3&-4\\3&4&5} &\xleftrightarrow[]{R_2\rightarrow2R_2-3R_1} 
        \myaugvec{2}{2&-3&-4\\0&17&22} \\
                      &\xleftrightarrow[]{R_1\rightarrow17R_1+3R_2} \myaugvec{2}{17&0&-1\\0&17&22} \\
                      &\xleftrightarrow[]{\substack{R_1\rightarrow\frac{R_1}{17}\\R_2\rightarrow\frac{R_2}{17}}} \myaugvec{2}{1&0&-\frac{1}{17}\\0&1&\frac{22}{17}}
        \label{eq:chapters/11/10/4/24/intersect}
    \end{align}
    the intersection of the lines is
    \begin{align}
        \vec{A} = \frac{1}{17}\myvec{-1\\22}
    \end{align}
    Clearly, the man should follow the path perpendicular to \eqref{eq:chapters/11/10/4/24/L3} from
    $\vec{A}$ to reach it in the shortest time. The normal vector 
    of \eqref{eq:chapters/11/10/4/24/L3} is 
    \begin{align}
        \vec{m} = \myvec{6\\-7}
        \label{eq:chapters/11/10/4/24/L3-norm}
    \end{align}
    which is consequently the direction vector of the required line. Therefore, 
    the required normal vector is given by
    \begin{align}
        \vec{n} = \myvec{7\\6}
        \label{eq:chapters/11/10/4/24/L4-norm}
    \end{align}
    and hence, the equation of the line is
   \begin{align}
        \vec{n}^\top\vec{x} &= \vec{n}^\top\vec{A} \\
        \implies \myvec{7&6}\vec{x} &= \frac{1}{17}\myvec{7&6}\myvec{-1\\22} = \frac{125}{17}
        \label{eq:chapters/11/10/4/24/L4}
    \end{align}
		See Fig. \ref{fig:chapters/11/10/4/24/crossing}. In this figure $\vec{F}$ represents 
    the foot of the prependicular drawn from $\vec{A}$ onto \eqref{eq:chapters/11/10/4/24/L3}.

\item Find perpendicular distance from the origin to the line joining the points$(\cos\theta,\sin\theta)$ and $(\cos\phi,\sin\phi)$.
\\
\solution
		\input{chapters/11/10/4/5/dist.tex}
\item Find the equation of line which is equidistant from parallel lines $9x+6y-7=0$ and $3x+2y+6=0$.
\\
\solution
		\iffalse
\documentclass[journal,12pt,twocolumn]{IEEEtran}
\usepackage{setspace}
\usepackage{gensymb}
\usepackage{xcolor}
\usepackage{caption}
\singlespacing
\usepackage{siunitx}
\usepackage[cmex10]{amsmath}
\usepackage{mathtools}
\usepackage{hyperref}
\usepackage{amsthm}
\usepackage{mathrsfs}
\usepackage{txfonts}
\usepackage{stfloats}
\usepackage{cite}
\usepackage{cases}
\usepackage{subfig}
\usepackage{longtable}
\usepackage{multirow}
\usepackage{enumitem}
\usepackage{mathtools}
\usepackage{listings}
\usepackage{tikz}
\usetikzlibrary{shapes,arrows,positioning}
\usepackage{circuitikz}
\let\vec\mathbf
\DeclareMathOperator*{\Res}{Res}
\renewcommand\thesection{\arabic{section}}
\renewcommand\thesubsection{\thesection.\arabic{subsection}}
\renewcommand\thesubsubsection{\thesubsection.\arabic{subsubsection}}

\renewcommand\thesectiondis{\arabic{section}}
\renewcommand\thesubsectiondis{\thesectiondis.\arabic{subsection}}
\renewcommand\thesubsubsectiondis{\thesubsectiondis.\arabic{subsubsection}}
\hyphenation{op-tical net-works semi-conduc-tor}

\lstset{
language=Python,
frame=single, 
breaklines=true,
columns=fullflexible
}
\begin{document}
\theoremstyle{definition}
\newtheorem{theorem}{Theorem}[section]
\newtheorem{problem}{Problem}
\newtheorem{proposition}{Proposition}[section]
\newtheorem{lemma}{Lemma}[section]
\newtheorem{corollary}[theorem]{Corollary}
\newtheorem{example}{Example}[section]
\newtheorem{definition}{Definition}[section]
\newcommand{\BEQA}{\begin{eqnarray}}
\newcommand{\EEQA}{\end{eqnarray}}
\newcommand{\define}{\stackrel{\triangle}{=}}
\newenvironment{amatrix}[1]{%
  \left(\begin{array}{@{}*{#1}{c}|c@{}}
}{%
  \end{array}\right)
}
\newcommand{\myvec}[1]{\ensuremath{\begin{pmatrix}#1\end{pmatrix}}}
\newcommand{\myaugvec}[2]{\ensuremath{\begin{amatrix}{#1}#2\end{amatrix}}}
\newcommand{\mydet}[1]{\ensuremath{\begin{vmatrix}#1\end{vmatrix}}}
\bibliographystyle{IEEEtran}
\providecommand{\nCr}[2]{\,^{#1}C_{#2}} % nCr
\providecommand{\nPr}[2]{\,^{#1}P_{#2}} % nPr
\providecommand{\mbf}{\mathbf}
\providecommand{\pr}[1]{\ensuremath{\Pr\left(#1\right)}}
\providecommand{\qfunc}[1]{\ensuremath{Q\left(#1\right)}}
\providecommand{\sbrak}[1]{\ensuremath{{}\left[#1\right]}}
\providecommand{\lsbrak}[1]{\ensuremath{{}\left[#1\right.}}
\providecommand{\rsbrak}[1]{\ensuremath{{}\left.#1\right]}}
\providecommand{\brak}[1]{\ensuremath{\left(#1\right)}}
\providecommand{\lbrak}[1]{\ensuremath{\left(#1\right.}}
\providecommand{\rbrak}[1]{\ensuremath{\left.#1\right)}}
\providecommand{\cbrak}[1]{\ensuremath{\left\{#1\right\}}}
\providecommand{\lcbrak}[1]{\ensuremath{\left\{#1\right.}}
\providecommand{\rcbrak}[1]{\ensuremath{\left.#1\right\}}}
\theoremstyle{remark}
\newtheorem{rem}{Remark}
\newcommand{\sgn}{\mathop{\mathrm{sgn}}}
\newcommand{\rect}{\mathop{\mathrm{rect}}}
\newcommand{\sinc}{\mathop{\mathrm{sinc}}}
\providecommand{\abs}[1]{\left\vert#1\right\vert}
\providecommand{\res}[1]{\Res\displaylimits_{#1}} 
\providecommand{\norm}[1]{\left\Vert#1\right\Vert}
\providecommand{\mtx}[1]{\mathbf{#1}}
\providecommand{\mean}[1]{E\left[ #1 \right]}
\providecommand{\fourier}{\overset{\mathcal{F}}{ \rightleftharpoons}}
\providecommand{\ztrans}{\overset{\mathcal{Z}}{ \rightleftharpoons}}
\providecommand{\system}[1]{\overset{\mathcal{#1}}{ \longleftrightarrow}}
\newcommand{\solution}{\noindent \textbf{Solution: }}
\providecommand{\dec}[2]{\ensuremath{\overset{#1}{\underset{#2}{\gtrless}}}}
\let\StandardTheFigure\thefigure
\def\putbox#1#2#3{\makebox[0in][l]{\makebox[#1][l]{}\raisebox{\baselineskip}[0in][0in]{\raisebox{#2}[0in][0in]{#3}}}}
     \def\rightbox#1{\makebox[0in][r]{#1}}
     \def\centbox#1{\makebox[0in]{#1}}
     \def\topbox#1{\raisebox{-\baselineskip}[0in][0in]{#1}}
     \def\midbox#1{\raisebox{-0.5\baselineskip}[0in][0in]{#1}}

\vspace{3cm}
\title{Line Assignment}
\author{Gautam Singh}
\maketitle
\bigskip

\begin{abstract}
    This document contains the solution to Question 24 of Exercise 4 
    in Chapter 10 of the class 11 NCERT textbook.
\end{abstract}

\begin{enumerate}
\fi
		We first find the coordinates of the intersection of \eqref{eq:chapters/11/10/4/24/L1}
    and \eqref{eq:chapters/11/10/4/24/L2}. Using the augmented matrix and row reduction methods,
    \begin{align}
        \myaugvec{2}{2&-3&-4\\3&4&5} &\xleftrightarrow[]{R_2\rightarrow2R_2-3R_1} 
        \myaugvec{2}{2&-3&-4\\0&17&22} \\
                      &\xleftrightarrow[]{R_1\rightarrow17R_1+3R_2} \myaugvec{2}{17&0&-1\\0&17&22} \\
                      &\xleftrightarrow[]{\substack{R_1\rightarrow\frac{R_1}{17}\\R_2\rightarrow\frac{R_2}{17}}} \myaugvec{2}{1&0&-\frac{1}{17}\\0&1&\frac{22}{17}}
        \label{eq:chapters/11/10/4/24/intersect}
    \end{align}
    the intersection of the lines is
    \begin{align}
        \vec{A} = \frac{1}{17}\myvec{-1\\22}
    \end{align}
    Clearly, the man should follow the path perpendicular to \eqref{eq:chapters/11/10/4/24/L3} from
    $\vec{A}$ to reach it in the shortest time. The normal vector 
    of \eqref{eq:chapters/11/10/4/24/L3} is 
    \begin{align}
        \vec{m} = \myvec{6\\-7}
        \label{eq:chapters/11/10/4/24/L3-norm}
    \end{align}
    which is consequently the direction vector of the required line. Therefore, 
    the required normal vector is given by
    \begin{align}
        \vec{n} = \myvec{7\\6}
        \label{eq:chapters/11/10/4/24/L4-norm}
    \end{align}
    and hence, the equation of the line is
   \begin{align}
        \vec{n}^\top\vec{x} &= \vec{n}^\top\vec{A} \\
        \implies \myvec{7&6}\vec{x} &= \frac{1}{17}\myvec{7&6}\myvec{-1\\22} = \frac{125}{17}
        \label{eq:chapters/11/10/4/24/L4}
    \end{align}
		See Fig. \ref{fig:chapters/11/10/4/24/crossing}. In this figure $\vec{F}$ represents 
    the foot of the prependicular drawn from $\vec{A}$ onto \eqref{eq:chapters/11/10/4/24/L3}.

	\item Prove that the products of the lengths of the perpendiculars drawn from the points $\myvec{\sqrt{a^2-b^2}\\0}$ and $\myvec{-\sqrt{a^2-b^2} \\0} $ to the line $\frac{x}{a} \cos{\theta} + \frac{y}{b}\sin{\theta} =1 $ is $ b^2 $.
\\
    \solution 
		\input{chapters/11/10/4/23/dist.tex}
\item Find the equation of line  drawn perpendicular to the line $\frac{x}{4}+\frac{y}{6}=1$ through the point where it meets the y-axis \\
\solution
		\iffalse
\documentclass[journal,12pt,twocolumn]{IEEEtran}
\usepackage{setspace}
\usepackage{gensymb}
\usepackage{xcolor}
\usepackage{caption}
\singlespacing
\usepackage{siunitx}
\usepackage[cmex10]{amsmath}
\usepackage{mathtools}
\usepackage{hyperref}
\usepackage{amsthm}
\usepackage{mathrsfs}
\usepackage{txfonts}
\usepackage{stfloats}
\usepackage{cite}
\usepackage{cases}
\usepackage{subfig}
\usepackage{longtable}
\usepackage{multirow}
\usepackage{enumitem}
\usepackage{mathtools}
\usepackage{listings}
\usepackage{tikz}
\usetikzlibrary{shapes,arrows,positioning}
\usepackage{circuitikz}
\let\vec\mathbf
\DeclareMathOperator*{\Res}{Res}
\renewcommand\thesection{\arabic{section}}
\renewcommand\thesubsection{\thesection.\arabic{subsection}}
\renewcommand\thesubsubsection{\thesubsection.\arabic{subsubsection}}

\renewcommand\thesectiondis{\arabic{section}}
\renewcommand\thesubsectiondis{\thesectiondis.\arabic{subsection}}
\renewcommand\thesubsubsectiondis{\thesubsectiondis.\arabic{subsubsection}}
\hyphenation{op-tical net-works semi-conduc-tor}

\lstset{
language=Python,
frame=single, 
breaklines=true,
columns=fullflexible
}
\begin{document}
\theoremstyle{definition}
\newtheorem{theorem}{Theorem}[section]
\newtheorem{problem}{Problem}
\newtheorem{proposition}{Proposition}[section]
\newtheorem{lemma}{Lemma}[section]
\newtheorem{corollary}[theorem]{Corollary}
\newtheorem{example}{Example}[section]
\newtheorem{definition}{Definition}[section]
\newcommand{\BEQA}{\begin{eqnarray}}
\newcommand{\EEQA}{\end{eqnarray}}
\newcommand{\define}{\stackrel{\triangle}{=}}
\newenvironment{amatrix}[1]{%
  \left(\begin{array}{@{}*{#1}{c}|c@{}}
}{%
  \end{array}\right)
}
\newcommand{\myvec}[1]{\ensuremath{\begin{pmatrix}#1\end{pmatrix}}}
\newcommand{\myaugvec}[2]{\ensuremath{\begin{amatrix}{#1}#2\end{amatrix}}}
\newcommand{\mydet}[1]{\ensuremath{\begin{vmatrix}#1\end{vmatrix}}}
\bibliographystyle{IEEEtran}
\providecommand{\nCr}[2]{\,^{#1}C_{#2}} % nCr
\providecommand{\nPr}[2]{\,^{#1}P_{#2}} % nPr
\providecommand{\mbf}{\mathbf}
\providecommand{\pr}[1]{\ensuremath{\Pr\left(#1\right)}}
\providecommand{\qfunc}[1]{\ensuremath{Q\left(#1\right)}}
\providecommand{\sbrak}[1]{\ensuremath{{}\left[#1\right]}}
\providecommand{\lsbrak}[1]{\ensuremath{{}\left[#1\right.}}
\providecommand{\rsbrak}[1]{\ensuremath{{}\left.#1\right]}}
\providecommand{\brak}[1]{\ensuremath{\left(#1\right)}}
\providecommand{\lbrak}[1]{\ensuremath{\left(#1\right.}}
\providecommand{\rbrak}[1]{\ensuremath{\left.#1\right)}}
\providecommand{\cbrak}[1]{\ensuremath{\left\{#1\right\}}}
\providecommand{\lcbrak}[1]{\ensuremath{\left\{#1\right.}}
\providecommand{\rcbrak}[1]{\ensuremath{\left.#1\right\}}}
\theoremstyle{remark}
\newtheorem{rem}{Remark}
\newcommand{\sgn}{\mathop{\mathrm{sgn}}}
\newcommand{\rect}{\mathop{\mathrm{rect}}}
\newcommand{\sinc}{\mathop{\mathrm{sinc}}}
\providecommand{\abs}[1]{\left\vert#1\right\vert}
\providecommand{\res}[1]{\Res\displaylimits_{#1}} 
\providecommand{\norm}[1]{\left\Vert#1\right\Vert}
\providecommand{\mtx}[1]{\mathbf{#1}}
\providecommand{\mean}[1]{E\left[ #1 \right]}
\providecommand{\fourier}{\overset{\mathcal{F}}{ \rightleftharpoons}}
\providecommand{\ztrans}{\overset{\mathcal{Z}}{ \rightleftharpoons}}
\providecommand{\system}[1]{\overset{\mathcal{#1}}{ \longleftrightarrow}}
\newcommand{\solution}{\noindent \textbf{Solution: }}
\providecommand{\dec}[2]{\ensuremath{\overset{#1}{\underset{#2}{\gtrless}}}}
\let\StandardTheFigure\thefigure
\def\putbox#1#2#3{\makebox[0in][l]{\makebox[#1][l]{}\raisebox{\baselineskip}[0in][0in]{\raisebox{#2}[0in][0in]{#3}}}}
     \def\rightbox#1{\makebox[0in][r]{#1}}
     \def\centbox#1{\makebox[0in]{#1}}
     \def\topbox#1{\raisebox{-\baselineskip}[0in][0in]{#1}}
     \def\midbox#1{\raisebox{-0.5\baselineskip}[0in][0in]{#1}}

\vspace{3cm}
\title{Line Assignment}
\author{Gautam Singh}
\maketitle
\bigskip

\begin{abstract}
    This document contains the solution to Question 24 of Exercise 4 
    in Chapter 10 of the class 11 NCERT textbook.
\end{abstract}

\begin{enumerate}
\fi
		We first find the coordinates of the intersection of \eqref{eq:chapters/11/10/4/24/L1}
    and \eqref{eq:chapters/11/10/4/24/L2}. Using the augmented matrix and row reduction methods,
    \begin{align}
        \myaugvec{2}{2&-3&-4\\3&4&5} &\xleftrightarrow[]{R_2\rightarrow2R_2-3R_1} 
        \myaugvec{2}{2&-3&-4\\0&17&22} \\
                      &\xleftrightarrow[]{R_1\rightarrow17R_1+3R_2} \myaugvec{2}{17&0&-1\\0&17&22} \\
                      &\xleftrightarrow[]{\substack{R_1\rightarrow\frac{R_1}{17}\\R_2\rightarrow\frac{R_2}{17}}} \myaugvec{2}{1&0&-\frac{1}{17}\\0&1&\frac{22}{17}}
        \label{eq:chapters/11/10/4/24/intersect}
    \end{align}
    the intersection of the lines is
    \begin{align}
        \vec{A} = \frac{1}{17}\myvec{-1\\22}
    \end{align}
    Clearly, the man should follow the path perpendicular to \eqref{eq:chapters/11/10/4/24/L3} from
    $\vec{A}$ to reach it in the shortest time. The normal vector 
    of \eqref{eq:chapters/11/10/4/24/L3} is 
    \begin{align}
        \vec{m} = \myvec{6\\-7}
        \label{eq:chapters/11/10/4/24/L3-norm}
    \end{align}
    which is consequently the direction vector of the required line. Therefore, 
    the required normal vector is given by
    \begin{align}
        \vec{n} = \myvec{7\\6}
        \label{eq:chapters/11/10/4/24/L4-norm}
    \end{align}
    and hence, the equation of the line is
   \begin{align}
        \vec{n}^\top\vec{x} &= \vec{n}^\top\vec{A} \\
        \implies \myvec{7&6}\vec{x} &= \frac{1}{17}\myvec{7&6}\myvec{-1\\22} = \frac{125}{17}
        \label{eq:chapters/11/10/4/24/L4}
    \end{align}
		See Fig. \ref{fig:chapters/11/10/4/24/crossing}. In this figure $\vec{F}$ represents 
    the foot of the prependicular drawn from $\vec{A}$ onto \eqref{eq:chapters/11/10/4/24/L3}.

 \item  In each of the following cases, determine the direction cosines of the normal to
the plane and the distance from the origin.
\begin{enumerate}
	\item $z=2$ 
	\item $x + y + z = 1$
	\item $2x + 3y – z = 5$
	\item $5y + 8 = 0$
\end{enumerate}
    \solution
		\input{chapters/12/11/3/1/dist.tex}
\item
Find the angle between the lines whose direction ratios are $a,b,c$ and $b-c,c-a,a-b$.

\textbf{Solution :}
    \begin{align}
    \vec{m _1} &= \myvec{a\\b\\c}\\
    \vec{m_2} &= \myvec{b-c\\c-a\\a-b}\\
    \cos{\theta}&= \frac{\vec{m_1}^{\top}\vec{m_2}}{\vec{\norm{m_1}\norm{m_2}}
   } \\
   &=\frac{\myvec{a&b&c}\myvec{b-c\\c-a\\a-b}}{\sqrt{a^2+b^2+c^2}\sqrt{\brak{b-c}^2+\brak{c-a}^2+\brak{a-b}^2}}\\
   &=0\\
   or,\theta&=\frac{\pi}{2}
    \end{align}

\end{enumerate}

\item Find the distance of the point $(-1,1)$ from the line $12\brak{x+6} = 5\brak{y-2}$. 
\label{chapters/11/10/3/4}
\iffalse
\documentclass[journal,12pt,twocolumn]{IEEEtran}
\usepackage{setspace}
\usepackage{gensymb}
\usepackage{xcolor}
\usepackage{caption}
\singlespacing
\usepackage{siunitx}
\usepackage[cmex10]{amsmath}
\usepackage{mathtools}
\usepackage{hyperref}
\usepackage{amsthm}
\usepackage{mathrsfs}
\usepackage{txfonts}
\usepackage{stfloats}
\usepackage{cite}
\usepackage{cases}
\usepackage{subfig}
\usepackage{longtable}
\usepackage{multirow}
\usepackage{enumitem}
\usepackage{mathtools}
\usepackage{listings}
\usepackage{tikz}
\usetikzlibrary{shapes,arrows,positioning}
\usepackage{circuitikz}
\let\vec\mathbf
\DeclareMathOperator*{\Res}{Res}
\renewcommand\thesection{\arabic{section}}
\renewcommand\thesubsection{\thesection.\arabic{subsection}}
\renewcommand\thesubsubsection{\thesubsection.\arabic{subsubsection}}

\renewcommand\thesectiondis{\arabic{section}}
\renewcommand\thesubsectiondis{\thesectiondis.\arabic{subsection}}
\renewcommand\thesubsubsectiondis{\thesubsectiondis.\arabic{subsubsection}}
\hyphenation{op-tical net-works semi-conduc-tor}

\lstset{
language=Python,
frame=single, 
breaklines=true,
columns=fullflexible
}
\begin{document}
\theoremstyle{definition}
\newtheorem{theorem}{Theorem}[section]
\newtheorem{problem}{Problem}
\newtheorem{proposition}{Proposition}[section]
\newtheorem{lemma}{Lemma}[section]
\newtheorem{corollary}[theorem]{Corollary}
\newtheorem{example}{Example}[section]
\newtheorem{definition}{Definition}[section]
\newcommand{\BEQA}{\begin{eqnarray}}
\newcommand{\EEQA}{\end{eqnarray}}
\newcommand{\define}{\stackrel{\triangle}{=}}
\newenvironment{amatrix}[1]{%
  \left(\begin{array}{@{}*{#1}{c}|c@{}}
}{%
  \end{array}\right)
}
\newcommand{\myvec}[1]{\ensuremath{\begin{pmatrix}#1\end{pmatrix}}}
\newcommand{\myaugvec}[2]{\ensuremath{\begin{amatrix}{#1}#2\end{amatrix}}}
\newcommand{\mydet}[1]{\ensuremath{\begin{vmatrix}#1\end{vmatrix}}}
\bibliographystyle{IEEEtran}
\providecommand{\nCr}[2]{\,^{#1}C_{#2}} % nCr
\providecommand{\nPr}[2]{\,^{#1}P_{#2}} % nPr
\providecommand{\mbf}{\mathbf}
\providecommand{\pr}[1]{\ensuremath{\Pr\left(#1\right)}}
\providecommand{\qfunc}[1]{\ensuremath{Q\left(#1\right)}}
\providecommand{\sbrak}[1]{\ensuremath{{}\left[#1\right]}}
\providecommand{\lsbrak}[1]{\ensuremath{{}\left[#1\right.}}
\providecommand{\rsbrak}[1]{\ensuremath{{}\left.#1\right]}}
\providecommand{\brak}[1]{\ensuremath{\left(#1\right)}}
\providecommand{\lbrak}[1]{\ensuremath{\left(#1\right.}}
\providecommand{\rbrak}[1]{\ensuremath{\left.#1\right)}}
\providecommand{\cbrak}[1]{\ensuremath{\left\{#1\right\}}}
\providecommand{\lcbrak}[1]{\ensuremath{\left\{#1\right.}}
\providecommand{\rcbrak}[1]{\ensuremath{\left.#1\right\}}}
\theoremstyle{remark}
\newtheorem{rem}{Remark}
\newcommand{\sgn}{\mathop{\mathrm{sgn}}}
\newcommand{\rect}{\mathop{\mathrm{rect}}}
\newcommand{\sinc}{\mathop{\mathrm{sinc}}}
\providecommand{\abs}[1]{\left\vert#1\right\vert}
\providecommand{\res}[1]{\Res\displaylimits_{#1}} 
\providecommand{\norm}[1]{\left\Vert#1\right\Vert}
\providecommand{\mtx}[1]{\mathbf{#1}}
\providecommand{\mean}[1]{E\left[ #1 \right]}
\providecommand{\fourier}{\overset{\mathcal{F}}{ \rightleftharpoons}}
\providecommand{\ztrans}{\overset{\mathcal{Z}}{ \rightleftharpoons}}
\providecommand{\system}[1]{\overset{\mathcal{#1}}{ \longleftrightarrow}}
\newcommand{\solution}{\noindent \textbf{Solution: }}
\providecommand{\dec}[2]{\ensuremath{\overset{#1}{\underset{#2}{\gtrless}}}}
\let\StandardTheFigure\thefigure
\def\putbox#1#2#3{\makebox[0in][l]{\makebox[#1][l]{}\raisebox{\baselineskip}[0in][0in]{\raisebox{#2}[0in][0in]{#3}}}}
     \def\rightbox#1{\makebox[0in][r]{#1}}
     \def\centbox#1{\makebox[0in]{#1}}
     \def\topbox#1{\raisebox{-\baselineskip}[0in][0in]{#1}}
     \def\midbox#1{\raisebox{-0.5\baselineskip}[0in][0in]{#1}}

\vspace{3cm}
\title{Line Assignment}
\author{Gautam Singh}
\maketitle
\bigskip

\begin{abstract}
    This document contains the solution to Question 24 of Exercise 4 
    in Chapter 10 of the class 11 NCERT textbook.
\end{abstract}

\begin{enumerate}
\fi
		We first find the coordinates of the intersection of \eqref{eq:chapters/11/10/4/24/L1}
    and \eqref{eq:chapters/11/10/4/24/L2}. Using the augmented matrix and row reduction methods,
    \begin{align}
        \myaugvec{2}{2&-3&-4\\3&4&5} &\xleftrightarrow[]{R_2\rightarrow2R_2-3R_1} 
        \myaugvec{2}{2&-3&-4\\0&17&22} \\
                      &\xleftrightarrow[]{R_1\rightarrow17R_1+3R_2} \myaugvec{2}{17&0&-1\\0&17&22} \\
                      &\xleftrightarrow[]{\substack{R_1\rightarrow\frac{R_1}{17}\\R_2\rightarrow\frac{R_2}{17}}} \myaugvec{2}{1&0&-\frac{1}{17}\\0&1&\frac{22}{17}}
        \label{eq:chapters/11/10/4/24/intersect}
    \end{align}
    the intersection of the lines is
    \begin{align}
        \vec{A} = \frac{1}{17}\myvec{-1\\22}
    \end{align}
    Clearly, the man should follow the path perpendicular to \eqref{eq:chapters/11/10/4/24/L3} from
    $\vec{A}$ to reach it in the shortest time. The normal vector 
    of \eqref{eq:chapters/11/10/4/24/L3} is 
    \begin{align}
        \vec{m} = \myvec{6\\-7}
        \label{eq:chapters/11/10/4/24/L3-norm}
    \end{align}
    which is consequently the direction vector of the required line. Therefore, 
    the required normal vector is given by
    \begin{align}
        \vec{n} = \myvec{7\\6}
        \label{eq:chapters/11/10/4/24/L4-norm}
    \end{align}
    and hence, the equation of the line is
   \begin{align}
        \vec{n}^\top\vec{x} &= \vec{n}^\top\vec{A} \\
        \implies \myvec{7&6}\vec{x} &= \frac{1}{17}\myvec{7&6}\myvec{-1\\22} = \frac{125}{17}
        \label{eq:chapters/11/10/4/24/L4}
    \end{align}
		See Fig. \ref{fig:chapters/11/10/4/24/crossing}. In this figure $\vec{F}$ represents 
    the foot of the prependicular drawn from $\vec{A}$ onto \eqref{eq:chapters/11/10/4/24/L3}.

\item Find the points on the x-axis, whose distances from the line $\frac{x}{3}+\frac{y}{4}=1$ are 4 units.
\label{chapters/11/10/3/5}
	\\
	\solution
\iffalse
\documentclass[journal,12pt,twocolumn]{IEEEtran}
\usepackage{setspace}
\usepackage{gensymb}
\usepackage{xcolor}
\usepackage{caption}
\singlespacing
\usepackage{siunitx}
\usepackage[cmex10]{amsmath}
\usepackage{mathtools}
\usepackage{hyperref}
\usepackage{amsthm}
\usepackage{mathrsfs}
\usepackage{txfonts}
\usepackage{stfloats}
\usepackage{cite}
\usepackage{cases}
\usepackage{subfig}
\usepackage{longtable}
\usepackage{multirow}
\usepackage{enumitem}
\usepackage{mathtools}
\usepackage{listings}
\usepackage{tikz}
\usetikzlibrary{shapes,arrows,positioning}
\usepackage{circuitikz}
\let\vec\mathbf
\DeclareMathOperator*{\Res}{Res}
\renewcommand\thesection{\arabic{section}}
\renewcommand\thesubsection{\thesection.\arabic{subsection}}
\renewcommand\thesubsubsection{\thesubsection.\arabic{subsubsection}}

\renewcommand\thesectiondis{\arabic{section}}
\renewcommand\thesubsectiondis{\thesectiondis.\arabic{subsection}}
\renewcommand\thesubsubsectiondis{\thesubsectiondis.\arabic{subsubsection}}
\hyphenation{op-tical net-works semi-conduc-tor}

\lstset{
language=Python,
frame=single, 
breaklines=true,
columns=fullflexible
}
\begin{document}
\theoremstyle{definition}
\newtheorem{theorem}{Theorem}[section]
\newtheorem{problem}{Problem}
\newtheorem{proposition}{Proposition}[section]
\newtheorem{lemma}{Lemma}[section]
\newtheorem{corollary}[theorem]{Corollary}
\newtheorem{example}{Example}[section]
\newtheorem{definition}{Definition}[section]
\newcommand{\BEQA}{\begin{eqnarray}}
\newcommand{\EEQA}{\end{eqnarray}}
\newcommand{\define}{\stackrel{\triangle}{=}}
\newenvironment{amatrix}[1]{%
  \left(\begin{array}{@{}*{#1}{c}|c@{}}
}{%
  \end{array}\right)
}
\newcommand{\myvec}[1]{\ensuremath{\begin{pmatrix}#1\end{pmatrix}}}
\newcommand{\myaugvec}[2]{\ensuremath{\begin{amatrix}{#1}#2\end{amatrix}}}
\newcommand{\mydet}[1]{\ensuremath{\begin{vmatrix}#1\end{vmatrix}}}
\bibliographystyle{IEEEtran}
\providecommand{\nCr}[2]{\,^{#1}C_{#2}} % nCr
\providecommand{\nPr}[2]{\,^{#1}P_{#2}} % nPr
\providecommand{\mbf}{\mathbf}
\providecommand{\pr}[1]{\ensuremath{\Pr\left(#1\right)}}
\providecommand{\qfunc}[1]{\ensuremath{Q\left(#1\right)}}
\providecommand{\sbrak}[1]{\ensuremath{{}\left[#1\right]}}
\providecommand{\lsbrak}[1]{\ensuremath{{}\left[#1\right.}}
\providecommand{\rsbrak}[1]{\ensuremath{{}\left.#1\right]}}
\providecommand{\brak}[1]{\ensuremath{\left(#1\right)}}
\providecommand{\lbrak}[1]{\ensuremath{\left(#1\right.}}
\providecommand{\rbrak}[1]{\ensuremath{\left.#1\right)}}
\providecommand{\cbrak}[1]{\ensuremath{\left\{#1\right\}}}
\providecommand{\lcbrak}[1]{\ensuremath{\left\{#1\right.}}
\providecommand{\rcbrak}[1]{\ensuremath{\left.#1\right\}}}
\theoremstyle{remark}
\newtheorem{rem}{Remark}
\newcommand{\sgn}{\mathop{\mathrm{sgn}}}
\newcommand{\rect}{\mathop{\mathrm{rect}}}
\newcommand{\sinc}{\mathop{\mathrm{sinc}}}
\providecommand{\abs}[1]{\left\vert#1\right\vert}
\providecommand{\res}[1]{\Res\displaylimits_{#1}} 
\providecommand{\norm}[1]{\left\Vert#1\right\Vert}
\providecommand{\mtx}[1]{\mathbf{#1}}
\providecommand{\mean}[1]{E\left[ #1 \right]}
\providecommand{\fourier}{\overset{\mathcal{F}}{ \rightleftharpoons}}
\providecommand{\ztrans}{\overset{\mathcal{Z}}{ \rightleftharpoons}}
\providecommand{\system}[1]{\overset{\mathcal{#1}}{ \longleftrightarrow}}
\newcommand{\solution}{\noindent \textbf{Solution: }}
\providecommand{\dec}[2]{\ensuremath{\overset{#1}{\underset{#2}{\gtrless}}}}
\let\StandardTheFigure\thefigure
\def\putbox#1#2#3{\makebox[0in][l]{\makebox[#1][l]{}\raisebox{\baselineskip}[0in][0in]{\raisebox{#2}[0in][0in]{#3}}}}
     \def\rightbox#1{\makebox[0in][r]{#1}}
     \def\centbox#1{\makebox[0in]{#1}}
     \def\topbox#1{\raisebox{-\baselineskip}[0in][0in]{#1}}
     \def\midbox#1{\raisebox{-0.5\baselineskip}[0in][0in]{#1}}

\vspace{3cm}
\title{Line Assignment}
\author{Gautam Singh}
\maketitle
\bigskip

\begin{abstract}
    This document contains the solution to Question 24 of Exercise 4 
    in Chapter 10 of the class 11 NCERT textbook.
\end{abstract}

\begin{enumerate}
\fi
		We first find the coordinates of the intersection of \eqref{eq:chapters/11/10/4/24/L1}
    and \eqref{eq:chapters/11/10/4/24/L2}. Using the augmented matrix and row reduction methods,
    \begin{align}
        \myaugvec{2}{2&-3&-4\\3&4&5} &\xleftrightarrow[]{R_2\rightarrow2R_2-3R_1} 
        \myaugvec{2}{2&-3&-4\\0&17&22} \\
                      &\xleftrightarrow[]{R_1\rightarrow17R_1+3R_2} \myaugvec{2}{17&0&-1\\0&17&22} \\
                      &\xleftrightarrow[]{\substack{R_1\rightarrow\frac{R_1}{17}\\R_2\rightarrow\frac{R_2}{17}}} \myaugvec{2}{1&0&-\frac{1}{17}\\0&1&\frac{22}{17}}
        \label{eq:chapters/11/10/4/24/intersect}
    \end{align}
    the intersection of the lines is
    \begin{align}
        \vec{A} = \frac{1}{17}\myvec{-1\\22}
    \end{align}
    Clearly, the man should follow the path perpendicular to \eqref{eq:chapters/11/10/4/24/L3} from
    $\vec{A}$ to reach it in the shortest time. The normal vector 
    of \eqref{eq:chapters/11/10/4/24/L3} is 
    \begin{align}
        \vec{m} = \myvec{6\\-7}
        \label{eq:chapters/11/10/4/24/L3-norm}
    \end{align}
    which is consequently the direction vector of the required line. Therefore, 
    the required normal vector is given by
    \begin{align}
        \vec{n} = \myvec{7\\6}
        \label{eq:chapters/11/10/4/24/L4-norm}
    \end{align}
    and hence, the equation of the line is
   \begin{align}
        \vec{n}^\top\vec{x} &= \vec{n}^\top\vec{A} \\
        \implies \myvec{7&6}\vec{x} &= \frac{1}{17}\myvec{7&6}\myvec{-1\\22} = \frac{125}{17}
        \label{eq:chapters/11/10/4/24/L4}
    \end{align}
		See Fig. \ref{fig:chapters/11/10/4/24/crossing}. In this figure $\vec{F}$ represents 
    the foot of the prependicular drawn from $\vec{A}$ onto \eqref{eq:chapters/11/10/4/24/L3}.

\item Find the distance between parallel lines
\label{chapters/11/10/3/6}
\begin{enumerate}
	\item $15x+8y-34=0$ and  $15x+8y+31=0$ \\
	\item  $l(x+y)+p=0$ and  $l(x+y)-r=0$
\end{enumerate}
	\solution
\input{chapters/11/10/3/6/dist.tex}
\item Find the coordinates of the foot of the perpendicular from $(-1, 3)$ to the line $3x-4y-16=0$.  
\label{chapters/11/10/3/14}
\\
\solution
\iffalse
\documentclass[journal,12pt,twocolumn]{IEEEtran}
\usepackage{setspace}
\usepackage{gensymb}
\usepackage{xcolor}
\usepackage{caption}
\singlespacing
\usepackage{siunitx}
\usepackage[cmex10]{amsmath}
\usepackage{mathtools}
\usepackage{hyperref}
\usepackage{amsthm}
\usepackage{mathrsfs}
\usepackage{txfonts}
\usepackage{stfloats}
\usepackage{cite}
\usepackage{cases}
\usepackage{subfig}
\usepackage{longtable}
\usepackage{multirow}
\usepackage{enumitem}
\usepackage{mathtools}
\usepackage{listings}
\usepackage{tikz}
\usetikzlibrary{shapes,arrows,positioning}
\usepackage{circuitikz}
\let\vec\mathbf
\DeclareMathOperator*{\Res}{Res}
\renewcommand\thesection{\arabic{section}}
\renewcommand\thesubsection{\thesection.\arabic{subsection}}
\renewcommand\thesubsubsection{\thesubsection.\arabic{subsubsection}}

\renewcommand\thesectiondis{\arabic{section}}
\renewcommand\thesubsectiondis{\thesectiondis.\arabic{subsection}}
\renewcommand\thesubsubsectiondis{\thesubsectiondis.\arabic{subsubsection}}
\hyphenation{op-tical net-works semi-conduc-tor}

\lstset{
language=Python,
frame=single, 
breaklines=true,
columns=fullflexible
}
\begin{document}
\theoremstyle{definition}
\newtheorem{theorem}{Theorem}[section]
\newtheorem{problem}{Problem}
\newtheorem{proposition}{Proposition}[section]
\newtheorem{lemma}{Lemma}[section]
\newtheorem{corollary}[theorem]{Corollary}
\newtheorem{example}{Example}[section]
\newtheorem{definition}{Definition}[section]
\newcommand{\BEQA}{\begin{eqnarray}}
\newcommand{\EEQA}{\end{eqnarray}}
\newcommand{\define}{\stackrel{\triangle}{=}}
\newenvironment{amatrix}[1]{%
  \left(\begin{array}{@{}*{#1}{c}|c@{}}
}{%
  \end{array}\right)
}
\newcommand{\myvec}[1]{\ensuremath{\begin{pmatrix}#1\end{pmatrix}}}
\newcommand{\myaugvec}[2]{\ensuremath{\begin{amatrix}{#1}#2\end{amatrix}}}
\newcommand{\mydet}[1]{\ensuremath{\begin{vmatrix}#1\end{vmatrix}}}
\bibliographystyle{IEEEtran}
\providecommand{\nCr}[2]{\,^{#1}C_{#2}} % nCr
\providecommand{\nPr}[2]{\,^{#1}P_{#2}} % nPr
\providecommand{\mbf}{\mathbf}
\providecommand{\pr}[1]{\ensuremath{\Pr\left(#1\right)}}
\providecommand{\qfunc}[1]{\ensuremath{Q\left(#1\right)}}
\providecommand{\sbrak}[1]{\ensuremath{{}\left[#1\right]}}
\providecommand{\lsbrak}[1]{\ensuremath{{}\left[#1\right.}}
\providecommand{\rsbrak}[1]{\ensuremath{{}\left.#1\right]}}
\providecommand{\brak}[1]{\ensuremath{\left(#1\right)}}
\providecommand{\lbrak}[1]{\ensuremath{\left(#1\right.}}
\providecommand{\rbrak}[1]{\ensuremath{\left.#1\right)}}
\providecommand{\cbrak}[1]{\ensuremath{\left\{#1\right\}}}
\providecommand{\lcbrak}[1]{\ensuremath{\left\{#1\right.}}
\providecommand{\rcbrak}[1]{\ensuremath{\left.#1\right\}}}
\theoremstyle{remark}
\newtheorem{rem}{Remark}
\newcommand{\sgn}{\mathop{\mathrm{sgn}}}
\newcommand{\rect}{\mathop{\mathrm{rect}}}
\newcommand{\sinc}{\mathop{\mathrm{sinc}}}
\providecommand{\abs}[1]{\left\vert#1\right\vert}
\providecommand{\res}[1]{\Res\displaylimits_{#1}} 
\providecommand{\norm}[1]{\left\Vert#1\right\Vert}
\providecommand{\mtx}[1]{\mathbf{#1}}
\providecommand{\mean}[1]{E\left[ #1 \right]}
\providecommand{\fourier}{\overset{\mathcal{F}}{ \rightleftharpoons}}
\providecommand{\ztrans}{\overset{\mathcal{Z}}{ \rightleftharpoons}}
\providecommand{\system}[1]{\overset{\mathcal{#1}}{ \longleftrightarrow}}
\newcommand{\solution}{\noindent \textbf{Solution: }}
\providecommand{\dec}[2]{\ensuremath{\overset{#1}{\underset{#2}{\gtrless}}}}
\let\StandardTheFigure\thefigure
\def\putbox#1#2#3{\makebox[0in][l]{\makebox[#1][l]{}\raisebox{\baselineskip}[0in][0in]{\raisebox{#2}[0in][0in]{#3}}}}
     \def\rightbox#1{\makebox[0in][r]{#1}}
     \def\centbox#1{\makebox[0in]{#1}}
     \def\topbox#1{\raisebox{-\baselineskip}[0in][0in]{#1}}
     \def\midbox#1{\raisebox{-0.5\baselineskip}[0in][0in]{#1}}

\vspace{3cm}
\title{Line Assignment}
\author{Gautam Singh}
\maketitle
\bigskip

\begin{abstract}
    This document contains the solution to Question 24 of Exercise 4 
    in Chapter 10 of the class 11 NCERT textbook.
\end{abstract}

\begin{enumerate}
\fi
		We first find the coordinates of the intersection of \eqref{eq:chapters/11/10/4/24/L1}
    and \eqref{eq:chapters/11/10/4/24/L2}. Using the augmented matrix and row reduction methods,
    \begin{align}
        \myaugvec{2}{2&-3&-4\\3&4&5} &\xleftrightarrow[]{R_2\rightarrow2R_2-3R_1} 
        \myaugvec{2}{2&-3&-4\\0&17&22} \\
                      &\xleftrightarrow[]{R_1\rightarrow17R_1+3R_2} \myaugvec{2}{17&0&-1\\0&17&22} \\
                      &\xleftrightarrow[]{\substack{R_1\rightarrow\frac{R_1}{17}\\R_2\rightarrow\frac{R_2}{17}}} \myaugvec{2}{1&0&-\frac{1}{17}\\0&1&\frac{22}{17}}
        \label{eq:chapters/11/10/4/24/intersect}
    \end{align}
    the intersection of the lines is
    \begin{align}
        \vec{A} = \frac{1}{17}\myvec{-1\\22}
    \end{align}
    Clearly, the man should follow the path perpendicular to \eqref{eq:chapters/11/10/4/24/L3} from
    $\vec{A}$ to reach it in the shortest time. The normal vector 
    of \eqref{eq:chapters/11/10/4/24/L3} is 
    \begin{align}
        \vec{m} = \myvec{6\\-7}
        \label{eq:chapters/11/10/4/24/L3-norm}
    \end{align}
    which is consequently the direction vector of the required line. Therefore, 
    the required normal vector is given by
    \begin{align}
        \vec{n} = \myvec{7\\6}
        \label{eq:chapters/11/10/4/24/L4-norm}
    \end{align}
    and hence, the equation of the line is
   \begin{align}
        \vec{n}^\top\vec{x} &= \vec{n}^\top\vec{A} \\
        \implies \myvec{7&6}\vec{x} &= \frac{1}{17}\myvec{7&6}\myvec{-1\\22} = \frac{125}{17}
        \label{eq:chapters/11/10/4/24/L4}
    \end{align}
		See Fig. \ref{fig:chapters/11/10/4/24/crossing}. In this figure $\vec{F}$ represents 
    the foot of the prependicular drawn from $\vec{A}$ onto \eqref{eq:chapters/11/10/4/24/L3}.

\item  If ${p}$ and ${q}$ are the lengths of perpendiculars from the origin to the lines ${x}\cos\theta - {y}\sin\theta =  {k}\cos2\theta$ and ${x}\sec\theta + {y}\cosec\theta = {k}$, respectively, prove that ${p}^2 + 4{q}^2 = {k}^2$
\label{chapters/11/10/3/16}
\\
\solution
\iffalse
\documentclass[journal,12pt,twocolumn]{IEEEtran}
\usepackage{setspace}
\usepackage{gensymb}
\usepackage{xcolor}
\usepackage{caption}
\singlespacing
\usepackage{siunitx}
\usepackage[cmex10]{amsmath}
\usepackage{mathtools}
\usepackage{hyperref}
\usepackage{amsthm}
\usepackage{mathrsfs}
\usepackage{txfonts}
\usepackage{stfloats}
\usepackage{cite}
\usepackage{cases}
\usepackage{subfig}
\usepackage{longtable}
\usepackage{multirow}
\usepackage{enumitem}
\usepackage{mathtools}
\usepackage{listings}
\usepackage{tikz}
\usetikzlibrary{shapes,arrows,positioning}
\usepackage{circuitikz}
\let\vec\mathbf
\DeclareMathOperator*{\Res}{Res}
\renewcommand\thesection{\arabic{section}}
\renewcommand\thesubsection{\thesection.\arabic{subsection}}
\renewcommand\thesubsubsection{\thesubsection.\arabic{subsubsection}}

\renewcommand\thesectiondis{\arabic{section}}
\renewcommand\thesubsectiondis{\thesectiondis.\arabic{subsection}}
\renewcommand\thesubsubsectiondis{\thesubsectiondis.\arabic{subsubsection}}
\hyphenation{op-tical net-works semi-conduc-tor}

\lstset{
language=Python,
frame=single, 
breaklines=true,
columns=fullflexible
}
\begin{document}
\theoremstyle{definition}
\newtheorem{theorem}{Theorem}[section]
\newtheorem{problem}{Problem}
\newtheorem{proposition}{Proposition}[section]
\newtheorem{lemma}{Lemma}[section]
\newtheorem{corollary}[theorem]{Corollary}
\newtheorem{example}{Example}[section]
\newtheorem{definition}{Definition}[section]
\newcommand{\BEQA}{\begin{eqnarray}}
\newcommand{\EEQA}{\end{eqnarray}}
\newcommand{\define}{\stackrel{\triangle}{=}}
\newenvironment{amatrix}[1]{%
  \left(\begin{array}{@{}*{#1}{c}|c@{}}
}{%
  \end{array}\right)
}
\newcommand{\myvec}[1]{\ensuremath{\begin{pmatrix}#1\end{pmatrix}}}
\newcommand{\myaugvec}[2]{\ensuremath{\begin{amatrix}{#1}#2\end{amatrix}}}
\newcommand{\mydet}[1]{\ensuremath{\begin{vmatrix}#1\end{vmatrix}}}
\bibliographystyle{IEEEtran}
\providecommand{\nCr}[2]{\,^{#1}C_{#2}} % nCr
\providecommand{\nPr}[2]{\,^{#1}P_{#2}} % nPr
\providecommand{\mbf}{\mathbf}
\providecommand{\pr}[1]{\ensuremath{\Pr\left(#1\right)}}
\providecommand{\qfunc}[1]{\ensuremath{Q\left(#1\right)}}
\providecommand{\sbrak}[1]{\ensuremath{{}\left[#1\right]}}
\providecommand{\lsbrak}[1]{\ensuremath{{}\left[#1\right.}}
\providecommand{\rsbrak}[1]{\ensuremath{{}\left.#1\right]}}
\providecommand{\brak}[1]{\ensuremath{\left(#1\right)}}
\providecommand{\lbrak}[1]{\ensuremath{\left(#1\right.}}
\providecommand{\rbrak}[1]{\ensuremath{\left.#1\right)}}
\providecommand{\cbrak}[1]{\ensuremath{\left\{#1\right\}}}
\providecommand{\lcbrak}[1]{\ensuremath{\left\{#1\right.}}
\providecommand{\rcbrak}[1]{\ensuremath{\left.#1\right\}}}
\theoremstyle{remark}
\newtheorem{rem}{Remark}
\newcommand{\sgn}{\mathop{\mathrm{sgn}}}
\newcommand{\rect}{\mathop{\mathrm{rect}}}
\newcommand{\sinc}{\mathop{\mathrm{sinc}}}
\providecommand{\abs}[1]{\left\vert#1\right\vert}
\providecommand{\res}[1]{\Res\displaylimits_{#1}} 
\providecommand{\norm}[1]{\left\Vert#1\right\Vert}
\providecommand{\mtx}[1]{\mathbf{#1}}
\providecommand{\mean}[1]{E\left[ #1 \right]}
\providecommand{\fourier}{\overset{\mathcal{F}}{ \rightleftharpoons}}
\providecommand{\ztrans}{\overset{\mathcal{Z}}{ \rightleftharpoons}}
\providecommand{\system}[1]{\overset{\mathcal{#1}}{ \longleftrightarrow}}
\newcommand{\solution}{\noindent \textbf{Solution: }}
\providecommand{\dec}[2]{\ensuremath{\overset{#1}{\underset{#2}{\gtrless}}}}
\let\StandardTheFigure\thefigure
\def\putbox#1#2#3{\makebox[0in][l]{\makebox[#1][l]{}\raisebox{\baselineskip}[0in][0in]{\raisebox{#2}[0in][0in]{#3}}}}
     \def\rightbox#1{\makebox[0in][r]{#1}}
     \def\centbox#1{\makebox[0in]{#1}}
     \def\topbox#1{\raisebox{-\baselineskip}[0in][0in]{#1}}
     \def\midbox#1{\raisebox{-0.5\baselineskip}[0in][0in]{#1}}

\vspace{3cm}
\title{Line Assignment}
\author{Gautam Singh}
\maketitle
\bigskip

\begin{abstract}
    This document contains the solution to Question 24 of Exercise 4 
    in Chapter 10 of the class 11 NCERT textbook.
\end{abstract}

\begin{enumerate}
\fi
		We first find the coordinates of the intersection of \eqref{eq:chapters/11/10/4/24/L1}
    and \eqref{eq:chapters/11/10/4/24/L2}. Using the augmented matrix and row reduction methods,
    \begin{align}
        \myaugvec{2}{2&-3&-4\\3&4&5} &\xleftrightarrow[]{R_2\rightarrow2R_2-3R_1} 
        \myaugvec{2}{2&-3&-4\\0&17&22} \\
                      &\xleftrightarrow[]{R_1\rightarrow17R_1+3R_2} \myaugvec{2}{17&0&-1\\0&17&22} \\
                      &\xleftrightarrow[]{\substack{R_1\rightarrow\frac{R_1}{17}\\R_2\rightarrow\frac{R_2}{17}}} \myaugvec{2}{1&0&-\frac{1}{17}\\0&1&\frac{22}{17}}
        \label{eq:chapters/11/10/4/24/intersect}
    \end{align}
    the intersection of the lines is
    \begin{align}
        \vec{A} = \frac{1}{17}\myvec{-1\\22}
    \end{align}
    Clearly, the man should follow the path perpendicular to \eqref{eq:chapters/11/10/4/24/L3} from
    $\vec{A}$ to reach it in the shortest time. The normal vector 
    of \eqref{eq:chapters/11/10/4/24/L3} is 
    \begin{align}
        \vec{m} = \myvec{6\\-7}
        \label{eq:chapters/11/10/4/24/L3-norm}
    \end{align}
    which is consequently the direction vector of the required line. Therefore, 
    the required normal vector is given by
    \begin{align}
        \vec{n} = \myvec{7\\6}
        \label{eq:chapters/11/10/4/24/L4-norm}
    \end{align}
    and hence, the equation of the line is
   \begin{align}
        \vec{n}^\top\vec{x} &= \vec{n}^\top\vec{A} \\
        \implies \myvec{7&6}\vec{x} &= \frac{1}{17}\myvec{7&6}\myvec{-1\\22} = \frac{125}{17}
        \label{eq:chapters/11/10/4/24/L4}
    \end{align}
		See Fig. \ref{fig:chapters/11/10/4/24/crossing}. In this figure $\vec{F}$ represents 
    the foot of the prependicular drawn from $\vec{A}$ onto \eqref{eq:chapters/11/10/4/24/L3}.

\item In the triangle $ABC$ with vertices $\vec{A} \brak{2, 3}$, $\vec{B} \brak{4, –1}$ and $\vec{C} \brak{1, 2}$, find the equation and length of altitude from the vertex $\vec{A}$.
\label{chapters/11/10/3/17}
\\
\solution
\iffalse
\documentclass[journal,12pt,twocolumn]{IEEEtran}
\usepackage{setspace}
\usepackage{gensymb}
\usepackage{xcolor}
\usepackage{caption}
\singlespacing
\usepackage{siunitx}
\usepackage[cmex10]{amsmath}
\usepackage{mathtools}
\usepackage{hyperref}
\usepackage{amsthm}
\usepackage{mathrsfs}
\usepackage{txfonts}
\usepackage{stfloats}
\usepackage{cite}
\usepackage{cases}
\usepackage{subfig}
\usepackage{longtable}
\usepackage{multirow}
\usepackage{enumitem}
\usepackage{mathtools}
\usepackage{listings}
\usepackage{tikz}
\usetikzlibrary{shapes,arrows,positioning}
\usepackage{circuitikz}
\let\vec\mathbf
\DeclareMathOperator*{\Res}{Res}
\renewcommand\thesection{\arabic{section}}
\renewcommand\thesubsection{\thesection.\arabic{subsection}}
\renewcommand\thesubsubsection{\thesubsection.\arabic{subsubsection}}

\renewcommand\thesectiondis{\arabic{section}}
\renewcommand\thesubsectiondis{\thesectiondis.\arabic{subsection}}
\renewcommand\thesubsubsectiondis{\thesubsectiondis.\arabic{subsubsection}}
\hyphenation{op-tical net-works semi-conduc-tor}

\lstset{
language=Python,
frame=single, 
breaklines=true,
columns=fullflexible
}
\begin{document}
\theoremstyle{definition}
\newtheorem{theorem}{Theorem}[section]
\newtheorem{problem}{Problem}
\newtheorem{proposition}{Proposition}[section]
\newtheorem{lemma}{Lemma}[section]
\newtheorem{corollary}[theorem]{Corollary}
\newtheorem{example}{Example}[section]
\newtheorem{definition}{Definition}[section]
\newcommand{\BEQA}{\begin{eqnarray}}
\newcommand{\EEQA}{\end{eqnarray}}
\newcommand{\define}{\stackrel{\triangle}{=}}
\newenvironment{amatrix}[1]{%
  \left(\begin{array}{@{}*{#1}{c}|c@{}}
}{%
  \end{array}\right)
}
\newcommand{\myvec}[1]{\ensuremath{\begin{pmatrix}#1\end{pmatrix}}}
\newcommand{\myaugvec}[2]{\ensuremath{\begin{amatrix}{#1}#2\end{amatrix}}}
\newcommand{\mydet}[1]{\ensuremath{\begin{vmatrix}#1\end{vmatrix}}}
\bibliographystyle{IEEEtran}
\providecommand{\nCr}[2]{\,^{#1}C_{#2}} % nCr
\providecommand{\nPr}[2]{\,^{#1}P_{#2}} % nPr
\providecommand{\mbf}{\mathbf}
\providecommand{\pr}[1]{\ensuremath{\Pr\left(#1\right)}}
\providecommand{\qfunc}[1]{\ensuremath{Q\left(#1\right)}}
\providecommand{\sbrak}[1]{\ensuremath{{}\left[#1\right]}}
\providecommand{\lsbrak}[1]{\ensuremath{{}\left[#1\right.}}
\providecommand{\rsbrak}[1]{\ensuremath{{}\left.#1\right]}}
\providecommand{\brak}[1]{\ensuremath{\left(#1\right)}}
\providecommand{\lbrak}[1]{\ensuremath{\left(#1\right.}}
\providecommand{\rbrak}[1]{\ensuremath{\left.#1\right)}}
\providecommand{\cbrak}[1]{\ensuremath{\left\{#1\right\}}}
\providecommand{\lcbrak}[1]{\ensuremath{\left\{#1\right.}}
\providecommand{\rcbrak}[1]{\ensuremath{\left.#1\right\}}}
\theoremstyle{remark}
\newtheorem{rem}{Remark}
\newcommand{\sgn}{\mathop{\mathrm{sgn}}}
\newcommand{\rect}{\mathop{\mathrm{rect}}}
\newcommand{\sinc}{\mathop{\mathrm{sinc}}}
\providecommand{\abs}[1]{\left\vert#1\right\vert}
\providecommand{\res}[1]{\Res\displaylimits_{#1}} 
\providecommand{\norm}[1]{\left\Vert#1\right\Vert}
\providecommand{\mtx}[1]{\mathbf{#1}}
\providecommand{\mean}[1]{E\left[ #1 \right]}
\providecommand{\fourier}{\overset{\mathcal{F}}{ \rightleftharpoons}}
\providecommand{\ztrans}{\overset{\mathcal{Z}}{ \rightleftharpoons}}
\providecommand{\system}[1]{\overset{\mathcal{#1}}{ \longleftrightarrow}}
\newcommand{\solution}{\noindent \textbf{Solution: }}
\providecommand{\dec}[2]{\ensuremath{\overset{#1}{\underset{#2}{\gtrless}}}}
\let\StandardTheFigure\thefigure
\def\putbox#1#2#3{\makebox[0in][l]{\makebox[#1][l]{}\raisebox{\baselineskip}[0in][0in]{\raisebox{#2}[0in][0in]{#3}}}}
     \def\rightbox#1{\makebox[0in][r]{#1}}
     \def\centbox#1{\makebox[0in]{#1}}
     \def\topbox#1{\raisebox{-\baselineskip}[0in][0in]{#1}}
     \def\midbox#1{\raisebox{-0.5\baselineskip}[0in][0in]{#1}}

\vspace{3cm}
\title{Line Assignment}
\author{Gautam Singh}
\maketitle
\bigskip

\begin{abstract}
    This document contains the solution to Question 24 of Exercise 4 
    in Chapter 10 of the class 11 NCERT textbook.
\end{abstract}

\begin{enumerate}
\fi
		We first find the coordinates of the intersection of \eqref{eq:chapters/11/10/4/24/L1}
    and \eqref{eq:chapters/11/10/4/24/L2}. Using the augmented matrix and row reduction methods,
    \begin{align}
        \myaugvec{2}{2&-3&-4\\3&4&5} &\xleftrightarrow[]{R_2\rightarrow2R_2-3R_1} 
        \myaugvec{2}{2&-3&-4\\0&17&22} \\
                      &\xleftrightarrow[]{R_1\rightarrow17R_1+3R_2} \myaugvec{2}{17&0&-1\\0&17&22} \\
                      &\xleftrightarrow[]{\substack{R_1\rightarrow\frac{R_1}{17}\\R_2\rightarrow\frac{R_2}{17}}} \myaugvec{2}{1&0&-\frac{1}{17}\\0&1&\frac{22}{17}}
        \label{eq:chapters/11/10/4/24/intersect}
    \end{align}
    the intersection of the lines is
    \begin{align}
        \vec{A} = \frac{1}{17}\myvec{-1\\22}
    \end{align}
    Clearly, the man should follow the path perpendicular to \eqref{eq:chapters/11/10/4/24/L3} from
    $\vec{A}$ to reach it in the shortest time. The normal vector 
    of \eqref{eq:chapters/11/10/4/24/L3} is 
    \begin{align}
        \vec{m} = \myvec{6\\-7}
        \label{eq:chapters/11/10/4/24/L3-norm}
    \end{align}
    which is consequently the direction vector of the required line. Therefore, 
    the required normal vector is given by
    \begin{align}
        \vec{n} = \myvec{7\\6}
        \label{eq:chapters/11/10/4/24/L4-norm}
    \end{align}
    and hence, the equation of the line is
   \begin{align}
        \vec{n}^\top\vec{x} &= \vec{n}^\top\vec{A} \\
        \implies \myvec{7&6}\vec{x} &= \frac{1}{17}\myvec{7&6}\myvec{-1\\22} = \frac{125}{17}
        \label{eq:chapters/11/10/4/24/L4}
    \end{align}
		See Fig. \ref{fig:chapters/11/10/4/24/crossing}. In this figure $\vec{F}$ represents 
    the foot of the prependicular drawn from $\vec{A}$ onto \eqref{eq:chapters/11/10/4/24/L3}.

\item If $p$ is the length of perpendicular from origin to the line whose intercepts on the axes are $a$ and $b$, then show that 
\begin{align}
	\frac{1}{p^2} = \frac{1}{a^2}+ \frac{1}{b^2}
\end{align}
\label{chapters/11/10/3/18}
\input{chapters/11/10/3/18/dist.tex}
\item What are the points on the y-axis whose distance from the line $\frac{x}{3}+\frac{y}{4}=1$ is 4 units.
\\
\solution
		\iffalse
\documentclass[journal,12pt,twocolumn]{IEEEtran}
\usepackage{setspace}
\usepackage{gensymb}
\usepackage{xcolor}
\usepackage{caption}
\singlespacing
\usepackage{siunitx}
\usepackage[cmex10]{amsmath}
\usepackage{mathtools}
\usepackage{hyperref}
\usepackage{amsthm}
\usepackage{mathrsfs}
\usepackage{txfonts}
\usepackage{stfloats}
\usepackage{cite}
\usepackage{cases}
\usepackage{subfig}
\usepackage{longtable}
\usepackage{multirow}
\usepackage{enumitem}
\usepackage{mathtools}
\usepackage{listings}
\usepackage{tikz}
\usetikzlibrary{shapes,arrows,positioning}
\usepackage{circuitikz}
\let\vec\mathbf
\DeclareMathOperator*{\Res}{Res}
\renewcommand\thesection{\arabic{section}}
\renewcommand\thesubsection{\thesection.\arabic{subsection}}
\renewcommand\thesubsubsection{\thesubsection.\arabic{subsubsection}}

\renewcommand\thesectiondis{\arabic{section}}
\renewcommand\thesubsectiondis{\thesectiondis.\arabic{subsection}}
\renewcommand\thesubsubsectiondis{\thesubsectiondis.\arabic{subsubsection}}
\hyphenation{op-tical net-works semi-conduc-tor}

\lstset{
language=Python,
frame=single, 
breaklines=true,
columns=fullflexible
}
\begin{document}
\theoremstyle{definition}
\newtheorem{theorem}{Theorem}[section]
\newtheorem{problem}{Problem}
\newtheorem{proposition}{Proposition}[section]
\newtheorem{lemma}{Lemma}[section]
\newtheorem{corollary}[theorem]{Corollary}
\newtheorem{example}{Example}[section]
\newtheorem{definition}{Definition}[section]
\newcommand{\BEQA}{\begin{eqnarray}}
\newcommand{\EEQA}{\end{eqnarray}}
\newcommand{\define}{\stackrel{\triangle}{=}}
\newenvironment{amatrix}[1]{%
  \left(\begin{array}{@{}*{#1}{c}|c@{}}
}{%
  \end{array}\right)
}
\newcommand{\myvec}[1]{\ensuremath{\begin{pmatrix}#1\end{pmatrix}}}
\newcommand{\myaugvec}[2]{\ensuremath{\begin{amatrix}{#1}#2\end{amatrix}}}
\newcommand{\mydet}[1]{\ensuremath{\begin{vmatrix}#1\end{vmatrix}}}
\bibliographystyle{IEEEtran}
\providecommand{\nCr}[2]{\,^{#1}C_{#2}} % nCr
\providecommand{\nPr}[2]{\,^{#1}P_{#2}} % nPr
\providecommand{\mbf}{\mathbf}
\providecommand{\pr}[1]{\ensuremath{\Pr\left(#1\right)}}
\providecommand{\qfunc}[1]{\ensuremath{Q\left(#1\right)}}
\providecommand{\sbrak}[1]{\ensuremath{{}\left[#1\right]}}
\providecommand{\lsbrak}[1]{\ensuremath{{}\left[#1\right.}}
\providecommand{\rsbrak}[1]{\ensuremath{{}\left.#1\right]}}
\providecommand{\brak}[1]{\ensuremath{\left(#1\right)}}
\providecommand{\lbrak}[1]{\ensuremath{\left(#1\right.}}
\providecommand{\rbrak}[1]{\ensuremath{\left.#1\right)}}
\providecommand{\cbrak}[1]{\ensuremath{\left\{#1\right\}}}
\providecommand{\lcbrak}[1]{\ensuremath{\left\{#1\right.}}
\providecommand{\rcbrak}[1]{\ensuremath{\left.#1\right\}}}
\theoremstyle{remark}
\newtheorem{rem}{Remark}
\newcommand{\sgn}{\mathop{\mathrm{sgn}}}
\newcommand{\rect}{\mathop{\mathrm{rect}}}
\newcommand{\sinc}{\mathop{\mathrm{sinc}}}
\providecommand{\abs}[1]{\left\vert#1\right\vert}
\providecommand{\res}[1]{\Res\displaylimits_{#1}} 
\providecommand{\norm}[1]{\left\Vert#1\right\Vert}
\providecommand{\mtx}[1]{\mathbf{#1}}
\providecommand{\mean}[1]{E\left[ #1 \right]}
\providecommand{\fourier}{\overset{\mathcal{F}}{ \rightleftharpoons}}
\providecommand{\ztrans}{\overset{\mathcal{Z}}{ \rightleftharpoons}}
\providecommand{\system}[1]{\overset{\mathcal{#1}}{ \longleftrightarrow}}
\newcommand{\solution}{\noindent \textbf{Solution: }}
\providecommand{\dec}[2]{\ensuremath{\overset{#1}{\underset{#2}{\gtrless}}}}
\let\StandardTheFigure\thefigure
\def\putbox#1#2#3{\makebox[0in][l]{\makebox[#1][l]{}\raisebox{\baselineskip}[0in][0in]{\raisebox{#2}[0in][0in]{#3}}}}
     \def\rightbox#1{\makebox[0in][r]{#1}}
     \def\centbox#1{\makebox[0in]{#1}}
     \def\topbox#1{\raisebox{-\baselineskip}[0in][0in]{#1}}
     \def\midbox#1{\raisebox{-0.5\baselineskip}[0in][0in]{#1}}

\vspace{3cm}
\title{Line Assignment}
\author{Gautam Singh}
\maketitle
\bigskip

\begin{abstract}
    This document contains the solution to Question 24 of Exercise 4 
    in Chapter 10 of the class 11 NCERT textbook.
\end{abstract}

\begin{enumerate}
\fi
		We first find the coordinates of the intersection of \eqref{eq:chapters/11/10/4/24/L1}
    and \eqref{eq:chapters/11/10/4/24/L2}. Using the augmented matrix and row reduction methods,
    \begin{align}
        \myaugvec{2}{2&-3&-4\\3&4&5} &\xleftrightarrow[]{R_2\rightarrow2R_2-3R_1} 
        \myaugvec{2}{2&-3&-4\\0&17&22} \\
                      &\xleftrightarrow[]{R_1\rightarrow17R_1+3R_2} \myaugvec{2}{17&0&-1\\0&17&22} \\
                      &\xleftrightarrow[]{\substack{R_1\rightarrow\frac{R_1}{17}\\R_2\rightarrow\frac{R_2}{17}}} \myaugvec{2}{1&0&-\frac{1}{17}\\0&1&\frac{22}{17}}
        \label{eq:chapters/11/10/4/24/intersect}
    \end{align}
    the intersection of the lines is
    \begin{align}
        \vec{A} = \frac{1}{17}\myvec{-1\\22}
    \end{align}
    Clearly, the man should follow the path perpendicular to \eqref{eq:chapters/11/10/4/24/L3} from
    $\vec{A}$ to reach it in the shortest time. The normal vector 
    of \eqref{eq:chapters/11/10/4/24/L3} is 
    \begin{align}
        \vec{m} = \myvec{6\\-7}
        \label{eq:chapters/11/10/4/24/L3-norm}
    \end{align}
    which is consequently the direction vector of the required line. Therefore, 
    the required normal vector is given by
    \begin{align}
        \vec{n} = \myvec{7\\6}
        \label{eq:chapters/11/10/4/24/L4-norm}
    \end{align}
    and hence, the equation of the line is
   \begin{align}
        \vec{n}^\top\vec{x} &= \vec{n}^\top\vec{A} \\
        \implies \myvec{7&6}\vec{x} &= \frac{1}{17}\myvec{7&6}\myvec{-1\\22} = \frac{125}{17}
        \label{eq:chapters/11/10/4/24/L4}
    \end{align}
		See Fig. \ref{fig:chapters/11/10/4/24/crossing}. In this figure $\vec{F}$ represents 
    the foot of the prependicular drawn from $\vec{A}$ onto \eqref{eq:chapters/11/10/4/24/L3}.

\item Find perpendicular distance from the origin to the line joining the points$(\cos\theta,\sin\theta)$ and $(\cos\phi,\sin\phi)$.
\\
\solution
		\input{chapters/11/10/4/5/dist.tex}
\item Find the equation of line which is equidistant from parallel lines $9x+6y-7=0$ and $3x+2y+6=0$.
\\
\solution
		\iffalse
\documentclass[journal,12pt,twocolumn]{IEEEtran}
\usepackage{setspace}
\usepackage{gensymb}
\usepackage{xcolor}
\usepackage{caption}
\singlespacing
\usepackage{siunitx}
\usepackage[cmex10]{amsmath}
\usepackage{mathtools}
\usepackage{hyperref}
\usepackage{amsthm}
\usepackage{mathrsfs}
\usepackage{txfonts}
\usepackage{stfloats}
\usepackage{cite}
\usepackage{cases}
\usepackage{subfig}
\usepackage{longtable}
\usepackage{multirow}
\usepackage{enumitem}
\usepackage{mathtools}
\usepackage{listings}
\usepackage{tikz}
\usetikzlibrary{shapes,arrows,positioning}
\usepackage{circuitikz}
\let\vec\mathbf
\DeclareMathOperator*{\Res}{Res}
\renewcommand\thesection{\arabic{section}}
\renewcommand\thesubsection{\thesection.\arabic{subsection}}
\renewcommand\thesubsubsection{\thesubsection.\arabic{subsubsection}}

\renewcommand\thesectiondis{\arabic{section}}
\renewcommand\thesubsectiondis{\thesectiondis.\arabic{subsection}}
\renewcommand\thesubsubsectiondis{\thesubsectiondis.\arabic{subsubsection}}
\hyphenation{op-tical net-works semi-conduc-tor}

\lstset{
language=Python,
frame=single, 
breaklines=true,
columns=fullflexible
}
\begin{document}
\theoremstyle{definition}
\newtheorem{theorem}{Theorem}[section]
\newtheorem{problem}{Problem}
\newtheorem{proposition}{Proposition}[section]
\newtheorem{lemma}{Lemma}[section]
\newtheorem{corollary}[theorem]{Corollary}
\newtheorem{example}{Example}[section]
\newtheorem{definition}{Definition}[section]
\newcommand{\BEQA}{\begin{eqnarray}}
\newcommand{\EEQA}{\end{eqnarray}}
\newcommand{\define}{\stackrel{\triangle}{=}}
\newenvironment{amatrix}[1]{%
  \left(\begin{array}{@{}*{#1}{c}|c@{}}
}{%
  \end{array}\right)
}
\newcommand{\myvec}[1]{\ensuremath{\begin{pmatrix}#1\end{pmatrix}}}
\newcommand{\myaugvec}[2]{\ensuremath{\begin{amatrix}{#1}#2\end{amatrix}}}
\newcommand{\mydet}[1]{\ensuremath{\begin{vmatrix}#1\end{vmatrix}}}
\bibliographystyle{IEEEtran}
\providecommand{\nCr}[2]{\,^{#1}C_{#2}} % nCr
\providecommand{\nPr}[2]{\,^{#1}P_{#2}} % nPr
\providecommand{\mbf}{\mathbf}
\providecommand{\pr}[1]{\ensuremath{\Pr\left(#1\right)}}
\providecommand{\qfunc}[1]{\ensuremath{Q\left(#1\right)}}
\providecommand{\sbrak}[1]{\ensuremath{{}\left[#1\right]}}
\providecommand{\lsbrak}[1]{\ensuremath{{}\left[#1\right.}}
\providecommand{\rsbrak}[1]{\ensuremath{{}\left.#1\right]}}
\providecommand{\brak}[1]{\ensuremath{\left(#1\right)}}
\providecommand{\lbrak}[1]{\ensuremath{\left(#1\right.}}
\providecommand{\rbrak}[1]{\ensuremath{\left.#1\right)}}
\providecommand{\cbrak}[1]{\ensuremath{\left\{#1\right\}}}
\providecommand{\lcbrak}[1]{\ensuremath{\left\{#1\right.}}
\providecommand{\rcbrak}[1]{\ensuremath{\left.#1\right\}}}
\theoremstyle{remark}
\newtheorem{rem}{Remark}
\newcommand{\sgn}{\mathop{\mathrm{sgn}}}
\newcommand{\rect}{\mathop{\mathrm{rect}}}
\newcommand{\sinc}{\mathop{\mathrm{sinc}}}
\providecommand{\abs}[1]{\left\vert#1\right\vert}
\providecommand{\res}[1]{\Res\displaylimits_{#1}} 
\providecommand{\norm}[1]{\left\Vert#1\right\Vert}
\providecommand{\mtx}[1]{\mathbf{#1}}
\providecommand{\mean}[1]{E\left[ #1 \right]}
\providecommand{\fourier}{\overset{\mathcal{F}}{ \rightleftharpoons}}
\providecommand{\ztrans}{\overset{\mathcal{Z}}{ \rightleftharpoons}}
\providecommand{\system}[1]{\overset{\mathcal{#1}}{ \longleftrightarrow}}
\newcommand{\solution}{\noindent \textbf{Solution: }}
\providecommand{\dec}[2]{\ensuremath{\overset{#1}{\underset{#2}{\gtrless}}}}
\let\StandardTheFigure\thefigure
\def\putbox#1#2#3{\makebox[0in][l]{\makebox[#1][l]{}\raisebox{\baselineskip}[0in][0in]{\raisebox{#2}[0in][0in]{#3}}}}
     \def\rightbox#1{\makebox[0in][r]{#1}}
     \def\centbox#1{\makebox[0in]{#1}}
     \def\topbox#1{\raisebox{-\baselineskip}[0in][0in]{#1}}
     \def\midbox#1{\raisebox{-0.5\baselineskip}[0in][0in]{#1}}

\vspace{3cm}
\title{Line Assignment}
\author{Gautam Singh}
\maketitle
\bigskip

\begin{abstract}
    This document contains the solution to Question 24 of Exercise 4 
    in Chapter 10 of the class 11 NCERT textbook.
\end{abstract}

\begin{enumerate}
\fi
		We first find the coordinates of the intersection of \eqref{eq:chapters/11/10/4/24/L1}
    and \eqref{eq:chapters/11/10/4/24/L2}. Using the augmented matrix and row reduction methods,
    \begin{align}
        \myaugvec{2}{2&-3&-4\\3&4&5} &\xleftrightarrow[]{R_2\rightarrow2R_2-3R_1} 
        \myaugvec{2}{2&-3&-4\\0&17&22} \\
                      &\xleftrightarrow[]{R_1\rightarrow17R_1+3R_2} \myaugvec{2}{17&0&-1\\0&17&22} \\
                      &\xleftrightarrow[]{\substack{R_1\rightarrow\frac{R_1}{17}\\R_2\rightarrow\frac{R_2}{17}}} \myaugvec{2}{1&0&-\frac{1}{17}\\0&1&\frac{22}{17}}
        \label{eq:chapters/11/10/4/24/intersect}
    \end{align}
    the intersection of the lines is
    \begin{align}
        \vec{A} = \frac{1}{17}\myvec{-1\\22}
    \end{align}
    Clearly, the man should follow the path perpendicular to \eqref{eq:chapters/11/10/4/24/L3} from
    $\vec{A}$ to reach it in the shortest time. The normal vector 
    of \eqref{eq:chapters/11/10/4/24/L3} is 
    \begin{align}
        \vec{m} = \myvec{6\\-7}
        \label{eq:chapters/11/10/4/24/L3-norm}
    \end{align}
    which is consequently the direction vector of the required line. Therefore, 
    the required normal vector is given by
    \begin{align}
        \vec{n} = \myvec{7\\6}
        \label{eq:chapters/11/10/4/24/L4-norm}
    \end{align}
    and hence, the equation of the line is
   \begin{align}
        \vec{n}^\top\vec{x} &= \vec{n}^\top\vec{A} \\
        \implies \myvec{7&6}\vec{x} &= \frac{1}{17}\myvec{7&6}\myvec{-1\\22} = \frac{125}{17}
        \label{eq:chapters/11/10/4/24/L4}
    \end{align}
		See Fig. \ref{fig:chapters/11/10/4/24/crossing}. In this figure $\vec{F}$ represents 
    the foot of the prependicular drawn from $\vec{A}$ onto \eqref{eq:chapters/11/10/4/24/L3}.

	\item Prove that the products of the lengths of the perpendiculars drawn from the points $\myvec{\sqrt{a^2-b^2}\\0}$ and $\myvec{-\sqrt{a^2-b^2} \\0} $ to the line $\frac{x}{a} \cos{\theta} + \frac{y}{b}\sin{\theta} =1 $ is $ b^2 $.
\\
    \solution 
		\input{chapters/11/10/4/23/dist.tex}
\item Find the equation of line  drawn perpendicular to the line $\frac{x}{4}+\frac{y}{6}=1$ through the point where it meets the y-axis \\
\solution
		\iffalse
\documentclass[journal,12pt,twocolumn]{IEEEtran}
\usepackage{setspace}
\usepackage{gensymb}
\usepackage{xcolor}
\usepackage{caption}
\singlespacing
\usepackage{siunitx}
\usepackage[cmex10]{amsmath}
\usepackage{mathtools}
\usepackage{hyperref}
\usepackage{amsthm}
\usepackage{mathrsfs}
\usepackage{txfonts}
\usepackage{stfloats}
\usepackage{cite}
\usepackage{cases}
\usepackage{subfig}
\usepackage{longtable}
\usepackage{multirow}
\usepackage{enumitem}
\usepackage{mathtools}
\usepackage{listings}
\usepackage{tikz}
\usetikzlibrary{shapes,arrows,positioning}
\usepackage{circuitikz}
\let\vec\mathbf
\DeclareMathOperator*{\Res}{Res}
\renewcommand\thesection{\arabic{section}}
\renewcommand\thesubsection{\thesection.\arabic{subsection}}
\renewcommand\thesubsubsection{\thesubsection.\arabic{subsubsection}}

\renewcommand\thesectiondis{\arabic{section}}
\renewcommand\thesubsectiondis{\thesectiondis.\arabic{subsection}}
\renewcommand\thesubsubsectiondis{\thesubsectiondis.\arabic{subsubsection}}
\hyphenation{op-tical net-works semi-conduc-tor}

\lstset{
language=Python,
frame=single, 
breaklines=true,
columns=fullflexible
}
\begin{document}
\theoremstyle{definition}
\newtheorem{theorem}{Theorem}[section]
\newtheorem{problem}{Problem}
\newtheorem{proposition}{Proposition}[section]
\newtheorem{lemma}{Lemma}[section]
\newtheorem{corollary}[theorem]{Corollary}
\newtheorem{example}{Example}[section]
\newtheorem{definition}{Definition}[section]
\newcommand{\BEQA}{\begin{eqnarray}}
\newcommand{\EEQA}{\end{eqnarray}}
\newcommand{\define}{\stackrel{\triangle}{=}}
\newenvironment{amatrix}[1]{%
  \left(\begin{array}{@{}*{#1}{c}|c@{}}
}{%
  \end{array}\right)
}
\newcommand{\myvec}[1]{\ensuremath{\begin{pmatrix}#1\end{pmatrix}}}
\newcommand{\myaugvec}[2]{\ensuremath{\begin{amatrix}{#1}#2\end{amatrix}}}
\newcommand{\mydet}[1]{\ensuremath{\begin{vmatrix}#1\end{vmatrix}}}
\bibliographystyle{IEEEtran}
\providecommand{\nCr}[2]{\,^{#1}C_{#2}} % nCr
\providecommand{\nPr}[2]{\,^{#1}P_{#2}} % nPr
\providecommand{\mbf}{\mathbf}
\providecommand{\pr}[1]{\ensuremath{\Pr\left(#1\right)}}
\providecommand{\qfunc}[1]{\ensuremath{Q\left(#1\right)}}
\providecommand{\sbrak}[1]{\ensuremath{{}\left[#1\right]}}
\providecommand{\lsbrak}[1]{\ensuremath{{}\left[#1\right.}}
\providecommand{\rsbrak}[1]{\ensuremath{{}\left.#1\right]}}
\providecommand{\brak}[1]{\ensuremath{\left(#1\right)}}
\providecommand{\lbrak}[1]{\ensuremath{\left(#1\right.}}
\providecommand{\rbrak}[1]{\ensuremath{\left.#1\right)}}
\providecommand{\cbrak}[1]{\ensuremath{\left\{#1\right\}}}
\providecommand{\lcbrak}[1]{\ensuremath{\left\{#1\right.}}
\providecommand{\rcbrak}[1]{\ensuremath{\left.#1\right\}}}
\theoremstyle{remark}
\newtheorem{rem}{Remark}
\newcommand{\sgn}{\mathop{\mathrm{sgn}}}
\newcommand{\rect}{\mathop{\mathrm{rect}}}
\newcommand{\sinc}{\mathop{\mathrm{sinc}}}
\providecommand{\abs}[1]{\left\vert#1\right\vert}
\providecommand{\res}[1]{\Res\displaylimits_{#1}} 
\providecommand{\norm}[1]{\left\Vert#1\right\Vert}
\providecommand{\mtx}[1]{\mathbf{#1}}
\providecommand{\mean}[1]{E\left[ #1 \right]}
\providecommand{\fourier}{\overset{\mathcal{F}}{ \rightleftharpoons}}
\providecommand{\ztrans}{\overset{\mathcal{Z}}{ \rightleftharpoons}}
\providecommand{\system}[1]{\overset{\mathcal{#1}}{ \longleftrightarrow}}
\newcommand{\solution}{\noindent \textbf{Solution: }}
\providecommand{\dec}[2]{\ensuremath{\overset{#1}{\underset{#2}{\gtrless}}}}
\let\StandardTheFigure\thefigure
\def\putbox#1#2#3{\makebox[0in][l]{\makebox[#1][l]{}\raisebox{\baselineskip}[0in][0in]{\raisebox{#2}[0in][0in]{#3}}}}
     \def\rightbox#1{\makebox[0in][r]{#1}}
     \def\centbox#1{\makebox[0in]{#1}}
     \def\topbox#1{\raisebox{-\baselineskip}[0in][0in]{#1}}
     \def\midbox#1{\raisebox{-0.5\baselineskip}[0in][0in]{#1}}

\vspace{3cm}
\title{Line Assignment}
\author{Gautam Singh}
\maketitle
\bigskip

\begin{abstract}
    This document contains the solution to Question 24 of Exercise 4 
    in Chapter 10 of the class 11 NCERT textbook.
\end{abstract}

\begin{enumerate}
\fi
		We first find the coordinates of the intersection of \eqref{eq:chapters/11/10/4/24/L1}
    and \eqref{eq:chapters/11/10/4/24/L2}. Using the augmented matrix and row reduction methods,
    \begin{align}
        \myaugvec{2}{2&-3&-4\\3&4&5} &\xleftrightarrow[]{R_2\rightarrow2R_2-3R_1} 
        \myaugvec{2}{2&-3&-4\\0&17&22} \\
                      &\xleftrightarrow[]{R_1\rightarrow17R_1+3R_2} \myaugvec{2}{17&0&-1\\0&17&22} \\
                      &\xleftrightarrow[]{\substack{R_1\rightarrow\frac{R_1}{17}\\R_2\rightarrow\frac{R_2}{17}}} \myaugvec{2}{1&0&-\frac{1}{17}\\0&1&\frac{22}{17}}
        \label{eq:chapters/11/10/4/24/intersect}
    \end{align}
    the intersection of the lines is
    \begin{align}
        \vec{A} = \frac{1}{17}\myvec{-1\\22}
    \end{align}
    Clearly, the man should follow the path perpendicular to \eqref{eq:chapters/11/10/4/24/L3} from
    $\vec{A}$ to reach it in the shortest time. The normal vector 
    of \eqref{eq:chapters/11/10/4/24/L3} is 
    \begin{align}
        \vec{m} = \myvec{6\\-7}
        \label{eq:chapters/11/10/4/24/L3-norm}
    \end{align}
    which is consequently the direction vector of the required line. Therefore, 
    the required normal vector is given by
    \begin{align}
        \vec{n} = \myvec{7\\6}
        \label{eq:chapters/11/10/4/24/L4-norm}
    \end{align}
    and hence, the equation of the line is
   \begin{align}
        \vec{n}^\top\vec{x} &= \vec{n}^\top\vec{A} \\
        \implies \myvec{7&6}\vec{x} &= \frac{1}{17}\myvec{7&6}\myvec{-1\\22} = \frac{125}{17}
        \label{eq:chapters/11/10/4/24/L4}
    \end{align}
		See Fig. \ref{fig:chapters/11/10/4/24/crossing}. In this figure $\vec{F}$ represents 
    the foot of the prependicular drawn from $\vec{A}$ onto \eqref{eq:chapters/11/10/4/24/L3}.

 \item  In each of the following cases, determine the direction cosines of the normal to
the plane and the distance from the origin.
\begin{enumerate}
	\item $z=2$ 
	\item $x + y + z = 1$
	\item $2x + 3y – z = 5$
	\item $5y + 8 = 0$
\end{enumerate}
    \solution
		\input{chapters/12/11/3/1/dist.tex}
\item
Find the angle between the lines whose direction ratios are $a,b,c$ and $b-c,c-a,a-b$.

\textbf{Solution :}
    \begin{align}
    \vec{m _1} &= \myvec{a\\b\\c}\\
    \vec{m_2} &= \myvec{b-c\\c-a\\a-b}\\
    \cos{\theta}&= \frac{\vec{m_1}^{\top}\vec{m_2}}{\vec{\norm{m_1}\norm{m_2}}
   } \\
   &=\frac{\myvec{a&b&c}\myvec{b-c\\c-a\\a-b}}{\sqrt{a^2+b^2+c^2}\sqrt{\brak{b-c}^2+\brak{c-a}^2+\brak{a-b}^2}}\\
   &=0\\
   or,\theta&=\frac{\pi}{2}
    \end{align}

\end{enumerate}
