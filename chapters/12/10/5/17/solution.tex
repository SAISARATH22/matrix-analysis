Given,
\begin{align}
	\norm{\vec{a}}=\norm{\vec{b}}=1\label{eq:12/10/5/17/1}
	\\
	\norm{\vec{a}+\vec{b}}=1\label{eq:12/10/5/17/2}
\end{align}
Squaring both sides of \eqref{eq:12/10/5/17/2}  , we get
\begin{align}
	\norm{\vec{a}+\vec{b}}^2=1^2
\\	
	\implies \norm{\vec{a}}^2 + \norm{\vec{b}}^2 + 2\vec{a}^{\top}\vec{b} = 1\label{eq:12/10/5/17/3}	
\end{align}
Substituting \eqref{eq:12/10/5/17/1} in \eqref{eq:12/10/5/17/3}, we get
\\
\begin{align}
	\implies 1+1+2(\norm{\vec{a}}\norm{\vec{b}}\cos{\theta})=1
	\\
	\implies 2+2(\norm{\vec{a}}\norm{\vec{b}}\cos{\theta})=1
        \\
	\implies 2(\norm{\vec{a}}\norm{\vec{b}}\cos{\theta})=-1
	\\
	\implies (\norm{\vec{a}}\norm{\vec{b}}\cos{\theta})=\frac{-1}{2}\label{eq:12/10/5/17/4}
\end{align}
Subtituting \eqref{eq:12/10/5/17/1} in \eqref{eq:12/10/5/17/4}, we get
\begin{align}
	\implies \cos{\theta}=\frac{-1}{2}
	\\
	\implies \theta=\frac{2\pi}{3}
\end{align}
